\documentclass[reqno]{amsart}
%\usepackage[english]{babel}
\usepackage[all,cmtip]{xy}
\usepackage{amssymb,amsmath,mathrsfs,amsthm,amscd,hyperref,amsfonts,graphicx}
\usepackage[all]{xy}
\bibliographystyle{plain}
\oddsidemargin 0mm
\evensidemargin 0mm
\textheight 214mm \textwidth 160mm
\headsep 10mm
\raggedbottom

%\include{formatting}
%\include{macros}


\newtheorem{thm}{Theorem}[section]
\newtheorem{theorem}{Theorem}[section]
%\newtheorem{defn}[thm]{Definition}
\newtheorem{cor}[thm]{Corollary}
\newtheorem{conj}[thm]{Conjecture}
\newtheorem{lem}[thm]{Lemma}
\newtheorem{lemma}[thm]{Lemma}
\newtheorem{prop}[thm]{Proposition}
\newtheorem{proposition}[thm]{Proposition}
\newtheorem{nota}[thm]{Notation}
%\newtheorem{rem}[thm]{Remark}

\theoremstyle{definition}
\newtheorem{rem}[thm]{Remark}
\newtheorem{remark}[thm]{Remark}
\newtheorem{defn}[thm]{Definition}
\renewcommand{\thedefn}{}

\theoremstyle{remark}
\newtheorem{ex}{Example}
\newtheorem{notation}{Notation}
\newtheorem{exs}{Examples}
\renewcommand{\theex}{}
\renewcommand{\thenotation}{}
\renewcommand{\theexs}{}



%\theoremstyle{plain}
%\newtheorem{thm}[equation]{Theorem}
%\newtheorem{prop}[equation]{Proposition}
%\newtheorem{cor}[equation]{Corollary}
%\newtheorem{lemma}[equation]{Lemma}
%\newtheorem{conj}[equation]{Conjecture}

%\theoremstyle{definition}
%\newtheorem{defn}[equation]{Definition}
%\newtheorem{defns}[equation]{Definitions}
%
%\theoremstyle{remark}
%\newtheorem{rem}[equation]{Remark}
%\newtheorem{rems}[equation]{Remarks}
%\newtheorem{exer}[equation]{Exercise}
%\newtheorem{exers}[equation]{Exercises}
%\newtheorem{rem-exer}[equation]{Remark/Exercise}
%\newtheorem{rem-exers}[equation]{Remark/Exercises}
%\newtheorem{ex}[equation]{Example}
%\newtheorem{exs}[equation]{Examples}

\newcommand{\Div}{\operatorname{Div}}
\newcommand{\divi}{\operatorname{div}}
\newcommand{\sep}{\operatorname{sep}}
\newcommand{\XminusT}{(x-T)}
\newcommand{\xminusT}{(x-T)'}
\newcommand{\GG}{\mathcal{G}}
\newcommand{\iso}{\stackrel{\sim}{\rightarrow}}
\newcommand{\val}{\operatorname{val}}
\newcommand{\pr}{\operatorname{pr}}
\newcommand{\rk}{\operatorname{rank}}


\def\bmu{{\boldsymbol\mu}}

\def \PP {{\mathbb P}^1}
\def \ZZ {{\mathbb Z}}
\def  \FF {{\mathbb F}}
\def \car {\mathop {\rm Car}}

\def\genus{\textrm{genus}}
\def\cond{\textrm{cond}}
\def\R{\mathbb{R}}
\def\Gal{\textrm{\upshape Gal}}
\def\ord{\textrm{\upshape ord}}
\def\p{\mathbb{P}}
\def\C{\mathbb{C}}
\def\Q{\mathbb{Q}}
%\def\A{\mathbf{A}}
\def\QQ{\overline{\mathbb{Q}}}
\def\Z{\mathbb{Z}}
\def\m{\mathfrak{m}}
\def\Oo{\mathcal{O}}
\def\F{\mathbb{F}}
\def\Hom{\mathrm{Hom}}
\def \End{\mathrm{End}}
\def\hh{\textrm{H}}
\def\s{\textrm{\upshape Spec}\,}
\def\Br{\textnormal{Br}}
\def\pr{\textnormal{pr}}
\def\Nm{\textnormal{Nm}}
\def\can{\textnormal{can}}
\def\rank{\textnormal{rank}}
\def\Pic{\textnormal{Pic}}
\def\h{\textnormal{H}}
%\def\PGL{\textnormal{PGL}}
\def\torsion{\textnormal{tors}}
\def\tors{\textnormal{tors}}


% provide the letter sha:
\usepackage[OT2,T1]{fontenc}
\DeclareSymbolFont{cyrletters}{OT2}{wncyr}{m}{n}
\DeclareMathSymbol{\sha}{\mathalpha}{cyrletters}{"58}

\def\JJ{\mathcal{J}}
\def\XX{\mathcal{X}}
\def\YY{\mathcal{Y}}
\def\O{\mathcal{O}}
\def\OO{\mathcal{O}}



% abbreviations/alternate names
\def\<{\langle}
\def\>{\rangle}
\def\into{\hookrightarrow}
\def\onto{\twoheadrightarrow}
\def\isoto{\tilde{\to}}
\def\tensor{\otimes}
\def\compose{\circ}
\def\sdp{{\rtimes}}
\def\nodiv{\not|}
\def\PGL{\mathrm{PGL}}
\def\PSL{\mathrm{PSL}}
\def\SL{\mathrm{SL}}
\def\GL{\mathrm{GL}}
\def\Sp{\mathrm{Sp}}
\def\P{\mathbb{P}}

\def\ker{\text{ker}}
\def\im{\text{im}}

\DeclareMathOperator{\res}{Res}
\DeclareMathOperator{\spec}{Spec}

\def\sce{\mathscr{C}^{\textsf{ex}}}
\def\scd{\mathscr{C}}
\def\caff{C_{\textsf{aff}}}
\def\sj{\mathscr{J}}
%\def\sA{\ensuremath{\mathscr{A}}}
\def\sO{\mathcal{O}}
%\def\sO{\ensuremath{\mathscr{O}}}
%\def\sL{\ensuremath{\mathscr{L}}}
%\def\sK{\ensuremath{\mathscr{K}}}
%\def\sF{\ensuremath{\mathscr{F}}}
\def\Pro{\ensuremath{\mathbb{P}}}
%\def\sG{\mathscr{G}}
%\def\sX{\mathscr{X}}
%\DeclareMathOperator{\Supp}{Supp}
%\DeclareMathOperator{\Div}{Div}
%\DeclareMathOperator{\dv}{div}
%\DeclareMathOperator{\Pic}{Pic}
%\DeclareMathOperator{\P0}{Pic^0}
%\DeclareMathOperator{\Spec}{Spec}
\def\isom{\cong}


\begin{document}
\title[Explicit Unbounded Ranks]{Explicit Unbounded Ranks}
\author[Explicit Unbounded Ranks]{L.\ Berger}
\address{}
\email{}
\author[]{C.\ Hall}
\address{}
\email{}
\author[]{R.\ Pannekoek}
\address{}
\email{}
\author[]{J.\ Park}
\address{}
\email{}
\author[]{R.\ Pries}
\address{}
\email{}
\author[]{S.\ Sharif}
\address{}
\email{}
\author[]{A.\ Silverberg}
\address{Department of Mathematics, UC Irvine, Irvine, CA 92697, USA}
\email{asilverb@math.uci.edu}
\author[]{D.\ Ulmer}
\address{School of Mathematics, Georgia Institute of Technology, Atlanta, GA 30332, USA}
\email{ulmer@math.gatech.edu}
\thanks{This material is based upon work supported by the 
National Science Foundation under grants....  We thank Karl Rubin for
help.}

%\date{\today}

\begin{abstract}  
to appear
\end{abstract}


\maketitle

\section{Introduction}

\section{Statements of Main Results}

Let $p$ be an odd prime and let $r \ge 2$ be an integer not divisible by $p$.
Let $C$ be the smooth projective curve over $\F_p(t)$ of genus $g:=r-1$ with affine model
\[y^r=x^{r-1}(x+1)(x+t).\]
Let $J$ be the Jacobian variety of $C$. Then $J$ is an abelian variety
over $\F_p(t)$ of dimension $g=r-1$. The automorphism
$(x,y) \mapsto (x,\zeta_r y)$ of $C$ induces an injection
$\Z[\zeta_r] \hookrightarrow\End(J)$.

When $r=2$, all the results were proved earlier by Ulmer in \cite{Legendre}.

\begin{thm}
\label{fullBSDthm}
The full Conjecture of Birch and Swinnerton-Dyer holds for
$J$ over $\F_q(t^{1/d})$, for all $d$ and for all $q=p^a$. 
\end{thm}


From now on, suppose further that $r$ is a prime divisor of $d$ 
and that $d$ is of the form $d=p^f+1$ for some $f\in\Z^+$.
Let
$$u=t^{1/d}, \qquad
K_d=\F_p(\bmu_d,u).$$

The restriction that $r \mid (p^f+1)$ for some $f$ is 
equivalent to the order of $p$ mod $r$ being even, which is
equivalent to $-1 \in \langle p \rangle \subset (\Z/r\Z)^\times$. 
For each $r$, this gives at least half the primes $p$. 
%Later we'll see a different example without this restriction.




\begin{thm}
\label{rkthm}
\[
\rank_\Z J(K_d) = (r-1)(d-2).
\]
\end{thm}


Let  %let $P_{0,0} := (u, u(u+1)^{d/r})$, and more generally let   
\[
P_{i,j} := (\zeta_d^i u, \zeta_d^{jd/r + i}u(\zeta_d^i u+1)^{d/r}) \in C(K_d)
\]
where $i$ is viewed modulo $d$, and $j$ modulo $r$, %,
%i.e., consider $P_{0,0} = (u, u(u+1)^{d/r})$ and its image under
%the action of the Galois group $\Gal()$.
let $Q_\infty \in C(K_d)$ denote the point at infinity,
and let
\[
D_{i,j} := [P_{i,j}] - [Q_\infty] \in J(K_d).
\]

\begin{thm}
\label{Dijgenthm}
The $dr$ divisors $D_{i,j}$ generate a subgroup of $J(K_d)$ of finite index.
\end{thm}

%\begin{proof}
%
%(Sketch)
%We compute the dimension of the image of $\langle D_{i,j} \rangle$
%under the $(\zeta_r-1)$-descent map (Poonen-Schaefer's 
%$(x-T)$ map):
%\begin{multline*}
%(x-T) : J(K_d)/(\zeta_r-1)J(K_d) \hookrightarrow H^1(K_d,J[\zeta_r-1])
%\\
%\isom \left[ \left( K_d[T]/(T(T+1)(T+t))\right)^\ast/\left(\ldots\right)^r \right]_1
%\isom \left[ \left(K_d^\ast/(K_d^\ast)^r\right)^3 \right]_1
%\end{multline*}
%$$
%(x,y) \in C(K_d) \mapsto (x,x+1,x+t)
%$$
%where $[\,\cdot\,]_1$ denotes the kernel of the weighted norm map
%$$(x,y,z)\mapsto x^{r-1}yz = yz/x \in K_d^\ast/(K_d^\ast)^r.$$
%
%
%\begin{multline*}
%\rank_{\Z[\zeta_r]} J(K_d) = 
%\dim_{\F_r} J(K_d)/(\zeta_r-1) - \dim_{\F_r} J(K_d)_\tors/(\zeta_r-1) \\
%\ge \dim_{\F_r} ((x-T)(\langle D_{i,j} \rangle)) - 2 = d-2
%\end{multline*}
%%A (mostly linear algebra) computation 
%giving
%$$
%\rank_\Z J(K_d) = (r-1)\rank_{\Z[\zeta_r]} J(K_d) \ge (r-1)(d-2).
%$$
%  
%Then
%\begin{align*}
%(r-1)(d-2) & \le \text{rank} \\
%& \le \text{($=$ with BSD) analytic rank} \\
%& \le \text{degree of $L$-function} \\
%& = (r-1)(d-2)
%\end{align*}
%giving a different proof of BSD.
%
%\bigskip 
%
%A sketch of the proof of the last equality is as follows:
%  
%Combining work of Raynaud, Ulmer, Milne, Hall, and others, one gets
%that the degree of the $L$-function is
%$$
%-4\dim(J) + \deg(\cond(J[\ell]))$$
%for any prime $\ell \nmid 2pr$, and
%$$
%\cond(J[\ell]) = 
%\sum_{x\in\P^1}(t_x + 2u_x)[x]
%$$
%where $t_x$ is the dimension of the toric part of the special fiber (over $x$)
%of the N\'eron model of $J$,
%and $u_x$ is the dimension of the unipotent part.
%  
%We compute that the reduction of $J$ at $u=0$ and $u=\infty$
%is totally multiplicative and the reduction at the $d$ places
%$u^d=1$ is half good and half additive.
%Thus,
%$$
%\deg(\cond(J[\ell])) = 
%\sum_{x\in\P^1}(t_x + 2u_x) = 
% (r-1) + (r-1) + d\cdot 2\cdot \frac{r-1}{2} = (r-1)(d+2)
%$$
%so
%$$
%\deg(L\text{-function}) = -4\dim(J) + %\deg(\cond(J[\ell])) 
%\sum(t_x + 2u_x) =  
%-4(r-1)+(r-1)(d+2) 
%= (r-1)(d-2).
%$$
%\end{proof}



Let $o_q(e)$ denote the order of $q$ in $(\Z/e\Z)^\times$.


\begin{thm}
\label{smallerfldthm}
Let $t=o_q(r)$. Then:
\begin{enumerate}
\item[(i)]
$\rank_\Z J(\F_q(\bmu_r,t^{1/d})) = (r-1)\left[\sum_{e\mid d}\frac{\varphi(e)}{o_{q^t}(e)}-
2 \right]$,
\item[(ii)]
$\rank_\Z J(\F_q(t^{1/d})) = {\frac{r-1}{t}}\left[\sum_{e\mid d}\frac{\varphi(e)}{o_{q^t}(e)}-
2 \right]$,
\item[(iii)]
over (fixed) $\F_q(u)$, Jacobians of curves of genus $r-1$ have unbounded rank.
\end{enumerate}
\end{thm}


\begin{thm}
\label{torsionthm}
As $\Z[\zeta_r]$-modules,
\[
J(K_d)_{\textup{tors}} \cong \Z[\zeta_r]/(\zeta_r-1) \times \Z[\zeta_r]/(\zeta_r-1)^2.
\]
\end{thm}

It follows that as abelian groups,
$$
J(K_d)_{\torsion} \cong 
\begin{cases}
(\Z/r\Z)^3 & \text{if $r>2$}, \\
\Z/2\Z \times \Z/4\Z & \text{if $r=2$}.
\end{cases}
$$

In particular,
$$
J(K_d)_{\torsion} =J(K_d)[r^\infty]
$$
and
$$
J(K_d)[\ell] = 0
$$
for all primes $\ell \neq r$.

%\begin{proof}
%
%(Sketch)
%Let $Q_0=(0,0), Q_1=(-1,0), Q_t=(-t,0)$. 
%Then $[Q_i] - [Q_\infty]$
%are $(\zeta_r-1)$-torsion points for $i=0,1,t$, and
%$[Q_0] - [Q_\infty]$ is in the kernel of the $(\zeta_r-1)$-descent map.
%
%We found a divisor $D\in \langle D_{i,j} \rangle$ such that
%$$
%(\zeta_r-1)D \sim [Q_0] - [Q_\infty].
%$$
%
%We show that the $\F_r$-dimension of the image of the known $(\zeta_r-1)^\infty$-torsion 
%under the $(\zeta_r-1)$-descent map is $2$; this shows we have all of it. 
%It's generated over $\Z[\zeta_r]$ by $[Q_1] - [Q_\infty]$ and $D$, so
%$$
%J(K_d)[r] \cong \Z[\zeta_r]/(\zeta_r-1) \times \Z[\zeta_r]/(\zeta_r-1)^2.
%$$
%To show
%$$
%J(K_d)[\ell] = 0
%$$
%for all $\ell \nmid 2pr$, we use the geometry of the N\'eron model and group theory to 
%understand the image of the mod $\ell$ representation
%$$
%\Gal(\bar{\F}_q(t)(J[\ell])/\bar{\F}_q(t)) \hookrightarrow \GL_{2(r-1)}(\F_\ell).
%$$
%
%We show $J(L)[\ell]=0$ for all solvable extensions $L$ of $\bar{\F}_q(t)$.
%
%To show
%$$
%J(K_d)[p] = 0,
%$$
%we show that $J$ is ordinary, i.e.,
%$$
%\# J(\overline{\F_q(t)})[p]=p^{r-1},
%$$
%and calculate the Kodaira-Spencer map to show that
%$$
%J({\F_p(t)}^{sep})[p]=0.
%$$
%
%To show
%$$
%J(K_d)[2] = 0:
%$$
%
%Use that $C$ is isomorphic to the hyperelliptic curve
%\begin{align*}
%y^2 & = x^{2r} - 2(t+1)x^r + t^2-2t+1  \\
%& = (x^r-(u^{d/2}+1)^2)(x^r-(u^{d/2}-1)^2).
%\end{align*}
%\end{proof}



\begin{thm}
\label{decompthm}
If $r>2$, then there is an $(r-1)/2$-dimensional absolutely simple abelian variety $B$ with real multiplication by $\Q(\zeta_r)^+$ such that $J$ is isogenous to $B^2$.
\end{thm}

\begin{thm}
\label{endothm}
If $r>2$, then
\begin{align*}
\End(J) \otimes_\Z \Q &\cong M_2(\Q(\zeta_r)^+) \\
\End_{\bar{\F}_q(t)}(J)\otimes_\Z\Q &\cong \Q(\zeta_r).
\end{align*}
\end{thm}

%\begin{proof}
%
%(Sketch)
%Consider the involution
%$$\sigma : (x,y) \mapsto (-1-\frac{t-1}{x+1},\frac{(t-1)^{2/r}}{y}).$$
%There is an isogeny 
%$$J\sim\ker(\sigma-1) \times\im(\sigma-1) \sim B^2.$$
%%for the involution
%%$$\sigma : (x,y) \mapsto (-1-\frac{v^r}{x+1},\frac{v^2}{y})$$
%%where $v^r=t-1$,
%
%We use group theory to show that $B$ is absolutely simple and
%has endomorphism algebra $\Q(\zeta_r)^+$.
%\end{proof}
%
%
%[We need to reconcile the notation in the next result.]

\begin{thm}
\label{Lfnthm}
Letting $q_1=|\F_q(\bmu_d)|$, then
\begin{align*}
L(J/\F_q(t),s) &= 1,
\\
L(J/K_{d},s) &= (1 - q_1^{1-s})^{(r-1)(d-2)},
\end{align*}
i.e., with $T=q_1^{-s}$ we have
\[
L(T,J/K_d) = (1-q_1T)^{(r-1)(d-2)}\in\Z[T].
\]
\end{thm}


%\begin{proof}
%
%(Sketch)
%This follows since we showed that
%$$
%\deg(L(T,J/K_d)) = (r-1)(d-2)  = \rank_\Z J(K_d) = \text{analytic rank}
%$$
%and we can similarly show that $\deg(L(T,J/\F_q(t)))=0$.
%\end{proof}

\section{Hyperelliptic model and $2$-torsion}


\begin{lem}
The curve $C$ is isomorphic to the hyperelliptic curve
$$
Y^2=X^{2r}-2(t+1)X^r+(t^2-2t+1).
$$
\end{lem}

\begin{proof}
The change of coordinates
$x=(Y+X^r+t+1)/2$ and $y=xX$ transforms
$C$ to the isomorphic curve  
$$Y^2 + 4(t+1)Y + (3t^2+10t+3)=X^{2r}-2(t+1)X^r.$$
Completing the square, via the substitution $Y \mapsto Y - 2(t+1)$, gives 
the hyperelliptic curve
\begin{equation}
\label{HEC}
Y^2=X^{2r}-2(t+1)X^r+(t^2-2t+1) = (X^r-(u^{d/2}+1)^2)(X^r-(u^{d/2}-1)^2)
\end{equation}
where $u=t^{1/d}$.
\end{proof}



\begin{lem}
If $r$ is an odd prime, $q$ is an odd prime power, $d$ is even, and $K_d=\F_q(\bmu_d,u)$,
then $J(K_d)[2]=0$.
\end{lem}

\begin{proof}
Since $d$ is even and $r$ is an odd prime, the two factors on the right hand side of
\eqref{HEC} are irreducible polynomials in $K_d[X]$.
Lemma 12.9 of \cite{ps} implies that if $k$ is a field of odd degree,
$f(X)$ has $n$ distinct irreducible factors over $k$, and at least one 
of these factors has odd degree,
then the $2$-torsion over $k$ on the Jacobian variety of $Y^2=f(X)$ has 
dimension $n-2$. 
Applying this to \eqref{HEC} with $k=K_d$ gives
$J(K_d)[2]=0$.
\end{proof}


\section{Conjecture of Birch and Swinnerton-Dyer}

Recall:

\begin{conj}[BSD I]
%The rank equals the analytic rank:
$$\rank_\Z J(K) = \ord_{s=1}L(J,s)$$
\end{conj}

\begin{conj}[BSD II]
As $s \to 1$,
$$
L(J,s) \sim \frac{R|\sha|\tau}{|J(K)|^2_{\tors}}(s-1)^r
$$
where $r$ is the analytic rank,
$R$ is the regulator, $\sha$ is the Tate-Shafarevich group, and
$\tau$ is the Tamagawa number.
\end{conj}
For function fields it is known that $\rank_\Z J(K) \le \ord_{s=1}L(J,s)$
and that BSD I$\implies$BSD II, by Tate, Milne, \ldots, Kato, Trihan.

\begin{thm}
%\label{fullBSDthm}
The full Conjecture of Birch and Swinnerton-Dyer holds for
$J$ over $\F_q(t^{1/d})$, for all $d$ and for all $q=p^a$. 
\end{thm}

\begin{proof}

[references needed below]
The curve $C$ is isomorphic to $y^r = \frac{(x+1)(x+t)}{x}$.
Let
$$
C_d : \beta^d = \alpha^r - 1,
$$
$$
D_d : \delta^{-d} = \gamma^r - 1.
$$
Then 
$$
\xymatrix{C_d \times D_d~ \ar@{-->}[r]& ~y^r = \displaystyle\frac{(x+1)(x+u^d)}{x}}
$$ 
$$
(\alpha,\beta,\gamma,\delta) \mapsto (x,y,u) = (\alpha^r-1, \alpha\gamma, \beta/\delta)
$$
Thus, the surface
$y^r =(x+1)(x+u^d)/x$ over $\F_q$ 
is dominated by a product of curves $C_d \times D_d$.

The Tate Conjecture for the surface then follows,
and this in turn implies (full) BSD for the
Jacobian of the curve $y^r =(x+1)(x+u^d)/x$ over $\F_q(u)$,
for all $q=p^a$.

This gives (full) BSD for $J$ over $\F_q(t^{1/d})$,
for all $q=p^a$ and all $d$.
\end{proof}

\section{$L$-functions}
%Chris
[Chris will supply this. Below is a place holder.]

Under the given conditions, the degree of the $L$-function is indeed $(r-1)(d-2)$.
Let $L/K$ be a finite extension and $C/k$ be a smooth projective curve with $K=k(C)$.
Let $J\dashrightarrow C$ be the Neron model of $J/L$ and $t_x$ and $u_x$ be the dimensions of, respectively, the toric and unipotent parts of the special fiber of $J\dashrightarrow C$ over a closed point $x$ of $C$.  Let $G_\ell$ be the Galois group of $K(J[\ell])/K$ and $I(x)\subset G_\ell$ be the inertia group of some point/prime over $x$.  If $\ell$ is odd, then the (tame) conductor of $J[\ell]$ is the divisor
$$\cond(J[\ell]) = \sum_x \dim(J[\ell]/J[\ell]^{I(x)})[x] = \sum_x (t_x+2u_x)[x].$$
Theorem 2.12 on p.~190 of Milne's Etale Cohomology together with section 6.2.1 of the attached preprint by Ulmer imply that
$$\deg(L(T,J/L) \mod \ell) = 4\dim(J)(\genus(C)-1) + \deg(\cond(J[\ell])).$$
For us, $C=\P^1$ and $L=K_d$.
If we take $d$ a multiple of $r$, then we can show that 
the reductions at $u=0,\infty$ are totally multiplicative,
and the reduction at the $d$ places $u^d=1$ is half good and half additive.
%All of this is deducible from Chris's old document (e.g. in section 1.3).  
The degree of the $L$-function then becomes
$$\deg(L(T,J/L))
=\deg(L(T,J/L) \mod \ell) =  -4(r-1) + 2(r-1) + d(r-1) = (d-2)(r-1).$$



\section{The descent map}

Let $q$ be a power of a prime $p$ and let $K$ be the rational function field $\F_q(t)$. For each integer $d>1$, define $K_d = K(\zeta_d,t^{1/d})$, where $\zeta_d$ is a primitive $d$-th root of unity, and $u = t^{1/d}$. The field extension $K(\zeta_d) \subset K_d$ is cyclic of degree $d$, for all $d$.

\subsection{A superelliptic curve.}

Choose an odd prime $r$ not dividing $q$. We will consider the smooth projective curve $C$ over $K$ associated to the affine equation
\begin{equation}
\label{equationOfC}
y^r = x^{r-1}(x+1)(x+t),
\end{equation}
whose projective closure in $\p^2_K$ is given by
$$
C' : Y^rZ = X^{r-1}(X+Z)(X+tZ).
$$
The curve $C'$ is non-singular at the unique point at infinity $Q_{\infty} = (0:1:0)$. Choose an integer $\nu$ and set $d = q^\nu+1$. We define
$$
P_{i,j} = \left( \zeta_d^i t^{1/d}, \zeta_d^{jd/r+i} t^{1/d} ( \zeta_d^i t^{1/d}+1 )^{d/r}  \right).
$$ 
for $0 \leq i \leq d-1$ and $0 \leq j \leq r-1$. Now assume that $r$ divides $d$. Then one verifies that $P_i$ is an element of $C(K_d)$ for each $i$. Let $J$ be the Jacobian of $C$.

Using descent, we will prove the following theorem.

\begin{theorem}
\label{subgoal}
The divisor classes $[P_{i,j}] - [Q_{\infty}]$ generate a subgroup of $J(K_d)$ of rank $(r-1)(d-2)$. Moreover, we have $J(K_d)[r^{\infty}] \cong (\Z/r\Z)^3$.
\end{theorem}

\begin{remark}
\upshape
In order for there to exist an integer $\nu$ such that $d = q^\nu+1$ is divisible by $r$, it is necessary and sufficient that $r$ is an odd prime divisor of $q^{\mu}+1$ for some integer $\mu$, and we must have $\nu = \mu \ell$ for some odd integer $\ell$. There are infinitely many $r$ that satisfy this condition, as can be seen by observing that $q^{2^a}+1$ and $q^{2^b}+1$ are coprime integers for all distinct positive integers $a$ and $b$. Since $q^\mu + 1$ divides $q^{\mu \ell}+1$ for any odd integer $\ell$, there exist infinitely many integers $\nu$ such that $d = q^\nu+1$ is divisible by $r$.
\end{remark}

The unique singular point of $C$ is $Q_0 = (0,0)$. The normalization map $C \rightarrow C'$ is a universal homeomorphism; in particular, it is bijective on $\overline{K}$-points. Let $Q_1 = (-1,0)$ and $Q_t = (-t,0)$. We denote $\Delta = \{Q_0,Q_1,Q_t\}$. We consider the covering
$$
\pi : C \rightarrow \p^1
$$
induced by the function $x$. The ramification points of $\pi$ are $Q_0,Q_1,Q_t$ and $Q_{\infty}$, each with ramification index $r$. Applying Riemann--Hurwitz gives that the genus of $C$ is $r - 1$. Note that $C_{K_d}$ has an automorphism given by $(x,y) \mapsto (x,\zeta_d^{d/r} y)$; we denote this automorphism by $\zeta_r$. The automorphism $\zeta_r$ of $C_{K_d}$ induces an automorphism $\zeta_r$ of $J_{K_d}$. The Rosati-involution $\alpha \mapsto \alpha^{\dagger}$ on $\End(J_{K_d})$ sends $\zeta_r$ to its inverse: this simply restates the fact that $\zeta_r$ respects the polarization on $J_{K_d}$, which it does, coming from an automorphism of $C_{K_d}$. We let $\phi : J_{K_d} \rightarrow J_{K_d}$ be the endomorphism $1 - \zeta_r$.

\begin{proposition}
\label{propertiesOfPhi}
The endomorphism $\phi$ is a separable isogeny of degree $r^2$. Its kernel is generated by $[Q_0] - [Q_{\infty}]$ and $[Q_1] - [Q_{\infty}]$.
\end{proposition}
\begin{proof}
Let $g = r-1$ be the genus of $C$. We claim that the endomorphism $(1-\zeta_r)^{r-1}$ and the separable isogeny $[r]:J\rightarrow J$ factor through each other. This follows from the well-known fact from algebraic number theory that the ideal $(r)$ of the Dedekind domain $\Z[\zeta_r]$ decomposes as $(1-\zeta_r)^{r-1}$. It follows that:
$$
\deg(1-\zeta_r)^{r-1} = \deg {[r]} =  r^{2g} = r^{2(r-1)},
$$
which proves that $\deg(1-\zeta_r) = r^2$.

For the final assertion, one easily verifies that the divisor classes $D_0 = [Q_0] - [Q_{\infty}]$ and $D_1 = [Q_1] - [Q_{\infty}]$ are contained in the kernel of $\phi$. To see that the $mD_0 + nD_1$ are distinct elements of $J(K_d)$ for all pairs $(m,n)$ with $m,n \in \{0,1,\ldots,r-1\}$, and hence that $\ker(\phi)$ is generated by $D_0$ and $D_1$, it suffices to show that $x^m(x+1)^n$ is not an $r$-th power in $K_d(C)$ unless $r \mid m$ and $r \mid n$. This is a routine exercise in field theory. 
\end{proof}

\begin{lemma}
\label{rosati}
We have $J[\phi] = J[\phi^{\dagger}]$, as group schemes.
\end{lemma}
\begin{proof}
The equality comes down to the observation that the endomorphisms $\phi = 1-\zeta_r$ and $\phi^\dagger = 1-\zeta_r^{-1}$ factor through each other. But this follows from the fact that $(1-\zeta_r)/(1-\zeta_r^{-1}) \in \Z[\zeta_r]^{\ast}$.
\end{proof}

%Let $\nu$ be such that $r \mid q^{\nu}+1$, and let $d = q^{\nu}+1$. 
%Let $L = K[T]/( T(T+1)(T+t) )$, and let $\Delta = \operatorname{Spec}L$. The underlying set of $\Delta$ can be identified with the set $\{ Q_0, Q_1, Q_t \}$ of those ramification points of $\pi : C \rightarrow \p^1$ that are not at infinity. We denote the base-change $L \otimes_K K_d$ by $L_d$. We will frequently use the identification $L_d = \prod_{Q \in \Delta} K_d$ furnished by the Chinese remainder theorem.

\subsection{Some relations among divisors on $C$}

By $\sim$ we denote linear equivalence in $\Div(C_{K_d})$.

\begin{lemma}
\label{tworels}
We have the following relations in $\Div(C_{K_d})$:
\begin{equation}
\label{divy}
(r+1)Q_\infty \sim (r-1)Q_0 + Q_1 + Q_t,
\end{equation}
\begin{equation}
\label{justsum}
\sum_{i=0}^{d-1} (P_{i,0} - Q_{\infty}) \sim Q_0 - Q_1
\end{equation}
and
\begin{equation}
\label{dougsfind}
\sum_{i=0}^{d-1} (P_{i,0} - P_{i,-i}) \sim Q_0 - Q_\infty.
\end{equation}
\end{lemma}
\begin{proof}Equation (\ref{divy}) follows from considering $\divi(y) \sim 0$. We define $f,g \in K_d(C)$ as follows: $f = y - x(x+1)^{d/r}$ and $g = yx^{d/r-1} - u^{d/r} (x+1)^{d/r}$. Then (\ref{justsum}) follows from considering $\divi(f/x) \sim 0$ and (\ref{dougsfind}) follows from $\divi(f/xg)\sim 0$.
\end{proof}

\begin{lemma}
\label{newtorsion}
Define $D \in \Div(C_{K_d})$ by
$$
D = \sum_{i=0}^{d-1} \sum_{j=0}^{[-1-i]} (P_{i,j} - Q_\infty),
$$
where $[-1-i] \in \{0,\ldots,r-1\}$ is congruent to $-1-i$ modulo $r$, then
$$
(1-\zeta_r)(D) \sim Q_0 - Q_\infty.
$$
Hence the class of $D$ is a $(1-\zeta_r)^2$-torsion element of $J(K_d)$.
\end{lemma}
\begin{proof}
A straightforward calculation shows $(1-\zeta_r)(\sum_{j=0}^{[-1-i]} P_{i,j}) = P_{i,0} - P_{i,-i}$. Hence $(1-\zeta_r)(D) \sim Q_0 - Q_\infty$ follows from (\ref{dougsfind}). The last statement follows from $(1-\zeta_r)[Q_0 - Q_\infty] = 0$, as noted in Lemma \ref{propertiesOfPhi}.
\end{proof}


\subsection{The homomorphism $\XminusT$.}

We will define the pivotal homomorphism
$$
\XminusT : \Div^0(C_{K_d}) \rightarrow \prod_{Q \in \Delta} K_d^{\ast}/K_d^{\ast r}.
$$
Its properties are described in Proposition \ref{propertiesOfCC}.  For an element $v \in \prod_{Q \in \Delta} K_d^{\ast}/K_d^{\ast r}$, we conveniently write $v = (v_0,v_1,v_t)$, where $v_i$ is the coordinate corresponding to $Q_i$.

Let $C_{K_d}^\circ \subset C_{K_d}$ be the complement of $\Delta \cup \{ Q_\infty \}$. We define the homomorphism
$$
\xminusT : \Div(C_{K_d}^\circ) \rightarrow \prod_{Q \in \Delta} K_d^{\ast}/K_d^{\ast r}
$$
by 
$$
P \mapsto \left( x(P) - x(Q) \right)_{Q \in \Delta},
$$
followed by taking the norm if $P$ is defined over a proper field extension of $K_d$. 

We now define the homomorphism 
$$
\XminusT : \Div^0(C_{K_d}) \rightarrow \prod_{Q \in \Delta} K_d^{\ast}/K_d^{\ast r}
$$
as follows: let $D \in \Div^0(C_{K_d})$ be a degree-zero divisor on $C_{K_d}$, then choose $D' \in \Div(C_{K_d}^\circ)$ in such a way that $D$ is linearly equivalent to $D'$. Then set $\XminusT(D) := \xminusT(D')$. For a proof that $\XminusT$ is well-defined, see \cite[6.2.2]{bps}.


\begin{proposition}
\label{propertiesOfCC}
There exists a homomorphism $\alpha : H^1(J[\phi]) \rightarrow \prod_{Q \in \Delta} K_d^{\ast}/K_d^{\ast r}$ such that the following diagram is commutative and its bottom row is exact:
\[
\xymatrix{ 
& \Div^0(C_{K_d}) \ar@{->>}[d] \ar@/^1pc/[ddr]^{\XminusT}  \\\
& J(K_d)/\phi J(K_d) \ar@{^(->}[d]^{\partial} \\\
0 \ar[r] & H^1(J[\phi]) \ar[r]^{\alpha~~~} & \prod_{Q \in \Delta} K_d^{\ast}/K_d^{\ast r} \ar[r]^{~~~~N} & K_d^{\ast}/K_d^{\ast r} \ar[r] & 0
} 
\]
Here $\partial$ is induced by the Galois cohomology coboundary map for the isogeny $\phi$, and $N$ is the map sending $(a_0,a_1,a_t)$ to $a_1a_t/a_0$.
\end{proposition}
\begin{proof}
The proof is an application of the general theory of descent as developed in \cite{bps}. Fix a separable closure $K_d^{\sep}$ of $K_d$, and let $\GG$ be $\Gal(K_d^{\sep}/K_d)$. For a finite $\GG$-module $M$ of cardinality not divisible by $p$, we denote by $M^{\vee}$ the dual $\GG$-module $\Hom(M,K_d^{\sep \ast})$.

Let $E$ be $(\Z/r \Z )^{ \Delta }$, the $\GG$-module of $\Z/r\Z$-valued functions on $\Delta$. Note that the $\GG$-action on $\Delta$ as well as $E$ is trivial. There is a $\GG$-module map $\alpha^{\vee} : E \rightarrow J[\phi]$ 
defined by $h \mapsto \sum_{Q \in \Delta} h(Q) \cdot [Q - Q_{\infty}]$. Proposition \ref{propertiesOfPhi} shows that $\alpha^\vee$ is surjective. Its kernel $R$ is the $\Z/r\Z$-submodule of $E$ generated by the map $\rho$ defined by $Q_0 \mapsto -1, Q_1 \mapsto 1, Q_t \mapsto 1$.
The resulting short exact sequence of $\GG$-modules
\begin{equation}
\label{isacomplex}
0 \rightarrow R \rightarrow  E \stackrel{\alpha^\vee}{\rightarrow} J[\phi] \rightarrow 0
\end{equation}
is split, since it consists of modules that are free as $\Z/r\Z$-modules and have trivial $\GG$-action. Dualizing (\ref{isacomplex}) and taking Galois cohomology, we obtain:
\begin{equation}
\label{embedH1UsingE}
0 \rightarrow H^1(J[\phi^{\dagger}]) \rightarrow H^1(E^{\vee}) \rightarrow H^1(R^\vee) \rightarrow 0.
\end{equation}
By Lemma \ref{rosati}, $H^1(J[\phi^{\dagger}])$ is the same as $H^1(J[\phi])$. We compute that $H^1(E^{\vee}) = H^1(\bmu_r^\Delta) = \prod_{Q \in \Delta} K_d^{\ast}/K_d^{\ast r}$, where the last step is Hilbert 90. Choosing the isomorphism $\Z/r\Z \iso R$ given by $1 \mapsto \rho$, we identify $H^1(R^\vee)$ with $H^1(\bmu_r) = K_d^{\ast}/K_d^{\ast r}$, where the last step is again Hilbert 90. With these identifications, the short exact sequence (\ref{embedH1UsingE}) becomes the bottom row in the diagram. The commutativity of the diagram is Proposition 6.4 in \cite{bps}.

Statement (ii) follows from the exactness of (\ref{embedH1UsingE}).
\end{proof}
%\begin{corollary}
%Let $\f \in K_d(C \times \Delta')$ be some function, and let $D \in \Div^0(C_{K_d}^\f)$ be the principal divisor associated to some function $g \in K_d(C)$. Then $\XminusT^\f(D) = 1$.
%\end{corollary}
%\begin{proof}
%Immediate from the previous proposition and the definition of $\XminusT$.
%\end{proof}
It follows from Proposition \ref{propertiesOfCC} that $\XminusT$ induces a map $J(K_d) \rightarrow \prod_{Q \in \Delta} K_d^{\ast}/K_d^{\ast r}$. We will also denote this map by $\XminusT$. The map $\XminusT$ can be seen as a computation-friendly substitute for the coboundary map $\delta : J(K_d) \rightarrow H^1(J[\phi])$, since we have $\XminusT = \alpha \circ \delta$, where $\alpha$ is an injection.

The rest of this subsection is devoted to the computation of $\XminusT(Q-Q_{\infty})$ for $Q \in \Delta$.

\begin{lemma}
\label{norm}
Let $D \in \Div(C_{K_d}^\circ)$. Then if $\xminusT(D)=(v_0,v_1,v_t)$, we have $v_1v_t/v_0 = v_0^{r-1}v_1v_t = 1$.
\end{lemma}
\begin{proof}
From equation (\ref{equationOfC}) it follows that, if $P \in C_{K_d}^\circ$ is a closed point, then (the norm of) $x(P)^{r-1}(x(P)+1)(x(P)+t)$ is contained in $K_d^{\ast r}$.
\end{proof}

The following lemma states that $\xminusT$ can be ``evaluated on the coordinates on which it makes sense''.

\begin{lemma}
\label{evaluateWhereItMakesSense}
Let $D \in \Div(C_{K_d}^\circ)$ and $D' \in \Div(C_{K_d})$ be linearly equivalent divisors, with $D'$ supported outside of $Q_\infty$. If $Q \in \Delta$ is such that $D'$ is also supported outside of $Q$, then we have
$$
\xminusT(D)_Q = \prod_P (x(P)-x(Q))^{\operatorname{ord}_P(D')}.
$$
\end{lemma}
\begin{proof}
Choose $g \in K_d(C)^{\ast}$ such that $D = D'+\divi(g)$. Observe that $\divi(g)$ is supported outside $Q$ and $Q_\infty$. Then
\begin{align*}
\xminusT(D)_Q & = \prod_P (x(P)-x(Q))^{\operatorname{ord}_P(D)}=\prod_P (x(P)-x(Q))^{\operatorname{ord}_P(D'+\divi(g))} = \\\
 & = \prod_P (x(P)-x(Q))^{\operatorname{ord}_P(D')} \prod_P (x(P)-x(Q))^{\operatorname{ord}_P(g)}.
\end{align*}
In the last expression however, the contribution of the second product is trivial:
$$
\prod_P (x(P)-x(Q))^{\operatorname{ord}_P(g)} = \prod_{P} g(P)^{\operatorname{ord}_P(x-x(Q))} = g(Q)^r g(\infty)^{-r} = 1,
$$
where the first equality is due to Weil reciprocity and the second one rests on the fact that for $Q \in \Delta$ we have $\divi(x-x(Q)) = r \cdot Q - r \cdot Q_\infty$, as is shown by direct calculation.
\end{proof}

For future use, we apply Lemmas \ref{norm} and \ref{evaluateWhereItMakesSense} to the computation of the images under $\XminusT$ of the divisors $Q_1-Q_\infty$ and $P_i - Q_\infty$.

\begin{proposition}
\label{computation}
We have $\XminusT(Q_1-Q_\infty)=(-1,1/(1-t),t-1)$ and $\XminusT(P_{i,j}-Q_\infty) = (\zeta_d^i u,\zeta_d^i u + 1,\zeta_d^i u + t)$.
\end{proposition}
\begin{proof}
For $\bullet \in \{ 0,1,t, \infty \}$, let $D_\bullet \in \Div(C_{K_d}^\circ)$ be a divisor that is linearly equivalent to $Q_\bullet$. Using Lemmas \ref{norm} and \ref{evaluateWhereItMakesSense}, one gets $\xminusT(D_0)=(t,1,t)$, $\xminusT(D_1)=(-1,1/(1-t),t-1)$, and $\xminusT(D_t)= (-t,1-t,t/(t-1))$. Applying (\ref{divy}), we then find $\xminusT(D_\infty) = (1,1,1)$. Hence $\XminusT(Q_1-Q_\infty)=\xminusT(D_1-D_\infty)=(-1,1/(1-t),t-1)$.

Finally, we have $\XminusT(P_{i,j} - Q_\infty) = \XminusT(P_{i,j} - D_\infty) = \xminusT(P_{i,j}) - \xminusT(D_\infty) = (\zeta_d^i u,\zeta_d^i u + 1,\zeta_d^i u + t)$.
\end{proof}

\subsection{The image of $\XminusT$}

Let $N \subset J(K_d)$ be the subgroup generated by the divisor classes $[P_{i,j} - Q_\infty]$, where $i \in \{0,\ldots,d-1\}$ and $j \in \{0,\ldots,r-1\}$. Observe that the known torsion elements $[Q_0 - Q_\infty]$, $[Q_1 - Q_\infty]$, $[Q_t - Q_1]$ and $[D] = [\sum_{i=0}^{d-1} \sum_{j=0}^{[-1-i]} P_{i,j}]$ (the $D$ is as in Lemma \ref{newtorsion}) are all contained in $N$ by Lemmas \ref{tworels} and \ref{newtorsion}. Therefore $N$ contains all elements of $J(K_d)$ described so far. 

\begin{proposition}
\label{dimensiond}
We have $\dim_{\F_r} \XminusT(N) = d$.
\end{proposition}
\begin{proof}
Since $\XminusT(P_{i,j}-Q_\infty) = \XminusT(\zeta_r^j (P_{i,0}-Q_\infty)) = \XminusT(P_{i,0} - Q_\infty)$, the dimension certainly cannot be larger than $d$. To show that it is precisely $d$, we project down from $\prod_{Q \in \Delta} K_d^{\ast} / K_d^{\ast r}$ to a finite-dimensional quotient space of dimension $d$, and conclude by showing that the projection is surjective.

We define the following map:
\begin{align*}
\pr : \prod_{Q \in \Delta} K_d^{\ast} / K_d^{\ast r} & \rightarrow \F_r^{d} \\\
(v_0,v_1,v_t)                                    & \mapsto (\val_{u+1}(v_1),\val_{u+\zeta_d^{-1}}(v_1),\val_{u+\zeta_d^{-2}}(v_1),\ldots,\val_{u+\zeta_d}(v_1))
\end{align*}

By Proposition \ref{computation}, we have $\XminusT(P_{i,j}-Q_\infty) = (\zeta_d^i u,\zeta_d^i u + 1,\zeta_d^i u + t)$. We see that $\pr$ maps the image of $P_{i,j} - Q_\infty$ to the $i$-th basis vector. Hence $\pr$ maps $\XminusT(N)$ surjectively onto $\F_r^d$. This establishes the proposition.
\end{proof}

\begin{lemma}
\label{dimensiontorsion}
The image under $\XminusT$ of the subgroup generated by $[D]$ and $[Q_1-Q_\infty]$ has $\F_r$-dimension $2$.
\end{lemma}
\begin{proof}
Since $\XminusT(P_{i,j}-Q_\infty) = \XminusT(P_{i,0} - Q_\infty)$, as noted in the proof of Proposition \ref{dimensiond}, we see that the image of $D=\sum_{i=0}^{d-1} \sum_{j=0}^{[-1-i]} P_{i,j}$ is the same as that of $\sum_{i=0}^{d-1} (d-i) (P_{i,0} - Q_\infty)$. If we resume the notation of the proof of Proposition \ref{dimensiond}, we find $\pr(\XminusT(D)) = (0,-1,-2,\ldots,-d+1) \in \F_r^d$.

Proposition \ref{computation} gives $\XminusT(Q_1-Q_\infty)=(-1,1/(1-t),t-1)$. Since in $K_d$ we have the factorization $1-t = \prod_{i=0}^{d-1} (1-\zeta_d^i u)$, we get $\pr(\XminusT(Q_1-Q_\infty)) = (-1,-1,-1,\ldots,-1)$. The lemma now follows.
\end{proof}

\subsection{Some algebra. The proof of Theorem \ref{subgoal}}

Since $J(K_d)/\phi J(K_d) = J(K_d) \otimes_{\Z[\zeta_r]} \F_r$, we have a commutative diagram
\[
\xymatrix{
J(K_d) \ar@/_/[dr]^{\XminusT} \ar[r] &  J(K_d) \otimes_{\Z[\zeta_r]} \F_r \ar@{^(->}[d] \\\
& \prod_{Q \in \Delta} K_d^{\ast}/K_d^{\ast r}
}
\]
Let $N$ be a $\Z[\zeta_r]$-submodule of $J(K_d)$. Then the image of $N$ under $\XminusT$ can be identified with the image of the map $N \rightarrow J(K_d) \otimes_{\Z[\zeta_r]} \F_r$. Part (ii) of the following result compares the $\F_r$-dimension of this image with the rank of $N$. Part (i) is particularly useful in pinning down the torsion subgroup $J(K_d)[r^{\infty}]$ of $J(K_d)$.

\begin{proposition}
\label{theAlgebraPart}
Let $R = \Z[\zeta_r]$ and $\phi = 1-\zeta_r$. Let $M$ and $N$ be $R$-modules with $N \subset M$.  
\begin{itemize}\itemsep=0pt
\item[(i)] We have: $M[r^\infty] = M[\phi^\infty] \cong \bigoplus_{i=1}^t R/(\phi^{e_i})$ as $R$-modules, with $t = \dim_{\F_r} M[\phi]$.
\end{itemize}
Let $\rho = \dim_{\Q(\zeta_r)} N \otimes_{\Z} \Q$ be the rank of $N$ as $R$-module, and let $V \subset M \otimes_R \F_r$ be the image of the map $N \rightarrow M \otimes_R \F_r$.
\begin{itemize}\itemsep=0pt
\item[(ii)] We have:
$$
\rho = \dim_{\F_r} V + \dim_{\F_r} (M/N)[\phi] - \dim_{\F_r} M[\phi].
$$
\end{itemize}
%Let $N_0 \subset M[\phi^\infty]$ be an $R$-submodule. Let $V_0 \subset M \otimes_R \F_r$ be the image of the map $N_0 \rightarrow M \otimes_R \F_r$.
%\begin{itemize}\itemsep=0pt
%\item[(iii)] We have: $N_0 = M[\phi^\infty]$ if and only if $\dim_{\F_r} V_0 = \dim_{\F_r} M[\phi]$.
%\end{itemize}
\end{proposition}
\begin{proof}
Since the elements $r$ and $\phi^{r-1}$ of $\Z[\zeta_r]$ generate the same ideal, they differ by a unit, and hence we have $M[r^\infty] = M[\phi^\infty]$. Localizing at the prime ideal $(\phi)$, we find, by the structure theorem for finitely generated modules over principal ideal domains:
$$
M_{(\phi)} \cong R_{(\phi)}^s \oplus \bigoplus_{i=1}^{t} R/(\phi^{e_i}),
$$
for some choice of non-negative integers $s,t$ and $e_i$. Since localizing at $(\phi)$ does not affect $\phi$-power torsion, we find $M[\phi^\infty] \cong \bigoplus_{i=1}^{t} R/(\phi^{e_i})$. From the isomorphism, it is clear that $t = \dim_{\F_r} M[\phi]$. This proves part (i).

For part (ii) of the statement, observe that we have a six-term exact sequence
\begin{equation}
\label{sixterm}
0 \rightarrow N[\phi] \rightarrow M[\phi] \rightarrow (M/N)[\phi] \rightarrow N\otimes_R \F_r \rightarrow M\otimes_R \F_r \rightarrow (M/N) \otimes_R \F_r \rightarrow 0,
\end{equation}
where the middle map sends $m + N$ to $\phi m \otimes 1$. (It is the long exact sequence that results from applying $- \otimes_R \F_r$ to $0 \rightarrow N \rightarrow M \rightarrow M/N \rightarrow 0$, but it is easy to verify the exactness without using this.) Truncating (\ref{sixterm}) at the fifth term, we get the exact sequence
 \begin{equation}
\label{fiveterm}
0 \rightarrow N[\phi] \rightarrow M[\phi] \rightarrow (M/N)[\phi] \rightarrow N\otimes_R \F_r \rightarrow V \rightarrow 0.
\end{equation}
Using
$$
\dim_{\F_r} N \otimes_R \F_r = \dim_{\F_r} N_{(\phi)} \otimes_{R_{(\phi)}} \F_r = \rho + \dim_{\F_r} N[\phi],
$$
and the fact that the $\F_r$-dimensions of the terms of (\ref{fiveterm}) add up to zero, part (ii) follows.
%Now part (iii). Applying (\ref{sixterm}) with $N$ replaced by $M[\phi^\infty]$ shows that
%$$
%M[\phi^\infty] \otimes_R \F_r \rightarrow M \otimes_R \F_r
%$$
%is injective. Hence $\dim_{\F_r} V_0$ equals the $\F_r$-dimension of the image of $N_0$ in $M[\phi^\infty] \otimes_R \F_r$. %Consider 
%$$
%0 \rightarrow N_0 \rightarrow M[\phi^\infty] \rightarrow M[\phi^\infty]/N_0 \rightarrow 0.
%$$
%By right-exactness of tensor products, the map $N_0 \rightarrow M[\phi^\infty]$ is surjective if and only if $N_0  \rightarrow M[\phi^\infty] \otimes_R \F_r$ is, and the latter condition is equivalent to the dimension of the image being $\dim_{\F_r} M[\phi^\infty] \otimes_R \F_r$, which by part (i) equals $\dim_{\F_r} M[\phi]$.
\end{proof}

We are now ready to give the proof of Theorem \ref{subgoal}. 

\begin{proof}
First, we determine $J(K_d)[r^\infty]$. By Propositions \ref{propertiesOfPhi} and \ref{theAlgebraPart}.(i) we find that
$$
J(K_d)[r^\infty] \cong \Z[\zeta_r]/(1-\zeta_r)^{e_1} \oplus \Z[\zeta_r]/(1-\zeta_r)^{e_2}
$$
for some positive integers $e_1,e_2$. Since the image under $\XminusT$ of the subgroup $T \subset J(K_d)[r^\infty]$ generated by $[D]$ and $[Q_1 - Q_\infty]$ has $\F_r$-dimension 2, we must have $T = J(K_d)[r^\infty]$ by Nakayama's Lemma.

Let $N \subset J(K_d)$ be the subgroup generated by the divisor classes $[P_{i,j} - Q_\infty]$, where $i \in \{0,\ldots,d-1\}$ and $j \in \{0,\ldots,r-1\}$. From Proposition \ref{dimensiond} and Proposition \ref{theAlgebraPart}.(ii) we find:
$$
\rk_{\Z[\zeta_r]}(N) = d-2+\dim_{\F_r} (J(K_d)/N)[1-\zeta_r].
$$
It follows from our computations that $N \otimes_{\Z[\zeta_r]} \F_p$ injects into $J(K_d) \otimes_{\Z[\zeta_r]} \F_p$, which implies that $\dim_{\F_r} (J(K_d)/N)[1-\zeta_r]=0$ by (\ref{sixterm}). Therefore, the $\Z$-rank of $N$ is equal to $(r-1)(d-2)$.
\end{proof}



\section{Proof of Theorem \ref{smallerfldthm}}
%\begin{proof}
Recall that $o_q(e)$ denotes the order of $q$ in $(\Z/e\Z)^\times$.

%We know how $\Gal(K_d/\F_q(t^{1/d}))$ acts on the $P_{i,j}$'s
%and we know all the relations among the $P_{i,j}$'s
%(from our rank calculations), so we can compute
%$$
%\langle D_{i,j} \rangle^{\Gal(K_d/\F_q(t^{1/d}))}.
%$$

%More precisely, n
We showed above that $J(K_d) \otimes \Q = \langle D_{\zeta,\rho} \rangle \otimes \Q$.
Let
$V$ denote the free $\Q(\bmu_r)$-vector space with basis
$\{ R_\zeta : \zeta \in \bmu_d \}$, where $R_\zeta$ is a formal symbol.
Then $\dim_{\Q(\bmu_r)} V = d$ and $\dim_{\Q} V = d(r-1)$.
Let $W$ be the subspace of $V$ generated by 
$\sum_{\zeta \in \bmu_d}R_\zeta$ and 
$\sum_{\zeta \in \bmu_d}\zeta^{-d/r}R_\zeta$.
Then $\dim_{\Q(\bmu_r)} W = 2$.
Thus, $\dim_{\Q(\bmu_r)} V/W = d-2$ and $\dim_{\Q} V/W = (d-2)(r-1)$.
Note that $W$ is in the kernel of 
the surjective map from $V$ to $\langle D_{\zeta,\rho} \rangle \otimes \Q$
defined by $R_\zeta \mapsto D_{\zeta,1}$.
Since 
$$
\dim_{\Q} V/W = (d-2)(r-1) \le \dim(\langle D_{\zeta,\rho} \rangle \otimes \Q),
$$
it follows that $W$ is exactly the kernel.
Thus, 
$$
V/W \cong \langle D_{\zeta,\rho} \rangle \otimes \Q = J(K_d) \otimes \Q.
$$

Let $u=t^{1/d}$ and $H = \Gal(K_d/\F_q(\bmu_r,u))$.
Then
$$
J(\F_q(\bmu_r,u)) \otimes \Q =  (J(K_d) \otimes \Q)^H \cong (V/W)^H.
$$
Note that 
$[\F_q(\bmu_r,u): \F_{q}(u)]=t$, 
$\F_q(\bmu_r,u) = \F_{q^t}(u)$,
and $H$ can be identified with the subgroup of $(\Z/d\Z)^\times$
generated by $q^t$.

We have
$$
\dim_{\Q(\bmu_r)} (V/W)^H = \dim_{\Q(\bmu_r)} V^H - \dim_{\Q(\bmu_r)} W^H.
$$
Clearly, $W^H = W$.
Further, $\dim_{\Q(\bmu_r)} V^H$ is the number of orbits of
$\{ R_\zeta \}$ under the action of $H$.
For each $e\mid d$, there are $\frac{\varphi(e)}{o_{q^t}(e)}$
orbits consisting of $R_\zeta$'s with $\zeta$ a primitive $e$-th
root of unity, since there are $o_{q^t}(e)$ of them in each orbit.
Thus, 
$\dim_{\Q(\bmu_r)} V^H = \sum_{e\mid d}\frac{\varphi(e)}{o_{q^t}(e)}$,
so
$$
\rank_\Z J(\F_q(\bmu_r,u)) = \dim_{\Q} (V/W)^H
= (r-1)\dim_{\Q(\bmu_r)} (V/W)^H
= (r-1)\left[\sum_{e\mid d}\frac{\varphi(e)}{o_{q^t}(e)}- 2 \right].
$$
Lemma 13.4 of \cite{ps} says that if $k$ is a global field of characteristic
not $r$, $f(x)$ is an $r$-th power-free polynomial with zeros in $k^{\sep}$,
and $J$ is the Jacobian of $y^r=f(x)$, then
$\rank_\Z J(k) = \rank_\Z J(k(\bmu_r))/[k(\bmu_r): k]$.
Thus,
\begin{equation}
\label{jfuz}
\rank_\Z J(\F_q(u)) = {\frac{r-1}{t}}\left[\sum_{e\mid d}\frac{\varphi(e)}{o_{q^t}(e)}-
2 \right].
\end{equation}


Fixing $q$, $r$, and $d$ with $d=p^f+1$, for all $s\in\Z^+$
let $d_s = p^{fs}+1$, so $r\mid d\mid d_s$.
Apply \eqref{jfuz} to 
the Jacobians of the curves $y^r=x^{r-1}(x+1)(x+u^{d_s})$ to obtain
unbounded rank over $\F_q(u)$ as $s$ grows.
%\end{proof}

\begin{rem}
Note that if $q$ is odd, then 
$$\rank_\Z J(\F_q(\bmu_r,t^{1/d})) = (r-1)\left[\sum_{2<e\mid d}\frac{\varphi(e)}{o_{q^t}(e)} \right]
$$
and
$$
\rank_\Z J(\F_q(t^{1/d})) = {\frac{r-1}{t}}\left[\sum_{2<e\mid d}\frac{\varphi(e)}{o_{q^t}(e)} \right],
$$
summing over the divisors $e$ of $d$ with $e>2$.
In the case of Legendre curves, we have $r=2$ so $t=1$, recovering 
Theorem 12.1(3) from \cite{Legendre}.
\end{rem}


\begin{rem}
Note also that if $r$ were composite, and if we knew that
the rank of $\langle P_{\zeta,\rho} \rangle \otimes \Q$ were 
$\ge (d-2)\varphi(r)$,
then we could replace $r-1$ by $\varphi(r)$ in Theorem \ref{smallerfldthm}.
\end{rem}




%\section{Torsion}


\section{$r$-torsion}
%Doug
[Mostly superseded by an earlier section. Some of the info here can be put in 
the earlier section.]

We work with the curve
$$y^r=x^{r-1}(x+1)(x+t)$$
where $r$ is prime.  Over $K_d$ with $d=q+1$ divisible by $r$, we have points
$$Q_0=(0,0),\quad Q_1=(-1,0),\quad Q_t=(-t,0),$$
the point at infinity $Q_\infty$, and
$$P_{i,j}=\left(\zeta^iu,\zeta^{jd/r+i}u(\zeta^iu+1)^{d/r}\right).$$
Here $u^d=t$, $\zeta$ is a fixed primitive $d$-th root of unity in
$K_d$ and we read $i$ modulo $d$ and $j$ modulo $r$.

We consider the functions $f=y-x(x+1)^{d/r}$,
$g=yx^{d/r-1}-u^{d/r}(x+1)^{d/r}$, and $x$.  Calculating as in
Prop.~3.2 of \cite{Legendre}, 
%(There is a typo there: In the second
%equality of the statement of 3.2, $Q_0$ should be $Q_1$.  In Remark
%3.3, $2\sum...$ should be $Q_0$ not $Q_1$), 
we find that
$$div(f)=\sum_iP_{i,0}+(r-1)Q_0+Q_1-(r+d)Q_\infty$$
$$div(g)=\sum_iP_{i,-i}+Q_1-(d+1)Q_\infty$$
and
$$div(x)=r\left(Q_0-Q_\infty\right).$$
It follows that 
$$div(f/(xg))=\sum_i\left(P_{i,0}-P_{i,-i}\right)-Q_0+Q_\infty.$$

Now I claim that $\sum_i\left(P_{i,0}-P_{i,-i}\right)$ is
$(1-\zeta^{d/r})D$ for a suitable $D$.  To see this, we group the sum
into pieces where $i$ lies in a given class modulo $r$:
$$\sum_{i\equiv -j\pmod r}\left(P_{i,0}-P_{i,-i}\right)
=\sum_{i\equiv -j\pmod r}\left(P_{i,0}-P_{i,j}\right)$$
and then we observe that 
$$(1-\zeta^{d/r})\left(P_{i,0}+P_{i,1}+\cdots+P_{i,j-1}\right)
=P_{i,0}-P_{i,j}.$$

This proves that $Q_0-Q_\infty$ is equivalent to $(1-\zeta^{d/r})D$
for a suitable $D$.

For future use, it might be worth introducing functions
$$f_j=\zeta^{-jd/r}y-x(x+1)^{d/r}$$
and 
$$g_j=\zeta^{-jd/r}yx^{d/r-1}-u^{d/r}(x+1)^{d/r}$$
whose divisors are
$$div(f_j)=\sum_iP_{i,j}+(r-1)Q_0+Q_1-(r+d)Q_\infty$$
and
$$div(g_j)=\sum_iP_{i,-i+j}+Q_1-(d+1)Q_\infty.$$


\section{$\ell$-torsion for $\ell \nmid 2pr$}
%Chris

Notation: $r$ is prime, $\ell$ is a prime not dividing $2pr$, $\lambda$ is a prime of $\Q(\zeta_r)$ above $\ell$, $K={\bar \F_q}(t)$, $K_\lambda = K(J[\lambda])$, and $G=\Gal(K_\lambda/K)$. 

The heart of the proof is showing that $G$ has a subgroup $H$ that is perfect (i.e., $H=[H,H]$) and satisfies $J[\lambda]^H = 0$. (From that, one can easily show that if $L/K$ is abelian then $J(L)[\lambda]=0$.)

Construct $H$ as follows. Using that $K_\lambda/K$ is unramified away from $t=0,1,\infty$ and tamely ramified otherwise, one can show that $G$ is generated by three elements $g_0,g_1,g_\infty$ satisfying $g_0g_1g_\infty=1$ and corresponding to generators of inertia for primes over $t=0,1,\infty$ respectively. Let $H$ be the subgroup of $G$ generated by the conjugates of $g_0$.

To show that $H$ has the desired properties, one uses the injection from $G$ into 
$\GL_2(F_\lambda)$, where $F_\lambda=\Z[\zeta_r]/\lambda$, and views $G$ and $H$ in $\GL_2(F_\lambda)$. Combining $g_0g_1g_\infty=1$ with information about the Jordan forms of $g_0,g_1,g_\infty$ that one can compute using the geometry of the Neron model, one can show that H is an irreducible subgroup of $\GL_2(F_\lambda)$ generated by transvections. The classification of such groups can be used to show that H has the desired properties (perfect and $J[\lambda]^H = 0$).







\section{The curve is generically ordinary}
%Rachel

Let $C$ be the smooth projective curve of genus $g=r-1$ with affine model
\[y^r=x^{r-1}(x+1)(x+t).\]
The purpose of this note is to show that $C$ is ordinary.

To state this more precisely, we view $C$ as a curve over $\FF_q(t)$.  It is a Galois cover of the projective $x$-line over $\FF_q(t)$ with Galois group $\ZZ/r$.
We choose a generator $\delta:(x,y) \mapsto (x, \zeta^{-1} y)$ for the Galois group where $\zeta$ is an $r$th root of unity.
Let $J$ be the Jacobian of $C$.

We say that $C$ and $J$ are {\it ordinary} if the number of $p$-torsion points on $J$ is as large as possible, 
namely $\#J[p](\overline{\FF_q(t)})=p^g$. 
One can also look at this condition as $t$ varies.
Given a geometric point $t$ of the projective line over $t$, let $C_t$ be the fiber of ${\mathcal C}$ above $t$.
Let $J_t$ denote the Jacobian of $C_t$.  
We say that $C_t$ and $J_t$ are {\it ordinary} if the number of $p$-torsion points on $J_t$ is as large as possible, 
namely $\#J_t[p](\overline{\FF}_p)=p^g$. 

We will prove that $C$ is ordinary using an equivalent condition:
$C$ is ordinary iff the Cartier operator $\car$ has rank $g$ on the vector space $H^0(C, \Omega^1_C)$ of regular $1$-forms on $C$.
Some of the properties of the Cartier operator that we will need are:
\[\car(\omega_1 + \omega_2)=\car(\omega_1) + \car(\omega_2);\]
\[\car(\frac{f^p dx}{x})=\frac{f dx}{x};\]
\[\car(\frac{x^i dx}{x})=0 \ {\rm if \ p \nmid i}.\] 

\begin{proposition}
The curve $C$ is ordinary.  Alternatively, 
there is a non-empty Zariski open subset of $\PP_t$ such that $C_t$ is ordinary.
\end{proposition}

\begin{proof}
Let $D_i$ denote the subspace of $H^0(C, \Omega^1_C)$ on which $\delta^*$ acts by multiplication by $\zeta^i$.
There is an eigenspace decomposition
\[H^0(C, \Omega^1_C) \simeq \oplus_{i=0}^{r-1} D_i.\]

One can show that $\omega_i:=x^{i-1}\frac{dx}{y^i}$ is a regular $1$-form and $\delta^*(\omega_i)=\zeta^i \omega_i$ for $1 \leq i \leq r-1$.
In particular, this implies that $D_0=0$ and $D_i = \langle \omega_i \rangle$ has dimension $1$ for $1 \leq i \leq r-1$.
Thus $\Omega=\{\omega_i \mid 1 \leq i \leq r-1\}$ is a basis for $H^0(C, \Omega^1_C)$.
With respect to the basis $\Omega$, we will show that the Cartier operator acts as a permutation matrix scaled by constants which are non-zero 
as long as $t$ does not satisfy certain algebraic equations.

Given $r$, $p$, and $i$ such that $1 \leq i \leq r-1$: 
let $a$ be the unique integer such that $1 \leq a \leq r-1$ and $ap \equiv i \bmod r$;
let $b$ be the unique integer such that $1 \leq b \leq p-1$ and $br \equiv -i \bmod p$.
Using the Chinese remainder theorem, we see that $ap-br=i$.

Then 
\[\car(\omega_i)=\car(x^i \frac{y^{br}}{y^{ap}}\frac{dx}{x})=\frac{1}{y^a}\car(h(x) \frac{dx}{x}),\]
where $h(x)=x^{i+(r-1)b}(x+1)^b(x+t)^b$.
The exponent of each monomial in $h(x)$ is in the range $[i+(r-1)b, i+(r+1)b]=[ap-b, ap+b]$; the only multiple of $p$ in this range is $ap$.
Let $c_i$ be the coefficient of $x^{ap}$ in $h(x)$.
Then $c_i$ is the coefficient of $x^b$ in $(x+1)^b(x+t)^b$.
In other words:
\[c_i=\sum_{j=0}^b {b \choose j}^2 t^j  = 1 + b^2 t + \cdots + b^2t^{b-1} + t^b.\]
It follows that: 
\[\car(\omega_i)=c_i^{1/p} \frac{x^a}{y^a} \frac{dx}{x}= c_i^{1/p} \omega_a.\]
In conclusion, as long as $t$ is not a root of any of the polynomials $c_i(t)$ as $i$ (or $b$) ranges from $1$ to $r-1$, then $\car$ has rank $g$ on $H^0(C, \Omega^1_C)$
and so $C_t$ is ordinary.
\end{proof}






\section{Kodaira-Spencer and $p$-torsion}
%Doug

[Comments in square brackets are editorial and not meant for any
public version.]

In this section, we calculate the Kodaira-Spencer map for the curve
$X/k(t)$ given by $y^r=(x+1)(x+t)/x$ and use it to bound the
$p$-torsion of the Jacobian of $X$.

\subsection{An integral model}\label{s:model}
Throughout $k$ will be field of characteristic $p\ge0$ and $R$ will be
the ring $k[t][1/t,1/(t-1)]$ with field of fractions $K=k(t)$. It is
convenient, although not essential, that $R$ is a PID.  
Let $U=\spec R$, an affine open subset of $\P^1_k$.

Let $r$ be a prime not divisible by $p$ and let $X$ be the smooth
proper curve over $K$ associated to the plane curve
$y^r=x^{r-1}(x+1)(x+t)$. In this section we will write down explicitly
a smooth projective model of $X$ over $U$, i.e., a surface $\XX$
equipped with a smooth projective morphism $\pi:\XX\to U$ with generic
fiber $X$.

Let $\YY$ be the closed subset of $\P^2_R$ defined by
$$Y^rZ=X^{r-1}(X+Z)(X+tZ).$$
A calculation using the Jacobian criterion shows that $\YY\to U$ is
smooth away from the divisor $X=Y=0$. Resolving the singularity at
$X=Y=0$ in each fiber will lead to $\XX\to U$.

Let $\XX_1$ be the affine scheme
$$\spec R[x,y]/\left(y-x^{r-1}(xy+1)(xy+t)\right).$$
By the Jacobian criterion, $\XX_1\to U$ is smooth. The map
$\XX_1\to\YY$ given by $X=xy$, $Y=y$, $Z=1$ is a morphism whose image
is the open subset of $\YY$ where $Z\neq0$ minus the points with
$[X,Y,Z]=[-1,0,1]$ and $[-t,0,1]$. It is an isomorphism away from
$x=y=0$. 
%[$\XX_1$ is the blow up of the open $Z\neq0$ along $X=Y=0$.]

Let $\XX_2$ be the affine scheme
$$\spec R[w,z]/\left(z-w^{r-1}(w+z)(w+tz)\right).$$
By the Jacobian criterion, $\XX_2\to U$ is smooth.  The map
$\XX_2\to\YY$ given by $X=w$, $Y=1$, $Z=z$ is a morphism which maps
$\XX_2$ isomorphically into the open subset $Y\neq0$ of $\YY$.

Let $\XX_3$ be the affine scheme
$$\spec R[u,v]/\left(u^rv-(1+v)(1+tv)\right).$$
By the Jacobian criterion, $\XX_3\to U$ is smooth.  The map
$\XX_3\to\YY$ given by $X=1$, $Y=u$, $Z=v$ is a morphism which maps
$\XX_3$ isomorphically into the open subset $X\neq0$ of $\YY$.

Now we glue $\XX_1$, $\XX_2$ and $\XX_3$ over $U$ via the identifications
$$(x,y)=(w,1/z)\qquad(w,z)=(1/u,v/u)\qquad(u,v)=(1/x,1/(xy))$$
[A more careful exposition would include the domains and ranges.]
and call the result $\XX$, with its projection $\pi:\XX\to U$ to the base
$U$.   Since each of the $\XX_i$ are smooth over $U$, so is $\XX$.
Note also that we have a factorization $\XX\to\YY\to U$ and
$\XX\to\YY$ is a homeomorphism, so $\XX\to U$ is projective.

If the $r$-th roots of unity lie in $k$, then there is an
action of $\bmu_r$ on $\XX$ which is given in coordinates by
$$(x,y)\mapsto(\zeta^{-1} x,\zeta y)\qquad
(w,z)\mapsto(\zeta^{-1}w,\zeta^{-1}z)\qquad
(u,v)\mapsto(\zeta u,v).$$

\subsection{Differentials}
Consider meromorphic 1-forms (sections of $\Omega^1_{\XX/U}$) defined
on $\XX_1$, $\XX_2$, and $\XX_3$ by the formulas
$$\frac{x^{i-1}d(xy)}{y},\qquad
w^{i-1}dw-\frac{w^idz}{z},\quad\text{and}\quad-\frac{dv}{u^iv}$$ 
for $1\le i\le r-1$. It is a simple exercise to check that these
1-forms agree on the overlaps and that the are everywhere regular and
therefore define global 1-forms (global sections of
$\Omega^1_{\XX/U}$) which we denote $\omega_i$. Considering the action
of $\bmu_r$ (or the order of vanishing at points where $x=y=0$) shows
that the restrictions of the $\omega_i$ to any fiber of $\XX\to U$ are
linearly independent. Since the genus of $X$ is $r-1$, we see that the
$\omega_i$ generate $\Omega^1_{\XX/U}$ at every point of $U$.
Equivalently, 
$$H^0(U,\pi_*\Omega^1_{\XX/U})=
H^0(\XX,\Omega^1_{\XX/U})=\bigoplus_{i=1}^{r-1}R\omega_i$$ 
where the latter is the free $R$-module with basis $\{\omega_i\}$.

\subsection{Kodaira-Spencer}\label{s:KS}
Consider the exact sequence of sheaves on $\XX$:
$$0\to\pi^*\Omega^1_U\to\Omega^1_{\XX}\to\Omega^1_{\XX/U}\to 0.$$
Taking the direct image under $\pi$ leads to a morphism
$$KS:\pi_*\Omega^1_{\XX/U}\to R^1\pi_*\O_\XX\tensor_{\OO_U}\Omega^1_{U}$$
which is the ``Kodaira-Spencer map'' of the family $\pi:\XX\to U$.

The main technical point of this fragment is the following.

\begin{prop}\label{prop:KS}
The map $KS$ is an
isomorphism of locally free $\OO_U$-modules of rank $r-1$ on $U$.
\end{prop}

It will be more convenient for us to consider the ``Kodaira-Spencer
pairing'' on global 1-forms 
\begin{align*}
H^0(U,\pi_*\Omega^1_{\XX/U})&\times H^0(U,\pi_*\Omega^1_{\XX/U})&&\to&
H^0(U,\Omega^1_U)=R\,dt\\
\omega_i&\times\omega_j&&\mapsto&(\omega_i,\omega_j)\qquad
\end{align*}
defined by taking the cup product
$$KS(\omega_i)\cup\omega_j
\in R^1\pi_*\Omega^1_{\XX/U}\tensor_{\OO_U}\Omega^1_U$$
followed by the trace
$$ R^1\pi_*\Omega^1_{\XX/U}\tensor_{\OO_U}\Omega^1_U\quad\isoto\quad \Omega^1_U.$$
To show that $KS$ is an isomorphism is the same as to show that the
Kodaira-Spencer pairing is a perfect pairing of free $R$-modules. A
proof that the pairing is perfect will be given in the next two
sections.

\subsection{Lifting 1-forms}
We will compute the Kodaira-Spencer pairing using Cech cohomology for the affine open
cover $\{\XX_1,\XX_2,\XX_3\}$ of $\XX$. The key point is to find for
each $\alpha=1,2,3$ a lift of $\omega_i$ to a section of
$\Omega^1_{\XX}(\XX_\alpha)$. Taking the differences on the overlaps
gives a 1-cocycle representing a class in
$R^1\pi_*\OO_\XX\tensor\Omega^1_U$. We will then compute the cup
product with $\omega_j$ and the trace to get an element of $R\,dt$.

[The calculations below also justify the regularity assertions made
about the $\omega_i$ above.]

First consider $\XX_1$, where we have the equality
\begin{equation}\label{eq:X1}
0=y-x^{r-1}(xy+1)(xy+t)
\end{equation}
the differential of which leads to the relation
\begin{multline}\label{eq:dX1}
0=dy\left(1-x^r(xy+t)-x^{r-1}(xy+1)\right)\\
+dx\left(-(r-1)x^{r-2}(xy+1)(xy+t)-x^{r-1}(xy+t)y-x^{r-1}(xy+1)y\right)\\
+dt\left(-x^{r-1}(xy+1)\right).
\end{multline}
Now consider the naive lift of $\omega_i$ to $\Omega^1_{\XX}$ on $\XX_1$:
$$\frac{x^{i-1}d(xy)}{y}=\frac{x^idy}{y}+x^{i-1}dx.$$
This is obviously regular away from $y=0$. Near $y=0$, the equality
\eqref{eq:X1} shows that in an open neighborhood of $y=0$, the
function $y$ is a unit times $x^{r-1}$. Also, near $y=0$ the
coefficient of $dy$ in \eqref{eq:dX1} is a unit and the coefficients of $dx$
and $dt$ are divisible by $x^{r-2}$, so we may rewrite $x^idy$ (with
$i\ge1$) as a regular 1-form times $x^{r-1}$. Therefore $x^{i}dy/y$ is
everywhere regular on $\XX_1$. This shows that
$\frac{x^{i-1}d(xy)}{y}$ is a section of $\Omega^1_{\XX}$ which maps
to $\omega_i$ in $\Omega^1_{\XX/U}$.

Next we turn to $\XX_2$, where we have the equality 
\begin{equation}\label{eq:X2}
0=z-w^{r-1}(w+z)(w+zt)
\end{equation}
the differential of which leads to the relation
\begin{multline}\label{eq:dX2}
0=dz\left(1-w^{r-1}(w+zt)-w^{r-1}(w+z)t\right)\\
+dw\left(-(r-1)w^{r-2}(w+z)(w+zt)-w^{r-1}(w+zt)-w^{r-1}(w+z)\right)\\
+dt\left(-w^{r-1}(w+z)z\right).
\end{multline}
Now consider the naive lift of $\omega_i$ to $\Omega^1_{\XX}$ on
$\XX_2$:
$$\frac{w^{i-1}d(w/z)}{1/z}=\frac{w^idz}{z}+w^{i-1}dw.$$
This is obviously regular away from $z=0$. Near $z=0$, the equality
\eqref{eq:X2} shows that $z$ is a unit times $w^{r+1}$. Also, near
$z=0$ the coefficient of $dz$ in \eqref{eq:dX2} is a unit and the
coefficients of $dw$ and $dt$ are divisible by $w^{r}$, so we may
rewrite $w^idz$ (with $i\ge1$) as a regular 1-form times $w^{r+1}$.
This shows that $w^idz/z$ is everywhere regular on $\XX_1$ and
therefore $\frac{w^{i-1}d(w/z)}{1/z}$ is a section of $\Omega^1_{\XX}$
which maps to $\omega_i$ in $\Omega^1_{\XX/U}$.

Finally, we turn to $\XX_3$, where we will have to work harder.  We
have the equality
\begin{equation}\label{eq:X3}
0=u^rv-(1+v)(1+tv)
\end{equation}
the differential of which leads to the relation
\begin{equation}\label{eq:dX3}
0=du\left(ru^{r-1}v\right)
+dv\left(u^r-(1+tv)-(1+v)t\right)
+dt\left(-(1+v)v\right).
\end{equation}
The naive lift of $\omega_i$ turns out not to be regular on all of
$\XX_3$.  Instead of it, we consider
$$\frac{-dv}{u^iv}-\frac{1+v}{u^i}\frac{dt}{t-1}.$$
The equations \eqref{eq:X3} and \eqref{eq:dX3} and some algebra lead
(eventually ...) to the equality
$$\frac{-dv}{u^iv}+\frac{1+v}{u^i}\frac{dt}{t-1}=
\frac{ru^{r-1-i}du}{u^r-(1+tv)-(1+v)t}+\frac{u^{r-i}(v-1)}{u^r-(1+tv)-(1+v)t}\frac{dt}{t-1}$$
Now we note that $t-1$ and $v$ are units on all of $\XX_3$ and that 
$u^r-(1+tv)-(1+v)t$ is a unit in a neighborhood of the locus where
$u=0$.  Thus the left hand side is regular where $u\neq0$ and the
right hand side is regular in a neighborhood of $u=0$.  Together they
define a section of $\Omega^1_{\XX/U}$ on $\XX_3$ which lifts
$\omega_i$.

\subsection{Computing the Kodaira-Spencer pairing}
Taking differences of lifts on overlaps yields the following 1-cocycle
with values in $\pi^*\Omega^1_U$:
$$g_{12}=g_{21}=0\qquad g_{23}=-g_{32}=g_{13}=-g_{31}=
\frac{1+v}{u^i}\frac{dt}{t-1}.$$
Here $g_{\alpha\beta}$ is a section of $\pi^*\Omega^1_U$ over
$\XX_{\alpha\beta}=\XX_{\alpha}\cap\XX_{\beta}$ for
$\alpha,\beta\in\{1,2,3\}$.

Taking the cup product of $KS(\omega_i)$ with $\omega_j$, we get a
class in $R^1\pi_*\Omega^1_{\XX/U}\tensor\Omega^1_U$ represented by
the cocycle
$$h_{12}=h_{21}=0\qquad h_{23}=-h_{32}=h_{13}=-h_{31}=
-\frac{1+v}{u^{i+j}}\frac{dv}{v}\frac{dt}{t-1}.$$

We recall an explicit form of the trace map
$H^1(X,\Omega^1_X)\cong K$ in Cech cohomology for $X$ a curve over a
field $K$. 
Given a class $c$ represented by a cocycle $h_{\alpha\beta}$
for an affine cover $\{U_\alpha\}$,
choose {\it meromorphic\/} 1-forms $\sigma_\alpha$ on $U_\alpha$ so that
\begin{equation}\label{eq:cocycle}
h_{\alpha\beta}=\sigma_\alpha-\sigma_\beta
\end{equation}
on $U_{\alpha\beta}$. Then if $P\in X$ is a closed point, choose an
$\alpha$ so that $P\in U_\alpha$ and set $r_P=\res_P(\sigma_\alpha)$.
The cocycle condition \eqref{eq:cocycle} shows that $r_P$ does not
depend on the choice of $\alpha$. The trace map then sends $c$ to
$\sum_Pr_P\in K$.

We apply this explicit trace map to the generic fiber of our
family $\XX\to U$. We may choose
$$\sigma_1=\sigma_2=0\qquad\sigma_3=
\frac{1+v}{u^{i+j}}\frac{dv}{v}\frac{dt}{t-1}.$$ It is clear that
$r_P=0$ except possibly at the points $(u,v)=(0,-1)$ and $(0,-1/t)$ in
(the generic fiber of) $\XX_3$. A short computation reveals that the
residue is zero at $(0,-1)$, and at $(0,-1/t)$ it is
$$\begin{cases}
\frac{r\,dt}{t(t-1)}&\text{if $i+j=r$}\\
0&\text{if $i+j\neq r$.}
\end{cases}
$$
Since $r$ and $t(t-1)$ are units in $R$, this proves that the pairing 
\begin{align*}
H^0(U,\pi_*\Omega^1_{\XX/U})&\times H^0(U,\pi_*\Omega^1_{\XX/U})&&\to
&H^0(U,\Omega^1_U)\\
\omega_i&\times\omega_j&&\mapsto&(\omega_i,\omega_j)\quad
\end{align*}
is a perfect pairing of free $R$-modules.  This completes the proof of
Proposition~\ref{prop:KS}.

[There should be a calculation of $KS$ for Legendre ($r=2$) in the
literature which can be quoted as a reality check.]

\subsection{A compatibility}
In this section we prove a compatibility between the Kodaira-Spencer
maps of a family of curves and the corresponding family of Jacobians.
[This is surely well-known and there should be an off-the-shelf reference,
but it was easier to just write it down.]

Let $V$ be a smooth (possibly open) curve over a field $k$ and let
$\pi:\YY\to V$ be a surjective morphism whose geometric fibers are smooth
projective curves of genus $g$,  As in Section~\ref{s:KS}, we have a
Kodaira-Spencer map $KS_\YY:\pi_*\Omega^1_{\YY/V}\to
R^1\pi_*\OO_{\YY}\tensor\Omega^1_V$.

Let $\tau:\JJ\to V$ be the family of Jacobian varieties attached to
$\pi$, so $\tau$ is a smooth, projective family of abelian varieties
of dimension $g$ and the fibers of $\tau$ are the Jacobians of the
corresponding fibers of $\pi$.  There is a Kodaira-Spencer map
associated to $\tau$, denoted $KS_\JJ:\tau_*\Omega^1_{\JJ/V}\to
R^1\tau_*\OO_{\JJ}\tensor\Omega^1_V$, defined in an entirely analogous
manner [mentioned in the proof below].

Now assume that $\pi:\YY\to V$ has a section.  Using it, we have the
Abel-Jacobi map $AJ:\YY\to\JJ$, which is a closed immersion of schemes
over $V$.  The associated structure map $\OO_\JJ\to AJ_*\OO_{\YY}$
induces a morphism
$$R^1\tau_*\OO_{\JJ}\to R^1\tau_*\left(AJ_*\OO_\YY\right)
\cong R^1\pi_*\OO_\YY$$ which is known to be an isomorphism of locally
free sheaves of rank $g$ on $V$. (In the second isomorphism above and
several places below, we use that $AJ_*$ is exact because $AJ$ is a a
closed immersion.)

Similarly, the pull-back map on 1-forms $\Omega^1_{\JJ/V}\to
AJ_*\Omega^1_{\YY/V}$ induces a morphism
$$\tau_*\Omega^1_{\JJ/V}\to
\tau_*AJ_*\Omega^1_{\YY/V}\cong\pi_*\Omega^1_{\YY/V}$$
which is also known to be an isomorphism of locally free sheaves of rank
$g$ on $V$.

The compatibility we need is the following.

\begin{lemma}\label{lemma:KS-comp}
The diagram of locally free $\OO_V$-modules:
$$\xymatrix{
\tau_*\Omega^1_{\JJ/V}\ar[rr]^{KS_\JJ}\ar[d]&&R^1\pi_*\OO_{\JJ}\tensor\Omega^1_V\ar[d]\\
\pi_*\Omega^1_{\YY/V}\ar[rr]^{KS_\YY}&&R^1\pi_*\OO_{\YY}\tensor\Omega^1_V
}$$
commutes.  Here the vertical maps are the isomorphisms described above.
\end{lemma}

\begin{proof}
Consider the short exact sequence of $\OO_\YY$-modules
$$0\to\pi^*\Omega^1_V\to\Omega^1_\YY\to\Omega^1_{\YY/V}\to0.$$
Applying the exact functor $AJ_*$, we get a short exact sequence of
$\OO_\JJ$-modules
$$0\to AJ_*\pi^*\Omega^1_V\to AJ_*\Omega^1_\YY\to
AJ_*\Omega^1_{\YY/V}\to0.$$
Note that by the projection formula,
$AJ_*\pi^*\Omega^1_V\cong\tau^*\Omega^1_V\tensor AJ_*\OO_\YY$.
Using that isomorphism and pull back maps, we get a commutative
diagram of $\OO_\JJ$-modules
$$\xymatrix{
0\ar[r]&\tau^*\Omega^1_V\ar[r]\ar[d]&
\Omega^1_\JJ\ar[r]\ar[d]&\Omega^1_{\JJ/V}\ar[r]\ar[d]&0\\
0\ar[r]&\tau^*\Omega^1_V\tensor AJ_*\OO_\YY\ar[r]&
AJ_*\Omega^1_\YY\ar[r]&AJ_*\Omega^1_{\YY/V}\ar[r]&0.}$$
Now we apply $\tau_*$ and take a portion of the resulting morphism of
long exact sequences to get a commutative diagram:
$$\xymatrix{
  \tau_*\Omega^1_{\JJ/V}\ar[r]\ar[d]&R^1\tau_*\OO_\JJ\tensor\Omega^1_V\ar[d]\\
  \tau_*AJ_*\Omega^1_{\YY/V}\ar[r]&R^1\tau_*AJ_*\OO_\YY\tensor\Omega^1_V.}$$
But we already noted that
$\tau_*AJ_*\Omega^1_{\YY/V}\cong\pi_*\Omega^1_{\YY/V}$ and
$R^1\tau_*AJ_*\OO_\YY\cong R^1\pi_*\OO_\YY$, so we have a commutative
diagram
$$\xymatrix{
\tau_*\Omega^1_{\JJ/V}\ar[r]\ar[d]&R^1\tau_*\OO_\JJ\tensor\Omega^1_V\ar[d]\\
\pi_*\Omega^1_{\YY/V}\ar[r]&R^1\pi_*\OO_\YY\tensor\Omega^1_V.}$$
The vertical arrows are the isomorphisms described before the
statement of the lemma and the horizontal maps are (by definition) the
Kodaira-Spencer maps of $\JJ$ and $\YY$, so we have proved the desired result.
\end{proof}


\begin{cor}\label{cor:KS-JJ}
Let $\JJ\to U$ be the family of Jacobians associated to the curve
$\XX\to U$ described in Section~\ref{s:model}.  Then the
Kodaira-Spencer map
$$\Omega^1_{\JJ/U}\to R^1\tau_*\OO_\JJ\tensor\Omega^1_U$$ 
is an isomorphism of locally free $\OO_U$-modules of rank $g=r-1$.
\end{cor}


\subsection{$p$-torsion}
Now assume that $p>0$. Combining Corollary~\ref{cor:KS-JJ}, the
ordinarity of $J_X$ (proved by Rachel), and the Proposition on page
1093 of Voloch, {\it American Journal of Mathematics\/} {\bf 117}
(1995), we have reached our goal:

\begin{prop}
We have $J_X(K)[p]=J_X(K^{sep})[p]=0$.
\end{prop}

\begin{rem} 
  This proof is not so simple. Unfortunately, it seems
  unlikely that the more straightforward idea of using $p$-descent
  will yield the desired result. In the Legendre case ($r=2$), the
  $p$-part of $\sha$ is often non-trivial and I would expect the same
  to be true for $r>2$. 
\end{rem}













\section{Regular model at infinity}
%{Jenn Park \and Shahed Sharif}


Let $k$ be the finite field $\F_q$ and $K = k(t)$. Let $r \geq 3$ be an integer. Let $C$ be the smooth projective curve with affine model
\[
C: xy^r = (x+1)(x+t).
\]
The purpose of this note is to compute the minimal proper regular model of $C$ at $t = \infty$, as well as the component group. In all of the following, we test regularity of various affine charts. These charts will be given by a single equation in $\mathbb{A}^3_k$ of the form $f(x_1,x_2,x_3) = g(x_1,x_2,x_3)$. We will make frequent use of the fact that a point $P_0$ on the surface is regular if and only if $\frac{\partial f}{\partial x_i}(P_0) \neq \frac{\partial g}{\partial x_i}(P_0)$ for some $i$.

\subsection{Desingularization}
\label{sec:desingularization}

Let $T = \frac{1}{t}$, so that our curve has affine piece given by
\[
\caff:Txy^r = (x+1)(Tx+1).
\]
To desingularize the generic fiber, consider the affine curves
\begin{align*}
  C'&: Tvy^r = (v+1)(v+T) \\
  C''&: Tu = z^{r-1}(z+uT)(z+u) \\
  C'''&: Tw = z^{r-1}(wz+T)(wz+1).
\end{align*}
Now glue the 4 equations together via
\begin{gather*}
  v = \frac{1}{x} = \frac{z}{u} = wz \qquad y = \frac{1}{z} \\
  u = \frac{x}{y} = \frac{1}{vy} = \frac{1}{w} \qquad z = \frac{1}{y}\\
  w = \frac{y}{x} = vy = \frac{1}{u} \qquad z = \frac{1}{y}
\end{gather*}
One checks that the resulting curve is generically smooth.
\begin{remark}
  If one projectivizes $\caff$, one obtains the equation
  \[
  TXY^r = Z^{r-1}(X+Z)(TX+Z)
  \]
  in $\Pro^2_K$. The chart $C''$ corresponds to the affine patch with $Y = 1$. When setting $X = 1$, one finds that the curve is not generically smooth; it has a cusp at $[1:0:0]$. Normalizing, one obtains the equation for $C'''$. 
\end{remark}


The special fibers on these charts, given by $T = 0$, are as follows:
\[
\begin{array}{l@{:\,\,}ll}
\caff & x + 1 = 0 & C_1 \\
C' & v(v + 1) = 0 & C_2, C_1 \\
C'' & z^r(z + u) = 0 & R_1, C_1 \\
C''' & z^rw(wz+1) = 0 & R_1, C_2, C_1.
\end{array}
\]
The third column gives names for the components of the special fiber, given in order of the factors in the equation. For example, in $C'''$, $C_2$ is given by the equation $w = 0$ (and $T = 0$, which holds for every component). The components $C_1$ and $C_2$ each have multiplicity 1, while $R_1$ has multiplicity $r$. Let $P = C_1 \cap R_1$. Note that $P$ lies in the chart $C''$, but in no others. Let $Q = C_2 \cap R_1$. Then $Q$ lies in $C'''$ but in no other chart. One checks that every point but $P$ and $Q$ is regular. Therefore we need to blow up at these two points.

One remark: the reason behind the notation $R_1$ will become apparent in \S~\ref{sec:blow-up-Q}.

\subsection{Blowing-up}
\label{sec:blowing-up}

\subsubsection{Blow-up at $P$.}
\label{sec:blow-up-P}

We first blow up at $P$, which is given by $(u,z,T) = (0,0,0)$. The blow-up is given by two charts. The first chart is computed via the change of variables
\[
z = uv \qquad T = u\tau.
\]
We obtain the arithmetic surface with equation
\[
\tau = u^{r-1}v^{r-1}(v + u\tau)(v + 1).
\]
This has special fiber given by $T = u\tau = 0$, with components
\begin{align*}
  \tilde{C_1}&: (v + 1 = \tau = 0) \\
  \tilde{R_1}&: (v = \tau = 0) \\
  D_r&: (u = \tau = 0).
\end{align*}
In the above, $\tilde{C_1}$, $\tilde{R_1}$ are the strict transforms of $C_1$, $R_1$ respectively with the same multiplicities as before. To compute the multiplicity of the third component $D_r$, we consider the ring
\[
{k[u,v,\tau]_{(u,\tau)}}/{(u^{r-1}v^{r-1}(v + u\tau)(v + 1) - \tau)}.
\]
This is the local ring for the subscheme $D_r$. Since $D_r$ is a prime divisor, the local ring is a discrete valuation ring $\sO_{C,D_r}$ with valuation, say, $\nu$. Then the multiplicity of $D_r$ in the special fiber is $\nu(T) = \nu(u\tau)$. But this valuation equals the length of the Artinian ring
\[
{\sO_{C,D_r}}/{(u\tau)}.
\]
Let $\alpha = v^{r}(v+1)$. Then the above ring equals
\[
{k[u,v,\tau]_{(u,\tau)}}/{(\alpha u^{r-1} - \tau, u\tau)}.
\]
The first expressions allows us to substitute $\tau = \alpha u^{r-1}$. The second expression then becomes $\alpha u^r$. Since $\nu(\alpha) = 0$, our Artinian ring is isomorphic to
\[
{k[u,v]_{(u)}}/{(u^r)}.
\]
Therefore the multiplicity of $D_r$ in the special fiber is $r$.

The intersection points are $P_1 = \tilde{C_1} \cap D_r$ given by $u=v+1=\tau=0$, and $P_z = \tilde{R_1} \cap D_r$ given by $u=v=\tau=0$. (Observe that $\tilde{C_1} \cap \tilde{R_1} = \emptyset$.) Writing the equation for our surface as
\[
u^{r-1}v^{r-1}(v + u\tau)(v + 1) - \tau = 0
\]
and taking the $\tau$ derivative on the left, we obtain
\[
u^rv^{r-1}(v + 1) - 1.
\]
We now observe that this expression equals $-1$ at both $P_1$ and $P_z$. Therefore both points are regular.

Now we compute the second chart. This chart is given by the change of variables
\[
u=zx \qquad T=zT'
\]
whence we obtain the equation
\[
xT' = z^{r-1}(1 + zxT')(1 + x).
\]
The special fiber is given by $T = zT' = 0$, which gives us the components
\begin{align*}
  \tilde{C_1}&: (1 + x = T' = 0) \\
  D_r&: (z = T' = 0) \\
  D_1&: (z = x = 0).
\end{align*}
Observe that $D_1$ has multiplicity $1$. The intersection $\tilde{C_1} \cap D_r$ occurs when $1 + x = T' = z = 0$. One checks that this intersection point is in fact $P_1$. The intersection $D_r \cap D_1$ occurs when $x = T' = z = 0$. But now note that $x = \frac{1}{v}$. Therefore this intersection point is distinct from $P_2$. (Or easier: $\tilde{R_1}$ does not appear in this chart for the same reason.) We call this new point $P_3$. One checks that $P_3$ is \emph{not} regular.

In order to resolve $P_3$, we will desingularize inductively. To ease notation, write $\alpha$ for $(1 + zxT')(1 + x)$, so that
\begin{itemize}
    \item the chart containing $P_3$ has equation $xT' = \alpha z^{r-1}$,
    \item $\alpha(P_3) \neq 0$, and
    \item $T = zT'$.
\end{itemize}

We now desingularize by replacing $P_3$ with 2 charts. The first chart is given by
\[
T'_0 = \alpha x^{r-3} z_0^{r-1}
\]
where $T' = x T'_0$ and $z = x z_0$, whence $T = x^2 T'_0 z_0$. The special fiber consists of 2 components, one given by $T'_0 = z_0 = 0$ and of multiplicity $r$, the other given by $T'_0 = x = 0$ having multiplicity $r-1$. One verifies that the first component is $D_r$; the second component, which we call $D_{r-1}$, is exceptional; and the intersection is transverse. Furthermore, taking the $T'_0$ derivative shows that this chart is regular.

  The second chart is given by $x_0 T' = \alpha z^{r-2}$, where $x = x_0 z$. The special fiber then consists of the components $T' = z = 0$ with multiplicity $r-1$ and $x_0 = z = 0$ with multiplicity $1$. One verifies that these are $D_{r-1}$ and $D_1$ respectively. The intersection point is regular if and only if $r = 3$, in which case the claim is proved. If $r > 3$, then we are in essentially the same situation as with $P_3$ except that $x$ has become $x_0$ and $r$ has been replaced with $r-1$. We can therefore iterate, thus obtaining a ``tail'' of rational curves $D_r, D_{r-1}, \dots, D_1$, each crossing transversely with adjacent curves in the list, and such that $D_i$ has multiplicity $i$ in the special fiber.


\subsubsection{Blow-up at $Q$.}
\label{sec:blow-up-Q}

Recall that $Q$ lies on the affine chart with equation
\begin{equation}
  C''':wT = z^{r-1}(wz + T)(wz + 1)\label{eq:C'''},
\end{equation}
is the intersection point of the components $C_2$ and $R_1$, and has coordinates $(w,z,T) = (0,0,0)$. Just as in the case of $P$, we will desingularize at $Q$ recursively; however, the situation is rather more complicated.  

One should keep in mind the following: blowing-up is a local procedure; and $wz + 1 = T = 0$ is an equation for $C_1$, a component which does not pass through $Q$. Therefore, the factor $wz + 1$ and its transforms can be safely ignored when computing special fibers below.

We start by replacing $Q$ with two charts. Chart 1 is given by the equation
\[
T_0 = b^{r-1}w^{r-2}(bw + T_0)(bw^2 + 1)
\]
where $z = bw$ and $T = T_0w$. The special fiber consists of 2 components:
\begin{itemize}
    \item $R_1: T_0 = b = 0$ with multiplicity $r$, and
    \item $R_2: T_0 = w = 0$ also with multiplicity $r$.
\end{itemize}
The first component is in fact identical to the $R_1$ appearing in \S~\ref{sec:blow-up-P} and earlier. The second component is exceptional. By considering the $T_0$ derivative, one sees that this chart is regular.

Chart 2 is given by the equation
\[
w_1 T_1 = z^{r-2}(w_1z + T_1)(w_1 z^2 + 1)
\]
where $w = w_1 z$ and $T = T_1 z$. The special fiber consists of three components:
\begin{itemize}
    \item $C_2: T_1 = w_1 = 0$ with multiplicity $1$,
    \item $R_2: T_1 = z = 0$ with multiplicity $r$, and
    \item $E_1: z = w_1 = 0$ with multiplicity $1$.
\end{itemize}
The third component is exceptional. The intersection point $Q_1$ has coordinates $(w_1, z, T_1) = (0,0,0)$ and, if $r \geq 3$, is \emph{not} regular.

We therefore continue by blowing-up chart 2. We do this recursively as follows. For $1 \leq i \leq r-3$, let chart $(2,i)$ be the chart given by
  \[
  w_i T_i = z^{r-i-1} (w_i z + T_i) (w_i z^{i+1} + 1).
  \]
  Let $T = T_i z^i$. Then the special fiber ($T = 0$) consists of the components
  \begin{itemize}
      \item $C_2: T_i = w_i = 0$ with multiplicity 1,
      \item $R_{i+1}: T_i = z = 0$ with multiplicity $r$, and
      \item $E_i: z = w_i = 0$ with multiplicity $i$.
  \end{itemize}
  Note that chart 2 is the same as chart $(2,1)$. One verifies that this chart is nonregular only at $Q_i: (w_i, z, T_i) = (0, 0, 0)$. We blow up $Q_i$ via 3 charts.

  The first chart is $(2,i+1)$, which we glue to $(2,i)$ away from $Q_i$ using $w_i = z w_{i+1}$ and $T_i = z T_{i+1}$. One verifies that this is consistent with the construction above.

  The second chart is $(2,i)a$ given by $w_i = w_a T_i$ and $z = z_a T_i$. (This is an abuse of notation; one should say $w_{i,a}$ and $z_{i,a}$ or some such.) Then $T = z_a^i T_i^{i+1}$. Our chart is given by
  \[
  w_a = z_a^{r-i-1} T_i^{r-i-2} (w_a z_a T_i + 1) (w_a z_a^{i+1} T_i^{i+2} + 1).
  \]
  The special fiber consists of
  \begin{itemize}
      \item $E_i: z_a = w_a = 0$ with multiplicity $i$, and
      \item $E_{i+1}: T_i = w_a = 0$ with multiplicity $i+1$.
  \end{itemize}
  By computing the $w_a$ derivative, one checks that this chart is regular.

  The third chart is $(2,i)b$ given by $z = w_i z_b$ and $T_i = w_i T_b$. Then $T = w_i^{i+1} z_b^i T_b$ and the equation is
  \[
  T_b = w_i^{r-i-2} z_b^{r-i-1} (w_i z_b + T_b) (w_i^{i+2} z_b^{i+1} + 1).
  \]
  The chart is regular (take the $T_b$ derivative) and the special fiber has components
  \begin{itemize}
      \item $R_{i+1}: T_b = z_b = 0$ with multiplicity $r$, and
      \item $R_{i+2}: T_b = w_i = 0$ with multiplicity $r$.
  \end{itemize}

  The remaining and, as it will turn out, final case occurs when $i = r - 2$. (Note that this is the only case we need consider when $r = 3$.) In this case, we have chart $(2, r-2)$ given by
  \[
  w_{r-2} T_{r-2} = z (w_{r-2} z + T_{r-2}) (w_{r-2} z^{r-1} + 1)
  \]
  with $w_{r-2}, T_{r-2}$ given as above, and $T = T_{r-2} z^{r-2}$. We blow up as before; that is, construct the charts $(2, r-1)$, $(2, r-2)a$ and $(2, r-2)b$ as above. We now verify that there are two differences: every chart is regular, and the special fibers for $(2, r-1)$ and $(2,r-1)b$ differ from the cases for smaller $i$.

  For chart $(2, r-1)$, the equation is
  \[
  w_{r-1} T_{r-1} = (w_{r-1} z + T_{r-1}) (w_{r-1} z^{r} + 1)
  \]
  and $T = T_{r-1} z^{r-1}$. The special fiber consists of the components
  \begin{itemize}
      \item $C_2: T_{r-1} = w_{r-1} = 0$ with multiplicity 1, and
      \item $E_r: T_{r-1} = z = 0$ with multiplicity $r$.
  \end{itemize}
  One checks that the $T_{r-1}$ partial derivative gives equality if and only if $z = 0$ and $w_{r-1} = 1$. But plugging these values into the $z$-derivative yields an inequality. Therefore this chart is regular.

  For chart $(2, r-2)a$, the equation is
  \[
  w_a = z_a (w_a z_a T_{r-2} + 1) ( w_a z_a^{r-1} T_i^{r} + 1)
  \]
  with special fiber consisting of the components
  \begin{itemize}
      \item $E_{r-2}: z_a = w_a = 0$ with multiplicity $r-2$, and
      \item $E_{r-1}: T_{r-2} = w_a - z_a = 0$ with multiplicity $r-1$.
  \end{itemize}
  The argument for regularity is the same as for the general charts $(2,i)a$.

  Finally, the chart $(2, r-2)b$ is given by
  \[
  T_b = z_b (w_{r-2} z_b + T_b) (w_{r-2}^{r} z_b^{r-1} + 1)
  \]
  with special fiber consisting of the components
  \begin{itemize}
      \item $R_{r-1}: T_b = z_b = 0$ with multiplicity $r$,
      \item $E_r: T_b = w_{r-2} = 0$ with multiplicity $r$, and
      \item $E_{r-1}: w_{r-2} = z_b - 1 = 0$ with multiplicity $r-1$.
  \end{itemize}
  Checking the $T_b$ derivative, we see that we have equality only along $E_{r-1}$. But the $w_{r-2}$ derivative along $E_{r-1}$ always yields an inequality. Therefore this chart is regular.

  Let $\scd$ be the resulting regular model for $C$, and write $C_k$ for its special fiber at $T = 0$.

\subsubsection{Dual graph}
\label{sec:dual-graph}

Recall that the dual graph of $C_k$ is defined to be the graph such that the set of irreducible components of $C_k$ is the set of vertices, and the set of intersection points between two such components is the set of edges between the corresponding vertices. Putting our calculations above together, we obtain
\begin{proposition}\label{prop:dual-graph-plus-multiplicities}
  The dual graph for $C_k$ is given by Figure~\ref{fig:superelliptic-dual-graph}. All intersections are transverse. The components $C_1$ and $C_2$ have multiplicity $1$, the $R_i$ each have multiplicity $r$, and the $D_i$ and $E_i$ have multiplicity $i$.
\begin{figure}[h]\centering
\[
\xygraph{
  !{<0cm,0cm>;<1.5cm,0cm>:<0cm,1.25cm>::}
  !{(1,5) }*{\bullet}="r1"
  !{(2,5) }*{\bullet}="r2"
  !{(3,5) }*{\quad\cdots\quad}="rspace"
  !{(4,5) }*{\bullet}="rrm"
  !{(1,5.6) }*{R_1}
  !{(2,5.6) }*{R_2}
  !{(4,5.6) }*{R_{r-1}}
  !{(0,4) }*{C_1\;\bullet}="c1"
  !{(5,4) }*{\bullet\; C_2}="c2"
  !{(1,0) }*{\bullet}="d1"
  !{(1,1) }*{\bullet}="d2"
  !{(1,1.5) }*{}="dbelow"
  !{(1,2.1) }*{\vdots}="dspace"
  !{(1,2.6) }*{}="dabove"
  !{(1,3) }*{\bullet}="drm"
  !{(1,4) }*{\bullet}="dr"
  !{(.6,0) }*{D_1}
  !{(.6,1) }*{D_2}
  !{(.5,3) }*{D_{r-1}}
  !{(1.4,4) }*{D_r}
  !{(4,0) }*{\bullet}="e1"
  !{(4,1) }*{\bullet}="e2"
  !{(4,1.5) }*{}="ebelow"
  !{(4,2.1) }*{\vdots}="espace"
  !{(4,2.6) }*{}="eabove"
  !{(4,3) }*{\bullet}="erm"
  !{(4,4) }*{\bullet}="er"
  !{(4.4,0) }*{E_1}
  !{(4.4,1) }*{E_2}
  !{(4.6,3) }*{E_{r-1}}
  !{(3.6,4) }*{E_r}
  "r1"-"r2"
  "r2"-"rspace"
  "rspace"-"rrm"
  "r1"-"dr"
  "c1"-"dr"
  "dr"-"drm"
  "drm"-"dabove"
  "dbelow"-"d2"
  "d2"-"d1"
  "rrm"-"er"
  "c2"-"er"
  "er"-"erm"
  "erm"-"eabove"
  "ebelow"-"e2"
  "e2"-"e1"
}
\]



%%% Local Variables:
%%% TeX-master: "AIM_Ranks_Infty"
%%% End:
\caption{Dual graph of $C_k$}
\label{fig:superelliptic-dual-graph}
\end{figure}
\end{proposition}


\subsection{Minimality}
\label{sec:minimality}

Next we must make sure our model is minimal via Castelnuovo's criterion, which we now state.
\begin{thm}[Prop.~IV.7.5 in \cite{silvermanATAEC}]\label{thm:castelnuovo}
  Let $R$ be a discrete valuation ring with algebraically closed residue field. Let $\scd$ be a regular arithmetic surface proper over $R$ whose generic fiber is a nonsingular projective curve of genus $g \geq 1$. Then $\scd$ is minimal if and only if the special fiber of $\scd$ has no divisors $E$ for which $E \isom \Pro^1$ and $E^2 = -1$.
\end{thm}

We may base-extend to the strict henselization of $\F_q[T]$ at $T = 0$, as unramified base-change affects neither minimality of models nor intersection numbers. Thus we can use Theorem~\ref{thm:castelnuovo} to check that $\scd$ is a minimal model. Every irreducible component is isomorphic to $\Pro^1$, so it remains to check the self-intersections. To compute these self-intersections we use the following theorem.
\begin{thm}[Prop.~IV.7.3 in \cite{silvermanATAEC}]\label{thm:fibral-intersect-total}
  Let $\scd$ be a regular arithmetic surface proper over a discrete valuation ring, and let $C_k$ be the special fiber. Let $D$ be a fibral divisor. Then $D \cdot C_k = 0$.
\end{thm}

In our case, all of the geometric intersections are transverse, so we can simplify the above theorem.
\begin{proposition}\label{prop:transverse-self-intersections}
  Let $\scd$ be a regular arithmetic surface proper over a discrete valuation ring, and let $C_k$ be the special fiber. Let $C_k^{\sf red}$ be the associated reduced scheme. Suppose that every singular point of $C_k^{\sf red}$ consists of a transverse crossing of two irreducible components. Let $V$ be an irreducible component of $C_k$. Label the remaining irreducible components $V_1, V_2,$ etc. Then
  \[
  V^2 = -\frac{1}{m} \sum m_i n_i
  \]
  where $m$ is the multiplicity of $V$, $m_i$ is the multiplicity of $V_i$, and $n_i$ is the number of intersection points of $V$ and $V_i$.
\end{proposition}

\begin{proof}
  The total fiber is
  \[
  C_k = mV + \sum m_i V_i.
  \]
  By Theorem~\ref{thm:fibral-intersect-total}, we have
  \begin{align*}
    0 &= V \cdot C_k \\
    &= mV^2 + \sum m_i (V \cdot V_i) \\
    &= mV^2 + \sum m_i n_i
  \end{align*}
  from which the claim follows.
\end{proof}

\begin{proposition}\label{prop:self-intersect-superelliptic}
  Let $\scd$ be the model constructed in sections \S~\ref{sec:blowing-up} with special fiber described in Prop.~\ref{prop:dual-graph-plus-multiplicities}. Then
  \begin{enumerate}
      \item $C_1^2 = C_2^2 = -r$, and
      \item $R_i^2 = D_i^2 = E_i^2 = -2$ for all $i$.
  \end{enumerate}
\end{proposition}

\begin{proof}
  We use the notation of Prop.~\ref{prop:transverse-self-intersections}. Inspection of Figure~\ref{fig:superelliptic-dual-graph} shows that for any choice of $V$, $V_i$, $n_i$ is either $0$ or $1$. For $V = C_1$ we have $m = 1$. This component only intersects $D_r$, which has multiplicity $r$. The claim for $C_1$ follows. Similar reasoning holds for $C_2$.

  For $V = R_i$, we have $m = r$. Also, $R_i$ intersects exactly two other components, both with multiplicity $r$. We thus obtain $R_i^2 = -2$.

  For $V = D_i$, $1 < i < r$, we have $m = i$. Also, $D_i$ intersects $D_{i-1}$ and $D_{i+1}$ which have multiplicities $i-1$ and $i+1$ respectively. Now Prop.~\ref{prop:transverse-self-intersections} gives us $D_i^2 = -2$.

  For $V = D_1$, $m = 1$ and $D_1$ only intersects $D_2$, which latter has multiplicity $2$. Again we get $D_1^2 = -2$.

  For $V = D_r$, $m = r$, and $D_r$ intersects $D_{r-1}, C_1$, and $R_1$ with multiplicities $r-1$, $1$, and $r$ respectively. We obtain $D_r^2 = -2$.

  The argument for the $E_i$ is completely analogous.
\end{proof}



\subsection{Component group}
\label{sec:component-group}

To compute the component group of the Jacobian of a curve, one typically first base-extends to a strictly henselian ring. The following result of Bosch-Liu allows us to do so without complication.

\begin{proposition}[Corollary~1.8 in \cite{bosch-liu1999}]\label{prop:bosch-liu-phi-constant}
  Let $K$ be the fraction field of a discrete valuation ring $R$ with residue field $k$. Let $C/K$ be a smooth proper geometrically integral curve, and $\scd/R$ a minimal proper regular model for $C$. Let $J$ be the Jacobian of $C$, and $\sj/R$ its N\'eron model. The connected components of the special fiber of $\sj$ form a finite \'etale group scheme $\Phi$ over $k$. If every component of the special fiber of $\scd$ is geometrically irreducible, then $\Phi$ is a constant group scheme.
\end{proposition}

The background material in the remainder of this section can be found in~\cite[Ch. 9]{blr}. In the latter, the authors define the \emph{geometric multiplicity} of a component of the special fiber, and use the notation $e_i$. Our base field is algebraically closed, and it follows that all geometric multiplicities are $1$. (See \cite[Defn~9.1.3]{blr}.)

In order to compute the component group at $\infty$ of $C$, we make use of the \emph{intersection matrix} of the special fiber. This is the matrix whose $(i,j)$ entry is the intersection number $(V_i \cdot V_j)$, where $V_i$ is some labeling of the irreducible components of the special fiber. We will order the components as follows: $D_1, D_2, \dots, D_r, R_1, \dots, R_{r-1}, E_r, \dots, E_1, C_1, C_2$. As all intersections are transverse and distinct components in $C_k$ intersect at most once (see Figure~\ref{fig:superelliptic-dual-graph}), the off-diagonal entries of the intersection matrix are $1$ if the two corresponding components meet and $0$ otherwise. The self-intersections were computed in Prop.~\ref{prop:self-intersect-superelliptic}. Therefore the intersection matrix is
\[
A = \left[\begin{array}{rrrrrrrrr|rr}
  -2 & 1 & & & & & & & & & \\
  1 & -2 & 1 & & & & & & & & \\
  & 1 & -2 & 1 & & & & & & & \\
  & & & & & & & & & & \\
  & & & & & & & & & 1 & \\
  & & & & & \ddots & & & & \vdots & \\
  & & & & & & & & & & 1 \\
  & & & & & & & & & & \\
  & & & & & & & 1 & -2 & & \\ \hline
  & & & & 1 & \dots & & & & -r & \\
  & & & & & & 1 & & & & -r
\end{array}\right].
\]
The $1$s in the upper right rectangle in the matrix occur at rows $r$ and $2r$---that is, at the rows corresponding to $D_r$ and $E_r$. The $1$s in the lower left rectangle similarly occur at the $r$th and $2r$th columns respectively. The component group can then be computed via the following.
\begin{thm}[Corollary 9.6.3 of \cite{blr}]\label{thm:elementary-divisors-comp-group}
  Suppose that $\scd$ is a flat, proper relative curve over a strictly henselian discrete valuation ring. Suppose also that $\scd$ is generically geometrically irreducible and that the residue field is algebraically closed. Let $n_1, \dots, n_{v-1}, 0$ be the elementary divisors of the intersection matrix $A$ for the special fiber of $\scd$. Let $J$ be the Jacobian of the generic fiber of $\scd$. Then the group of connected components of the N\'eron model of $J$ is isomorphic to
  \[
  {\Z}/{n_1\Z} \oplus \cdots \oplus {\Z}/{n_{v-1}\Z}.
  \]
\end{thm}
Implicit in the statement of the theorem is that the intersection matrix has rank $r-1$. In our case, one can at least see that the vector of multiplicities $(1,2,\dots,r,r,\dots,r,r-1,\dots,2,1,1,1)$ lies in the kernel of $A$; in fact, one can use Theorem~\ref{thm:fibral-intersect-total} to show that this holds in general.

We must therefore compute the elementary divisors of $A$. Recall that the elementary divisors are the diagonal entries in the Smith normal form of $A$. The Smith normal form is a diagonal matrix obtained from $A$ via a combination of unipotent elementary row and column operations. (Unipotency is required for invertibility of the operations over $\Z$.)

\paragraph{Step 1:}
\label{sec:step-1}

For $i = 1, \dots, 3r-3$ in turn, we do the following column operations:
\begin{itemize}
    \item Add twice column $i$ to column $i+1$.
    \item Subtract column $i$ from column $i+2$.
\end{itemize}
We then add twice column $3r-2$ to column $3r-1$. An easy induction shows that the resulting matrix is
\[
\left[\begin{array}{rrrrrrrr|rr}
  -2 & -3 & & & \dots & & & -3r & & \\
  1 & & & & & & & & & \\
  & 1 & & & & & & & & \\
  & & & & & & & & 1 & \\
  & & & & \ddots & & & & & \\
  & & & & & & & & & 1 \\
  & & & & & & 1 & 0 & & \\ \hline
  & & 1 & 2 & & \dots & & 2r & -r & \\
  & & & & 1 & 2 & \dots & r & & -r
\end{array}\right].
\]
Once again, the $1$s in the upper right rectangle occur at rows $r$ and $2r$ respectively, and the $1$s in the lower left rectangle occur in columns $r$ and $2r$ respectively.

\paragraph{Step 2:}
\label{sec:step-2}

We now subtract column $r-1$ from column $3r$ (the second column from the right), and subtract column $2r-1$ from column $3r+1$:
\[
\left[\begin{array}{rrrrrrrr|rr}
  -2 & -3 & & & \dots & & & -3r & r & 2r \\
  1 & & & & & & & & & \\
  & 1 & & & & & & & & \\
  & & & & \ddots & & & & & \\
  & & & & & & 1 & 0 & & \\ \hline
  & & 1 & 2 & & \dots & & 2r & -r & -r \\
  & & & & 1 & 2 & \dots & r & & -r
\end{array}\right].
\]

\paragraph{Step 3:}
\label{sec:step-3}

By the obvious row operations, we can zero out the first $3r-2$ entries of the first row and the last two rows. We then rearrange rows to obtain
\[
\left[\begin{array}{c|rrr}
  I & & & \\ \hline
  & -3r & r & 2r \\
  & 2r & -r & -r \\
  & r & & -r
\end{array}\right].
\]
Here $I$ denotes the $(3r-2) \times (3r-2)$ identity matrix. By Theorem~\ref{thm:elementary-divisors-comp-group}, we may consider only the $3 \times 3$ block in the bottom right and omit everything else. A short calculation shows that the Smith normal form for this latter block is
\[
\begin{bmatrix}
  r & & \\
  & r & \\
  & & 0
\end{bmatrix}.
\]
Now applying Theorem~\ref{thm:elementary-divisors-comp-group}, we obtain
\begin{thm}
  Let $C$ be the complete nonsingular curve birational to the affine curve with equation $\caff:Txy^r = (x+1)(Tx+1)$ over $K$, with $r \geq 3$. Let $\scd$ be the minimal proper regular model for $C$ at $T = 0$. Let $J$ be the Jacobian for $C$. Then the component group for the N\'eron model of $J$ at $T = 0$ is isomorphic to
  \[
  {\Z}/{r\Z} \oplus {\Z}/{r\Z}.
  \]
\end{thm}

\begin{thebibliography}{BLR90}

\bibitem{bosch-liu1999}
Siegfried Bosch and Qing Liu.
\newblock Rational points of the group of components of a {N}\'eron model.
\newblock {\em Manuscripta Math.}, 98(3):275--293, 1999.

\bibitem{blr}
Siegfried Bosch, Werner L{\"u}tkebohmert, and Michel Raynaud.
\newblock {\em N\'eron models}, volume~21 of {\em Ergebnisse der Mathematik und
  ihrer Grenzgebiete (3) [Results in Mathematics and Related Areas (3)]}.
\newblock Springer-Verlag, Berlin, 1990.

\bibitem{bps}
Nils Bruin, Bjorn Poonen, and Michael Stoll.
\newblock Generalized explicit descent and its application to curves of genus 3.
\newblock \url{http://arxiv.org/abs/1205.4456}, 2012.


%\bibitem[C01]{Cornelissen}
%Gunther Cornelissen.  
%\newblock Two-torsion in the Jacobian of hyperelliptic curves over finite fields. 
%\newblock {\em Arch.\ Math.\ (Basel)} {\bf 77} (2001), no.\ 3, 241�-246.

\bibitem{ps}
Bjorn Poonen and Edward F.\ Schaefer.
\newblock  Explicit descent for Jacobians of cyclic covers of the projective line.
\newblock   J.\ Reine Angew.\ Math.\ 488, 141--188, 1997.

\bibitem{silvermanATAEC}
Joseph~H.\ Silverman.
\newblock {\em Advanced topics in the arithmetic of elliptic curves}.
\newblock Springer-Verlag, New York, 1994.

\bibitem{Legendre}
Douglas Ulmer.
\newblock Explicit points on the Legendre curve.
\newblock \url{http://arxiv.org/abs/1002.3313}.

\end{thebibliography}

\end{document}
\documentclass[pagesize,paper=letter]{scrartcl}
%\documentclass{article}
\usepackage{amsmath, amsthm, amssymb}
\usepackage[T1]{fontenc}
\usepackage{lmodern}
\usepackage{microtype}
\usepackage[all]{xy}
\usepackage[pagebackref,colorlinks]{hyperref}
\usepackage{mathrsfs}

% Color comments!
\usepackage[usenames,dvipsnames]{color}
% Color comments

\newcommand{\sce}{\mathscr{C}^{\textsf{ex}}}
\newcommand{\scd}{\mathscr{C}}
\newcommand{\caff}{C_{\textsf{aff}}}
\newcommand{\sj}{\mathscr{J}}


\theoremstyle{plain}
\newtheorem*{reftheorem}{Theorem}
\newtheorem{theorem}{Theorem}[section]
\newtheorem{corollary}[theorem]{Corollary}
\newtheorem{proposition}[theorem]{Proposition}
\newtheorem{lemma}[theorem]{Lemma}
\newtheorem{conjecture}[theorem]{Conjecture}
\newtheorem{problem}{Problem}
\newtheorem{question}{Question}
\newtheorem*{question*}{Question}
\newtheorem{claim}{Claim}

\theoremstyle{definition}
\newtheorem{definition}[theorem]{Definition}

\theoremstyle{remark}
\newtheorem{remark}[theorem]{Remark}
\newtheorem{example}[theorem]{Example}

% General
\renewcommand{\emptyset}{\varnothing}
\newcommand{\hra}{\hookrightarrow}
\newcommand{\righthookarrow}{\hookrightarrow}
\newcommand{\isom}{\cong}
\newcommand{\too}{\longrightarrow}
\newcommand{\isomto}{\overset{\sim}{\longrightarrow}}
\newcommand{\nto}[1]{\overset{#1}{\longrightarrow}}
\newcommand{\nsubset}{\not\subset}
\renewcommand{\phi}{\varphi}
\newcommand{\To}{\Rightarrow}
\newcommand{\ilim}{\displaystyle\lim_{\leftarrow}}
\newcommand{\dirlim}{\displaystyle\lim_{\rightarrow}}
\newcommand{\eps}{\varepsilon}
\renewcommand{\bar}[1]{\overline{#1}}
\renewcommand{\tilde}[1]{\widetilde{#1}}
\DeclareMathOperator{\car}{char}
\DeclareMathOperator{\rk}{rk}
\DeclareMathOperator{\coker}{coker}
\DeclareMathOperator{\Hom}{Hom}
\DeclareMathOperator{\Aut}{Aut}
\DeclareMathOperator{\End}{End}
\DeclareMathOperator{\im}{im}
\DeclareMathOperator{\pgl}{PGL}
\DeclareMathOperator{\Gl}{GL}
\DeclareMathOperator{\Sl}{SL}

% Number theory
\newcommand{\Qbar}{\ensuremath{\overline{\Q}}}
\newcommand{\Kb}{\overline{K}}
\newcommand{\Fb}{\overline{F}}
\newcommand{\kb}{\overline{k}}
\newcommand{\Xbar}{\overline{X}}
\newcommand{\Cbar}{\overline{C}}
\newcommand{\R}{\ensuremath{\mathbb{R}}}
\newcommand{\C}{\ensuremath{\mathbb{C}}}
\newcommand{\F}{\ensuremath{\mathbb{F}}}
\newcommand{\fp}{\ensuremath{\mathbb{F}_p}}
\newcommand{\sm}{\ensuremath{\mathfrak{m}}}
\newcommand{\Q}{\ensuremath{\mathbb{Q}}}
\newcommand{\Z}{\ensuremath{\mathbb{Z}}}
\newcommand{\ok}{\mathscr{O}_K}
\DeclareMathOperator{\Gal}{Gal}
\DeclareMathOperator{\inv}{inv}
\DeclareMathOperator{\Nm}{Nm}
\DeclareMathOperator{\tr}{Tr}

% Algebraic geometry
\newcommand{\sA}{\ensuremath{\mathscr{A}}}
\newcommand{\sO}{\ensuremath{\mathscr{O}}}
\newcommand{\sL}{\ensuremath{\mathscr{L}}}
\newcommand{\sK}{\ensuremath{\mathscr{K}}}
\newcommand{\sF}{\ensuremath{\mathscr{F}}}
\newcommand{\A}{\ensuremath{\mathbb{A}}}
\newcommand{\Pro}{\ensuremath{\mathbb{P}}}
\newcommand{\G}{\ensuremath{\mathbb{G}}}
\newcommand{\sG}{\mathscr{G}}
\newcommand{\sX}{\mathscr{X}}
\DeclareMathOperator{\Supp}{Supp}
\DeclareMathOperator{\Div}{Div}
\DeclareMathOperator{\dv}{div}
\DeclareMathOperator{\Pic}{Pic}
\DeclareMathOperator{\P0}{Pic^0}
\DeclareMathOperator{\Spec}{Spec}

\begin{document}

\title{Regular model at $t=1$}
\maketitle

Let $k$ be the finite field $\F_q$ and $K = k(t)$. Let $r \geq 3$ be an integer. Let $C$ be the smooth projective curve with affine model
\[
C: xy^r = (x+1)(x+t).
\]
The purpose of this note is to compute the minimal proper regular model of $C$ at $t = 1$, as well as the component group.

\section{Desingularization}
\label{sec:desingularization}

\subsection{Generic fiber}
\label{sec:generic-fiber}

Our first step is to substitute $x_0 = x+1$ and $t_0 = t-1$, so that our equation becomes
\[
(x_0 - 1) y^r = x_0(x_0 + t_0).
\]
This means we are concerned with the fiber $t_0 = 0$. Furthermore, there is a nonregular point at $(x_0, y, t_0) = (0, 0, 0)$. Before blowing-up, we must find a model which is generically smooth. We do this by gluing together the equations
\begin{align*}
  \caff&: (x_0 - 1) y^r = x_0(x_0 + t_0)\\
  C^\dagger&: (1-z)z = u^r(1+t_0z) \\
  C^\ddag&: v-u = vu^{r-1}(v + t_0 u)
\end{align*}
via
\begin{gather*}
  x_0 = \frac{1}{z} = \frac{v}{u} \qquad y = \frac{1}{u} \\
  u = \frac{1}{y} \qquad z = \frac{1}{x_0} \\
  v = \frac{x_0}{y} = \frac{1}{uz}.
\end{gather*}
One verifies that the resulting curve is generically smooth. (Note: the chart $C^\ddag$ is obtained by normalizing the projective closure of $\caff$ at a single cusp.) Furthermore, one can show that $C^\dag$ and $C^\ddag$ are regular.

\subsection{Blow-ups}
\label{sec:blow-ups}

It remains to resolve the singularity at $(x_0, y, t_0) = (0, 0, 0)$. To do this, we consider the family
\[
(x_iy^i - 1) y^k = x_i(x_i + t_i)
\]
with central fiber $t_iy^i = 0$; call this curve chart $(i,k)$. Observe that our $\caff$ is exactly the chart $(0,r)$. For most values of $i, k$, the chart will be nonregular at exactly one point: $(x_i, y, t_i) = (0, 0, 0)$. As we will see, a blow-up results in 3 charts, 2 of which are regular, and the third of which is chart $(i+1, k-2)$. The process terminates when chart $(i, k)$ is regular; that is, for sufficiently small $k$. This will turn out to occur when $k = 0$ or $1$. We first study properties of the charts $(i, k)$ in 4 cases:
\begin{enumerate}
    \item The initial chart $(0, k)$ with $k = r$; in particular, $k \geq 3$.
    \item The terminal chart $(i, 0)$ with $i \geq 1$.
    \item The terminal chart $(i, 1)$ with $i \geq 1$.
    \item The generic chart $(i, k)$ with $i \geq 1$, $k \geq 2$.
\end{enumerate}
Once this is done, we demonstrate the recursive blow-up procedure.

\paragraph{Case $(0, k)$ with $k \geq 3$:}
\label{sec:case-i=0}

If $i = 0$, we have $\caff$ with special fiber
\[
(x_0 - 1) y^k = x_0^2.
\]
The fiber is irreducible with multiplicity $1$, and we call it $F_0$. One checks that $F_0$ intersects itself at the $k + 1$ points given by $y = 0$ or $y^k = 4$. The latter $k$ points are transverse crossings, but under our hypothesis that $k \geq 3$, the point $(x_0, y) = (0, 0)$ is a singularity of higher order. As observed earlier, the only nongregular point on $\caff$ is given by $(x_0, y, t_0) = (0, 0, 0)$.

\paragraph{Case $(i, 0)$ with $i \geq 1$:}
\label{sec:case-k=0}




\paragraph{Case $(i, 1)$ with $i \geq 1$:}
\label{sec:case-i-1}


\paragraph{Generic case:}
\label{sec:generic-case}


% \paragraph{Case $0 < i < \lfloor r/2 \rfloor$:}
% \label{sec:case-0-leq}

% The special fiber of chart $(i)$ has components
% \begin{itemize}
%     \item $F_0: t_i = (x_iy^i - 1) y^{r-2i} - x_i^2 = 0$ with multiplicity $1$;
%     \item $A_i: y_i = x_i = 0$ with multiplicity $$; and
%     \item $B_i: y_i = x_i + t_i = 0$ with multiplicity $$.
% \end{itemize}
% As observed earlier, the curve is not regular at $(x_i, y, t_i) = (0,0,0)$. We blow up at the latter point by replacing it with 3 charts, given as follows.

% The first chart is $(i+1)$ with gluing $x_i = yx_{i+1}$ and $t_i = yt_{i+1}$. The second chart, which we call $(i)a$, is given by

\bibliographystyle{alpha}
\bibliography{./aimbiblio}
\end{document}
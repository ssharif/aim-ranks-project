\documentclass[pagesize,paper=letter]{scrartcl}
%\documentclass{article}
\usepackage{amsmath, amsthm, amssymb}
\usepackage[T1]{fontenc}
\usepackage{lmodern}
\usepackage{microtype}
\usepackage[all]{xy}
\usepackage[pagebackref,colorlinks]{hyperref}
\usepackage{mathrsfs}

% Color comments!
\usepackage[usenames,dvipsnames]{color}
% Color comments

\newcommand{\sce}{\mathscr{C}^{\textsf{ex}}}
\newcommand{\scd}{\mathscr{C}}
\newcommand{\caff}{C_{\textsf{aff}}}
\newcommand{\sj}{\mathscr{J}}


\theoremstyle{plain}
\newtheorem*{reftheorem}{Theorem}
\newtheorem{theorem}{Theorem}[section]
\newtheorem{corollary}[theorem]{Corollary}
\newtheorem{proposition}[theorem]{Proposition}
\newtheorem{lemma}[theorem]{Lemma}
\newtheorem{conjecture}[theorem]{Conjecture}
\newtheorem{problem}{Problem}
\newtheorem{question}{Question}
\newtheorem*{question*}{Question}
\newtheorem{claim}{Claim}

\theoremstyle{definition}
\newtheorem{definition}[theorem]{Definition}

\theoremstyle{remark}
\newtheorem{remark}[theorem]{Remark}
\newtheorem{example}[theorem]{Example}

% General
\renewcommand{\emptyset}{\varnothing}
\newcommand{\hra}{\hookrightarrow}
\newcommand{\righthookarrow}{\hookrightarrow}
\newcommand{\isom}{\cong}
\newcommand{\too}{\longrightarrow}
\newcommand{\isomto}{\overset{\sim}{\longrightarrow}}
\newcommand{\nto}[1]{\overset{#1}{\longrightarrow}}
\newcommand{\nsubset}{\not\subset}
\renewcommand{\phi}{\varphi}
\newcommand{\To}{\Rightarrow}
\newcommand{\ilim}{\displaystyle\lim_{\leftarrow}}
\newcommand{\dirlim}{\displaystyle\lim_{\rightarrow}}
\newcommand{\eps}{\varepsilon}
\renewcommand{\bar}[1]{\overline{#1}}
\renewcommand{\tilde}[1]{\widetilde{#1}}
\DeclareMathOperator{\car}{char}
\DeclareMathOperator{\rk}{rk}
\DeclareMathOperator{\coker}{coker}
\DeclareMathOperator{\Hom}{Hom}
\DeclareMathOperator{\Aut}{Aut}
\DeclareMathOperator{\End}{End}
\DeclareMathOperator{\im}{im}
\DeclareMathOperator{\pgl}{PGL}
\DeclareMathOperator{\Gl}{GL}
\DeclareMathOperator{\Sl}{SL}

% Number theory
\newcommand{\Qbar}{\ensuremath{\overline{\Q}}}
\newcommand{\Kb}{\overline{K}}
\newcommand{\Fb}{\overline{F}}
\newcommand{\kb}{\overline{k}}
\newcommand{\Xbar}{\overline{X}}
\newcommand{\Cbar}{\overline{C}}
\newcommand{\R}{\ensuremath{\mathbb{R}}}
\newcommand{\C}{\ensuremath{\mathbb{C}}}
\newcommand{\F}{\ensuremath{\mathbb{F}}}
\newcommand{\fp}{\ensuremath{\mathbb{F}_p}}
\newcommand{\sm}{\ensuremath{\mathfrak{m}}}
\newcommand{\Q}{\ensuremath{\mathbb{Q}}}
\newcommand{\Z}{\ensuremath{\mathbb{Z}}}
\newcommand{\ok}{\mathscr{O}_K}
\DeclareMathOperator{\Gal}{Gal}
\DeclareMathOperator{\inv}{inv}
\DeclareMathOperator{\Nm}{Nm}
\DeclareMathOperator{\tr}{Tr}

% Algebraic geometry
\newcommand{\sA}{\ensuremath{\mathscr{A}}}
\newcommand{\sO}{\ensuremath{\mathscr{O}}}
\newcommand{\sL}{\ensuremath{\mathscr{L}}}
\newcommand{\sK}{\ensuremath{\mathscr{K}}}
\newcommand{\sF}{\ensuremath{\mathscr{F}}}
\newcommand{\A}{\ensuremath{\mathbb{A}}}
\newcommand{\Pro}{\ensuremath{\mathbb{P}}}
\newcommand{\G}{\ensuremath{\mathbb{G}}}
\newcommand{\sG}{\mathscr{G}}
\newcommand{\sX}{\mathscr{X}}
\DeclareMathOperator{\Supp}{Supp}
\DeclareMathOperator{\Div}{Div}
\DeclareMathOperator{\dv}{div}
\DeclareMathOperator{\Pic}{Pic}
\DeclareMathOperator{\P0}{Pic^0}
\DeclareMathOperator{\Spec}{Spec}

\begin{document}

\title{Regular model at $t=1$}
\author{Shahed Sharif}
\maketitle

Let $k$ be the finite field $\F_q$ and $K = k(t)$. Let $r \geq 3$ be an integer. Let $C$ be the smooth projective curve with affine model
\[
C: xy^r = (x+1)(x+t).
\]
The purpose of this note is to compute the minimal proper regular model of $C$ at $t = 1$, as well as the component group.

\section{Desingularization}
\label{sec:desingularization}

\subsection{Generic fiber}
\label{sec:generic-fiber}

Our first step is to substitute $x_0 = x+1$ and $t_0 = t-1$, so that our equation becomes
\[
(x_0 - 1) y^r = x_0(x_0 + t_0).
\]
This means we are concerned with the fiber $t_0 = 0$. Furthermore, there is a nonregular point at $(x_0, y, t_0) = (0, 0, 0)$. Before blowing-up, we must find a model which is generically smooth. We do this by gluing together the equations
\begin{align*}
  \caff&: (x_0 - 1) y^r = x_0(x_0 + t_0)\\
  C^\dagger&: (1-z)z = u^r(1+t_0z) \\
  C^\ddag&: v-u = vu^{r-1}(v + t_0 u)
\end{align*}
via
\begin{gather*}
  x_0 = \frac{1}{z} = \frac{v}{u} \qquad y = \frac{1}{u} \\
  u = \frac{1}{y} \qquad z = \frac{1}{x_0} \\
  v = \frac{x_0}{y} = \frac{1}{uz}.
\end{gather*}
One verifies that the resulting curve is generically smooth. (Note: the chart $C^\ddag$ is obtained by normalizing the projective closure of $\caff$ at a single cusp.) Furthermore, one can show that $C^\dag$ and $C^\ddag$ are regular.

\subsection{Blow-ups}
\label{sec:blow-ups}

It remains to resolve the singularity at $(x_0, y, t_0) = (0, 0, 0)$. To do this, we consider the family
\[
(x_iy^i - 1) y^k = x_i(x_i + t_i)
\]
with central fiber $t_iy^i = 0$; call this curve chart $(i,k)$. Observe that our $\caff$ is exactly the chart $(0,r)$. For most values of $i, k$, the chart will be nonregular at exactly one point: $(x_i, y, t_i) = (0, 0, 0)$. As we will see, a blow-up results in 3 charts, 2 of which are regular, and the third of which is chart $(i+1, k-2)$. The process terminates when chart $(i, k)$ is regular; that is, for sufficiently small $k$. This will turn out to occur when $k = 0$ or $1$. We first study properties of the charts $(i, k)$ in 4 cases:
\begin{enumerate}
    \item The initial chart $(0, k)$ with $k = r$; in particular, $k \geq 3$.
    \item The terminal chart $(i, 0)$ with $i \geq 1$.
    \item The terminal chart $(i, 1)$ with $i \geq 1$.
    \item The generic chart $(i, k)$ with $i \geq 1$, $k \geq 2$.
\end{enumerate}
Once this is done, we demonstrate the recursive blow-up procedure.

\paragraph{Chart $(0, k)$ with $k \geq 3$:}
\label{sec:case-i=0}

If $i = 0$, we have $\caff$ with special fiber
\[
(x_0 - 1) y^k = x_0^2.
\]
The fiber is irreducible with multiplicity $1$, and we call it $F$. One checks that $F$ intersects itself at point $(x_0, y) = (0, 0)$. As observed earlier, the only nonregular point on $\caff$ is given by $(x_0, y, t_0) = (0, 0, 0)$.

\paragraph{Chart $(i, 0)$ with $i \geq 1$:}
\label{sec:case-k=0}

The special fiber consists of two components:
\begin{itemize}
    \item $F_{i,0}: t_i = x_i^2 - x_iy^i + 1 = 0$ with multiplicity 1; and
    \item $G_{i,0}: y = x_i^2 + x_it_i + 1 = 0$ with multiplicity $i$.
\end{itemize}
These components intersect twice transversally, at $(x_i, y, t_i) = (\pm i, 0, 0)$. This chart is regular.



\paragraph{Chart $(i, 1)$ with $i \geq 1$:}
\label{sec:case-i-1}

The special fiber consists of 3 components:
\begin{itemize}
    \item $F_{i,1}: t_i = x_i^2 - x_iy^{i+1} + y = 0$ with multiplicity 1;
    \item $D_{i,1}: y = x_i = 0$ with multiplicity $i$; and
    \item $E_{i,1}: y = x_i + t_i = 0$ with multiplicity $i$.
\end{itemize}
These components intersect each other transversally at $(x_i, y, t_i) = (0,0,0)$, and nowhere else. This chart is regular.



\paragraph{Chart $(i, k)$ with $i\geq 1$, $k\geq 2$:}
\label{sec:higher-i-k}

The special fiber consists of 3 components:
\begin{itemize}
    \item $F_{i,k}: t_i = x_i^2 - x_iy^{i+k} + y^k = 0$ with multiplicity 1;
    \item $D_{i,k}: y = x_i = 0$ with multiplicity $i$; and
    \item $E_{i,k}: y = x_i + t_i = 0$ with multiplicity $i$.
\end{itemize}
The component $F_{i,k}$ has a node at $(x_i,y,t_i) = (0,0,0)$, and the other two components also pass through this point. This intersection point is not regular, but the rest of the chart is regular.

The only nonregular charts are $(0,k)$ and the last chart above. Thus we now investigate the blow-up of these charts at $(x_i,y,t_i) = (0,0,0)$. The first chart of the blow-up is obtained via $x_{i+1} = x_i y$, $t_{i+1} = t_i y$; one verifies that the resulting chart is $(i+1, k-2)$, where the strict transform of $F_{i,k}$ is $F_{i+1, k-2}$, and the other 2 components of chart $(i + 1, k-2)$ are exceptional. There are two more charts in the blow-up, however the specifics depend on whether $k=2$ or $k \geq 3$. In all cases, as we shall see, these charts are regular.

\paragraph{Chart $(i,k)_b$:}
\label{sec:chart-i-k_b}

Our next chart, which we call $(i,k)_b$, is obtained via $y = x_ib_i$, $t_i = x_i\tau_i$, whence the central fiber is given by $t_0 = x_i^{i+1}b_i^i\tau_i$. The resulting model is
\[
(x_i^{i+1}b_i^i - 1)x_i^{k-2} b_i^k = 1 + \tau_i.
\]
The special fiber has 3 components. When $k \geq 3$, one verifies that these are
\begin{itemize}
    \item $F_{i,k}: \tau_i = (x_i^{i+1}b_i^i - 1)x^{k-2} b_i^k - 1 = 0$ with multiplicity 1;
    \item $E_{i,k}: b_i = 1+\tau_i = 0$ with multiplicity $i$; and
    \item $E_{i+1,k}: x_i = 1+\tau_i = 0$ with multiplicity $i+1$.
\end{itemize}
Note that if $i = 0$, $E_{i,k}$ has multiplicity $0$; this actually means that the component does not exist in this case. Otherwise,
the components $E_{i,k}$ and $E_{i+1,k}$ cross transversally at $(x_i, b_i, \tau_i) = (0, 0, -1)$. The component $F_{i,k}$ does not intersect the others regardless of the value of $i$.

When $k = 2$, we get
\begin{itemize}
    \item $F_{i,2}: \tau_i = (x_i^{i+1}b_i^i - 1) b_i^2 - 1 = 0$ with multiplicity 1;
    \item $E_{i,2}: b_i = 1+\tau_i = 0$ with multiplicity $i$; and
    \item $G_{i+1,2}: x_i = 1+\tau_i+b_i^2 = 0$ with multiplicity $i+1$.
\end{itemize}
The intersection $E_{i,2} \cap G_{i+1,2}$ is transversal at $(0,0,-1)$, while $F_{i,2}$ intersects $G_{i+1,2}$ transversally at two points: $(x_i, b_i, \tau_i) = (0, \pm i, 0)$. One checks that these latter intersections are identical to those in chart $(i+1,0)$.

Independent of the value of $k$, one verifies that this chart is regular by computing the $\tau_i$-derivative.

\paragraph{Chart $(i,k)_u$:}
\label{sec:chart-i-k_u}

This last chart is obtained via $x_i = u_it_i$, $y = v_it_i$, giving the model
\[
(u_iv_i^it_i^{i+1} - 1) v_i^k t_i^{k-2} = u_i(u_i + 1).
\]
The central fiber is given by $t_0 = t_i^{i+1}v_i^i$. One checks that the special fiber when $k \geq 3$ consists of the components
\begin{itemize}
    \item $D_{i,k}: v_i = u_i = 0$ with multiplicity $i$;
    \item $E_{i,k}: v_i = u_i + 1 = 0$ with multiplicity $i$;
    \item $D_{i+1,k}: t_i = u_i = 0$ with multiplicity $i+1$; and
    \item $E_{i+1,k}: t_i = u_i + 1 = 0$ with multiplicity $i+1$.
\end{itemize}
As with the chart $(i,k)_b$, multiplicity $0$ components mean that such components are absent when $i = 0$.
The components $D_{i+1,k}$ and $D_{i,k}$ (when the latter exists) intersect transversally once at $(u_i, v_i, t_i) = (0,0,0)$, and the components $E_{i+1,k}$ and $E_{i,k}$ intersect transversally at $(-1,0,0)$. One checks that the latter intersection point is identical to the intersection point in chart $(i,k)_b$. There are no other intersections.

When $k = 2$, we get
\begin{itemize}
    \item $D_{i,2}: v_i = u_i = 0$ with multiplicity $i$;
    \item $E_{i,2}: v_i = u_i + 1 = 0$ with multiplicity $i$; and
    \item $G_{i+2,2}: t_i = u_i^2 + u_i + v_i^2 = 0$ with multiplicity $i+1$.
\end{itemize}
The component $G_{i+1,2}$ intersects each of $D_{i,2}, E_{i,2}$ once transversally. The intersection with $E_{i,2}$ is the same as in chart $(i,2)_b$.

When $k \geq 3$, the $u_i$-derivative shows that this chart is regular. When $k = 2$, the $u_i$-derivative is zero if and only if $u_i = -\frac12$. Substituting into the $v_i$-derivative shows that the chart is regular.


\subsubsection{Putting the charts together}
\label{sec:putt-charts-togeth}

We start with the chart $(0, r)$ and blow up, inductively obtaining charts of the form $(i, r-2i)$, $(i, r-2i)b$ and $(i, r-2i)u$. The procedure terminates upon reaching a chart of the form $(s, 1)$ or $(s, 0)$; thus we treat the cases of $r$ even and odd separately. Since $r$ is fixed, the value of $k$ for each case is a function of $i$. Therefore we will frequently leave off reference to $k$ in all subscripts. Notice that all components of the form $F_{\cdot}$ are glued together, and so we may leave off the subscript entirely. Lastly, the components $D_i$ and $E_i$ each have multiplicity $i$.

We first suppose that $r$ is odd. Let $s = \frac12(r-1)$. Then the resulting regular proper model consists of the 3-fold intersection of components $F$, $D_s$, and $E_s$ as in chart $(s, 1)$, plus two ``tails'': $D_s$ connected to $D_{s-1}$, which is then connected to $D_{s-2}, \dots$, terminating with $D_1$, and similarly $E_s$ connected to $E_{s-1}, \dots, E_1$. The associated reduced curve is not semi-stable, but I decided to bend the rules slightly to come up with the ``dual graph-like object'' in Figure~\ref{fig:superelliptic-dual-graph-odd}. In that figure, the $\ast$ denotes the triple intersection point of $F$, $D_{s-1}$, and $E_{s-1}$.
\begin{figure}[h]\centering
    \[
\xygraph{
  !{<0cm,0cm>;<1.5cm,0cm>:<0cm,1.25cm>::}
  !{(2,5.5) }*{F}
  !{(2,5) }*{\bullet}="f"
  !{(2,4) }*{\ast}="triple"
  !{(.5,3) }*{D_{s-1}}
  !{(1,3) }*{\bullet}="ds"
  !{(1,2.6) }*{}="dabove"
  !{(1,2.1) }*{\vdots}="dspace"
  !{(1,1.5) }*{}="dbelow"
  !{(1,1) }*{\bullet}="d2"
  !{(.5,1) }*{D_2}
  !{(1,0) }*{\bullet}="d1"
  !{(.5,0) }*{D_1}
  !{(3.5,3) }*{E_{s-1}}
  !{(3,3) }*{\bullet}="es"
  !{(3,2.6) }*{}="eabove"
  !{(3,2.1) }*{\vdots}="espace"
  !{(3,1.5) }*{}="ebelow"
  !{(3,1) }*{\bullet}="e2"
  !{(3.5,1) }*{E_2}
  !{(3,0) }*{\bullet}="e1"
  !{(3.5,0) }*{E_1}
  % !{(4,0) }*{\bullet}="e1"
  % !{(4,1) }*{\bullet}="e2"
  % !{(4,1.5) }*{}="ebelow"
  % !{(4,2.1) }*{\vdots}="espace"
  % !{(4,2.6) }*{}="eabove"
  % !{(4,3) }*{\bullet}="erm"
  % !{(4,4) }*{\bullet}="er"
  % !{(4.4,0) }*{E_1}
  % !{(4.4,1) }*{E_2}
  % !{(4.6,3) }*{E_{r-1}}
  % !{(3.6,4) }*{E_r}
  "f"-"triple"
  "triple"-"ds"
  "ds"-"dabove"
  "dbelow"-"d2"
  "d2"-"d1"
  "triple"-"es"
  "es"-"eabove"
  "ebelow"-"e2"
  "e2"-"e1"
}
\]



%%% Local Variables:
%%% TeX-master: "AIM_Ranks_one"
%%% End:

  \caption{Dual graph of $C_k$, $t=1$, $r$ odd}
\label{fig:superelliptic-dual-graph-odd}
\end{figure}

Now suppose $r$ is even. Let $s = \frac12 r$. Then we end up with $F$ connected to $G_s$ by 2 intersection points, and $G_s$ has 2 ``tails'' attached to it at distinct places: $D_{s-1}, \dots, D_1$ and $E_{s-1}, \dots, E_1$ respectively. In this case, the associated reduced curve is semi-stable, so that we may construct the dual graph in Figure~\ref{fig:superelliptic-dual-graph-even}.
\begin{figure}[h]\centering
    \[
\xygraph{
  !{<0cm,0cm>;<1.5cm,0cm>:<0cm,1.25cm>::}
  !{(2,6) }*{F}
  !{(2,5.5) }*{\bullet}="f"
  !{(2,4) }*{\bullet}="g"
  !{(2.5,4) }*{G_s}
  !{(.5,3) }*{D_{s-1}}
  !{(1,3) }*{\bullet}="ds"
  !{(1,2.6) }*{}="dabove"
  !{(1,2.1) }*{\vdots}="dspace"
  !{(1,1.5) }*{}="dbelow"
  !{(1,1) }*{\bullet}="d2"
  !{(.5,1) }*{D_2}
  !{(1,0) }*{\bullet}="d1"
  !{(.5,0) }*{D_1}
  !{(3.5,3) }*{E_{s-1}}
  !{(3,3) }*{\bullet}="es"
  !{(3,2.6) }*{}="eabove"
  !{(3,2.1) }*{\vdots}="espace"
  !{(3,1.5) }*{}="ebelow"
  !{(3,1) }*{\bullet}="e2"
  !{(3.5,1) }*{E_2}
  !{(3,0) }*{\bullet}="e1"
  !{(3.5,0) }*{E_1}
  % !{(4,0) }*{\bullet}="e1"
  % !{(4,1) }*{\bullet}="e2"
  % !{(4,1.5) }*{}="ebelow"
  % !{(4,2.1) }*{\vdots}="espace"
  % !{(4,2.6) }*{}="eabove"
  % !{(4,3) }*{\bullet}="erm"
  % !{(4,4) }*{\bullet}="er"
  % !{(4.4,0) }*{E_1}
  % !{(4.4,1) }*{E_2}
  % !{(4.6,3) }*{E_{r-1}}
  % !{(3.6,4) }*{E_r}
  "f"-@/^0.5cm/"g"
  "f"-@/_0.5cm/"g"
  "g"-"ds"
  "ds"-"dabove"
  "dbelow"-"d2"
  "d2"-"d1"
  "g"-"es"
  "es"-"eabove"
  "ebelow"-"e2"
  "e2"-"e1"
}
\]



%%% Local Variables:
%%% TeX-master: "AIM_Ranks_one"
%%% End:

  \caption{Dual graph of $C_k$, $t=1$, $r$ even}
\label{fig:superelliptic-dual-graph-even}
\end{figure}

One checks that the self-intersection number of each component is $-2$, except for $F$, which has $F^2 = -r$. By Castelnuovo's criterion, the model is minimal. Our calculations above show that the model is regular. Therefore we have found the minimal proper regular model we were looking for.

\section{Component group}
\label{sec:component-group}

We follow a similar procedure as when $t = \infty$ to compute the component group. 

\subsection{Case 1: $r$ odd}
\label{sec:case-1:-r-odd}

Recall that $r = 2s + 1$. Put the components of $C_k$ in the order $D_1, D_2, \dots, D_{s}, E_{s}, \dots, E_1, F$. With respect to this ordering, the intersection matrix is the $r \times r$ matrix
\[
A = \left[\begin{array}{rrrrrrrrr|rr}
  -2 & 1 & & & & & & & & \\
  1 & -2 & 1 & & & & & & & \\
  & 1 & -2 & 1 & & & & & & \\
  & & & & & \ddots & & & & 1 \\
  & & & & & & & & & 1 \\
  & & & & & & & & & \\
  & & & & & & & 1 & -2 & \\ \hline
  & & & & 1 & 1 & & & & -2s
\end{array}\right].
\]
The 1s in the right-hand column occur at rows $s$ and $s+1$, and similarly the 1s in the bottom row occur at columns $s$ and $s+1$.

\paragraph{Step 1:}
\label{sec:step-1-odd}

For $i = 1, \dots, 2s-1$ in turn, we do the following column operations:
\begin{itemize}
    \item Add twice column $i$ to column $i+1$.
    \item Subtract column $i$ from column $i+2$.
\end{itemize}
We then add twice column $2s-1$ to column $2s$. An easy induction shows that the resulting matrix is
\[
\left[\begin{array}{rrrrrrrr|r}
  -2 & -3 & & & \dots & & & -r & \\
  1 & & & & & & & & \\
  & 1 & & & & & & & \\
  & & & & \ddots & & & & 1 \\
  & & & &  & & & & 1 \\
  & & & &  & & & & \\
  & & & & & & 1 & 0 & \\ \hline
  & & & & 1 & 3 & \dots & r & -2s
\end{array}\right].
\]
The nonzero entries in the bottom row are the consecutive odd numbers from 1 to $r$.

\paragraph{Step 2:}
\label{sec:step-2-odd}

Zero out the first $r-2$ entries of the top and bottom rows:
\[
\left[\begin{array}{rrrrrrrr|r}
  0 & 0 & & & \dots & & & -r & r \\
  1 & & & & & & & & \\
  & 1 & & & & & & & \\
  & & & & \ddots & & & & 1 \\
  & & & &  & & & & 1 \\
  & & & &  & & & & \\
  & & & & & & 1 & 0 & \\ \hline
  & & & & 0 & 0 & \dots & r & -r
\end{array}\right].
\]

\paragraph{Step 3:}
\label{sec:step-3-odd}

Eliminate the 1s in the last column and swap rows and columns to obtain
\[
\left[\begin{array}{c|rr}
  I & & \\ \hline
 & -r & r \\
& r & -r
\end{array}\right]
\]
where $I$ is the $(r-2) \times (r-2)$ identity matrix. From this we conclude that the component group is $\Phi \isom \dfrac{\Z}{r\Z}$.

\subsection{Case 2: $r$ even}
\label{sec:case-2:-r-even}

Recall that $r = 2s$. Put the components of $C_k$ in the order $D_1, D_2, \dots, D_{s-1}, G_s, E_{s-1}, \dots, E_1, F$. With respect to this ordering, the intersection matrix is the $r \times r$ matrix
\[
A = \left[\begin{array}{rrrrrrrr|r}
  -2 & 1 & & & & & & & \\
  1 & -2 & 1 & & & & & & \\
  & 1 & -2 & 1 & & & & & \\
  & & & & & & & & \\
  & & & & \ddots & & & & 2 \\
  & & & & & & & & \\
  & & & & & & 1 & -2 & \\ \hline
  & & & & 2 &  & & & -r
\end{array}\right].
\]
The 2 in the right-hand column occurs at row $s$, and similarly the 2 on the bottom row is at column $s$.

\paragraph{Step 1:}
\label{sec:step-1-even}

For $i = 1, \dots, 2s-2$ in turn, we do the following column operations:
\begin{itemize}
    \item Add twice column $i$ to column $i+1$.
    \item Subtract column $i$ from column $i+2$.
\end{itemize}
We then add twice column $2s-2$ to column $2s-1$. An easy induction shows that the resulting matrix is
\[
\left[\begin{array}{rrrrrrrr|r}
  -2 & -3 & & & \dots & & & -r & \\
  1 & & & & & & & & \\
  & 1 & & & & & & & \\
  & & & & \ddots & & & & 2 \\
  & & & & & & 1 & 0 & \\ \hline
  & & & & 2 & 4 & \dots & r & -r
\end{array}\right].
\]
The 2 in the last column is in row $s$, and the 2 in the last row occurs in column $s$.

\paragraph{Step 2:}
\label{sec:step-2-even}

Zero out the first $r-2$ entries of the top and bottom rows:
\[
\left[\begin{array}{rrrrrrrr|r}
  0 & 0 & & & \dots & & & -r & r \\
  1 & & & & & & & & \\
  & 1 & & & & & & & \\
  & & & & \ddots & & & & 2 \\
  & & & & & & 1 & 0 & \\ \hline
  & & & & 0 & 0 & \dots & r & -r
\end{array}\right].
\]

\paragraph{Step 3:}
\label{sec:step-3-even}

Eliminate the 2 in the last column and swap rows and columns to obtain
\[
\left[\begin{array}{c|rr}
  I & & \\ \hline
 & -r & r \\
& r & -r
\end{array}\right]
\]
where $I$ is the $(r-2) \times (r-2)$ identity matrix. Thus in this case the component group is $\Phi \isom \dfrac{\Z}{r\Z}$ as well.

Therefore we have
\begin{theorem}
  Let $C$ be the complete nonsingular curve birational to the affine curve with equation $\caff:xy^r = (x+1)(x+t)$ over $K$, with $r \geq 3$. Let $\scd$ be the minimal proper regular model for $C$ at $t = 1$. Let $J$ be the Jacobian for $C$. Then the component group for the N\'eron model of $J$ at $t = 1$ is isomorphic to
  $
  \frac{\Z}{r\Z}.
  $
\end{theorem}

\bibliographystyle{alpha}
\bibliography{./aimbiblio}
\end{document}
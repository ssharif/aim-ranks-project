%\documentclass[reqno]{amsart}
\documentclass[pagesize,paper=letter]{scrartcl}
%\usepackage[english]{babel}
\usepackage[T1]{fontenc}
\usepackage{lmodern}
\usepackage{microtype}
\usepackage[all,cmtip]{xy}
\usepackage{amssymb,amsmath,mathrsfs,amsthm,amscd,hyperref,amsfonts,graphicx}
\bibliographystyle{plain}
\oddsidemargin 0mm
\evensidemargin 0mm
\textheight 214mm \textwidth 160mm
\headsep 10mm
\raggedbottom



\newtheorem{thm}{Theorem}[section]
\newtheorem{theorem}{Theorem}[section]
%\newtheorem{defn}[thm]{Definition}
\newtheorem{cor}[thm]{Corollary}
\newtheorem{conj}[thm]{Conjecture}
\newtheorem{lem}[thm]{Lemma}
\newtheorem{lemma}[thm]{Lemma}
\newtheorem{prop}[thm]{Proposition}
\newtheorem{proposition}[thm]{Proposition}
\newtheorem{nota}[thm]{Notation}
%\newtheorem{rem}[thm]{Remark}

\theoremstyle{definition}
\newtheorem{rem}[thm]{Remark}
\newtheorem{remark}[thm]{Remark}
\newtheorem{defn}[thm]{Definition}
\renewcommand{\thedefn}{}

\theoremstyle{remark}
\newtheorem{ex}{Example}
\newtheorem{notation}{Notation}
\newtheorem{exs}{Examples}
\renewcommand{\theex}{}
\renewcommand{\thenotation}{}
\renewcommand{\theexs}{}



%\theoremstyle{plain}
%\newtheorem{thm}[equation]{Theorem}
%\newtheorem{prop}[equation]{Proposition}
%\newtheorem{cor}[equation]{Corollary}
%\newtheorem{lemma}[equation]{Lemma}
%\newtheorem{conj}[equation]{Conjecture}

%\theoremstyle{definition}
%\newtheorem{defn}[equation]{Definition}
%\newtheorem{defns}[equation]{Definitions}
%
%\theoremstyle{remark}
%\newtheorem{rem}[equation]{Remark}
%\newtheorem{rems}[equation]{Remarks}
%\newtheorem{exer}[equation]{Exercise}
%\newtheorem{exers}[equation]{Exercises}
%\newtheorem{rem-exer}[equation]{Remark/Exercise}
%\newtheorem{rem-exers}[equation]{Remark/Exercises}
%\newtheorem{ex}[equation]{Example}
%\newtheorem{exs}[equation]{Examples}

\DeclareMathOperator{\dvsr}{div}
\newcommand{\one}{{\bf 1}}
\newcommand{\Div}{\operatorname{Div}}
\newcommand{\divi}{\operatorname{div}}
\newcommand{\sep}{\operatorname{sep}}
\newcommand{\XminusT}{(x-T)}
\newcommand{\xminusT}{(x-T)'}
\newcommand{\GG}{\mathcal{G}}
\newcommand{\iso}{\stackrel{\sim}{\rightarrow}}
\newcommand{\val}{\operatorname{val}}
\newcommand{\pr}{\operatorname{pr}}
\newcommand{\rk}{\operatorname{rank}}
\newcommand{\sy}{\bar{y}}
\newcommand{\vd}{\mathbf{d}}
\newcommand{\ve}{\mathbf{e}}


\def\bmu{{\boldsymbol\mu}}

\def \PP {{\mathbb P}^1}
\def \ZZ {{\mathbb Z}}
\def  \FF {{\mathbb F}}
\def \car {\mathop {\rm Car}}

\def\genus{\textrm{genus}}
\def\cond{\textrm{cond}}
\def\R{\mathbb{R}}
\def\Gal{\textrm{\upshape Gal}}
\def\ord{\textrm{\upshape ord}}
\def\p{\mathbb{P}}
\def\C{\mathbb{C}}
\def\Q{\mathbb{Q}}
%\def\A{\mathbf{A}}
\def\QQ{\overline{\mathbb{Q}}}
\def\Z{\mathbb{Z}}
\def\m{\mathfrak{m}}
\def\Oo{\mathcal{O}}
\def\F{\mathbb{F}}
\def\Hom{\mathrm{Hom}}
\def\tor{\mathrm{tor}}
\def \End{\mathrm{End}}
\def\hh{\textrm{H}}
\def\s{\textrm{\upshape Spec}\,}
\def\Br{\textnormal{Br}}
\def\pr{\textnormal{pr}}
\def\Nm{\textnormal{Nm}}
\def\can{\textnormal{can}}
\def\rank{\textnormal{rank}}
\def\Pic{\textnormal{Pic}}
\def\h{\textnormal{H}}
%\def\PGL{\textnormal{PGL}}
\def\torsion{\textnormal{tors}}
\def\tors{\textnormal{tors}}


% provide the letter sha:
\usepackage[OT2,T1]{fontenc}
\DeclareSymbolFont{cyrletters}{OT2}{wncyr}{m}{n}
\DeclareMathSymbol{\sha}{\mathalpha}{cyrletters}{"58}

\def\JJ{\mathcal{J}}
\def\XX{\mathcal{X}}
\def\YY{\mathcal{Y}}
\def\O{\mathcal{O}}
\def\OO{\mathcal{O}}



% abbreviations/alternate names
\def\<{\langle}
\def\>{\rangle}
\def\into{\hookrightarrow}
\def\onto{\twoheadrightarrow}
\def\isoto{\tilde{\to}}
\def\tensor{\otimes}
\def\compose{\circ}
\def\sdp{{\rtimes}}
\def\nodiv{\not|}
\def\PGL{\mathrm{PGL}}
\def\PSL{\mathrm{PSL}}
\def\SL{\mathrm{SL}}
\def\GL{\mathrm{GL}}
\def\Sp{\mathrm{Sp}}
\def\P{\mathbb{P}}

\def\ker{\text{ker}}
\def\im{\text{im}}

\DeclareMathOperator{\res}{Res}
\DeclareMathOperator{\spec}{Spec}

\def\sce{\mathscr{C}^{\textsf{ex}}}
\def\scd{\mathscr{C}}
\def\caff{C_{\textsf{aff}}}
\def\sj{\mathscr{J}}
%\def\sA{\ensuremath{\mathscr{A}}}
\def\sO{\mathcal{O}}
%\def\sO{\ensuremath{\mathscr{O}}}
%\def\sL{\ensuremath{\mathscr{L}}}
%\def\sK{\ensuremath{\mathscr{K}}}
%\def\sF{\ensuremath{\mathscr{F}}}
\def\Pro{\ensuremath{\mathbb{P}}}
%\def\sG{\mathscr{G}}
\def\sX{\mathcal{X}}
\newcommand{\sxi}{\mathcal{X}_\infty}
%\DeclareMathOperator{\Supp}{Supp}
%\DeclareMathOperator{\Div}{Div}
%\DeclareMathOperator{\dv}{div}
%\DeclareMathOperator{\Pic}{Pic}
%\DeclareMathOperator{\P0}{Pic^0}
%\DeclareMathOperator{\Spec}{Spec}
\def\isom{\cong}

\begin{document}
\title{Height pairing on our curve}
\author{Shahed Sharif}

\section{Height pairing}
\label{sec:height-pairing}

The purpose of this write-up is to compute the height pairing on various sections of our curve. Specifically, we will compute $\langle P_{i,j} - Q_\infty, P_{i', j'} - Q_\infty \rangle$. By bilinearity, it suffices to compute $\langle P_{i,j} - Q_\infty, P_{0,0} - Q_\infty \rangle$ for all $i,j$. In all of the following, we base-extend to the field $K(u)$ where $u^d = t$.

\subsection{Basic theory}
\label{sec:basic-theory}

Let $P, P', Q$ be three distinct points on our curve; we will later set $P = P_{i,j}$, $P' = P_{0,0}$, and $Q = Q_\infty$. Then
\[
\langle P - Q, P' - Q\rangle = -P \cdot P' + P \cdot Q + P' \cdot Q - Q^2 + D_P \cdot P',
\]
 where $D_P$ is a fibral divisor for which $D_P \cdot Q = 0$, for every fibral divisor $D$, $D_P \cdot D = (Q - P) \cdot D$. Since $D_P$ is fibral, the calculation of $D_P \cdot P'$ will be done at each fiber. On each fiber with only a single component (notably smooth fibers), we see that $D_P$ has empty support; thus we can restrict ourselves to the bad fibers of our arithmetic surface to calculate $D_P \cdot P'$. 

We must also compute $P \cdot P'$, $P \cdot Q$, $P' \cdot Q$, and $Q^2$. The former three can be computed on each bad fiber separately. It will turn out that these intersection numbers are 0 on every fiber except $u = -1$. We will compute $Q^2$ in \S~??? by computing the degree of a certain invertible sheaf.


\subsection{Auxiliary results}
\label{sec:auxiliary-results}

The following results will help us compute the various intersection numbers. Let us suppose we restrict ourselves to a special fiber with components $C_0, C_1, \dots, C_n$.

\begin{proposition}\label{prop:dp-dot-p-cofactor}
  Suppose $Q$ intersects $C_0$, $P$ intersects $C_i$, and $P'$ intersects $C_j$, with $i,j \neq 0$. Let $A$ be the intersection matrix for the fiber, and let $B$ be the matrix obtained by deleting the first row and column from $A$. Let $\beta_{ij}$ be the $ij$ cofactor for $B$. Let $\delta$ be the multiplicity of $C_0$ in the special fiber. Finally, let $\phi$ be the order of the component group of the Jacobian of $C$. Then
  \[
  D_P \cdot P' = (-1)^{n+1} \frac{\beta_{ij}}{\delta^2 \phi}.
  \]
\end{proposition}

\begin{proof}
  Recall that we choose $D_P = \sum d_h C_h$ so that $d_0 = 0$ and $D_P \cdot C_h = (Q - P) \cdot C_h$ for every $h > 1$. If we let $\vd = (d_1, \dots, d_n)$, then this is equivalent to saying
  \[
  B\vd = -\ve_i
  \]
  where $\ve_i$ is the $i$th standard basis vector. The intersection form $A$ has the property that every $n \times n$ submatrix is nonsingular, so we have
  \[
  \vd = -B^{-1}\ve_i.
  \]
  The intersection number $P' \cdot D_P$ is simply the coefficient $d_j$, and
  \[
  D_P \cdot P' = d_j = \ve_j^T \vd = -\ve_j^TB^{-1}\ve_i.
  \]
  By symmetry of $A$, and hence of $B$, and the cofactor formula for the inverse of a matrix, we have
  \[
  D_P \cdot P' = - \frac{\beta_{ij}}{\det B}.
  \]
  But according to Corollary~1.3 in~\cite{lorenzini},
  \[
  \det (-B) = \delta^2 \phi.
  \]
  The claim now follows.
\end{proof}

\begin{proposition}\label{prop:am-defn-det}
  Let $A_m$ be the $m \times m$ matrix whose entries are given by
  \[
  a_{ij} = \begin{cases}
    -2 & \text{if } i = j \\
    1 & \text{if } |i - j| = 1\\
    0 & \text{otherwise}
  \end{cases}.
  \]
  Then $\det A_m = (-1)^m (m+1)$.
\end{proposition}

\begin{proof}
  Consider an elliptic curve over a discrete valuation ring with type $I_{m+1}$ reduction. Let $A$ be the intersection matrix for the special fiber, where the components are ordered in the usual manner. One observes that $A_m$ is the $m \times m$ submatrix obtained from $A$ by deleting the first row and column. By \cite[Corollary~1.3]{lorenzini}, $\det A_m = (-1)^m \phi$, where $\phi$ is the order of the component group of our elliptic curve. But as our curve has type $I_{m+1}$ reduction, the component group is $\Z/(m+1)\Z$.
\end{proof}

\begin{proposition}\label{prop:fg-equals-u-blow-up}
  Let $R$ be a discrete valuation ring with maximal ideal $(u)$. Suppose $C$ is given locally by $fg = u^d$, where $d \geq 3$. Let $P$ be a point on $C$ such that $f(P), g(P) \in (u)$. Let $C_1$ and $C_2$ be the components of the special fiber of $C$ given by $f = u = 0$ and $g = u = 0$ respectively, and suppose these components cross transversally. Let the minimal proper regular model of $C$ have special fiber given by the chain of rational curves $C_1, E_1, \dots, E_{d-1}, C_2$.
  \begin{enumerate}
      \item  If
  \[
  \frac{g(P)}{u}, \frac{g(P)}{f(P)}, f(P) \in (u)
  \]
  then $P$ lies on $E_{d-1}$.
    \item If $P$ lies on $E_{d-1}$ and $\frac{f(P)}{u}$ is a unit in $R$, then $P$ has smooth reduction.
    \item $\frac{f}{u}$ is a degree 1 parameter for $E_{d-1}$.
  \end{enumerate}
\end{proposition}

\begin{proof}
  After doing a single blow-up, one has the chain of rational curves $C_1, E_1, E_{d-1}, C_2$. The result follows from checking each chart.
\end{proof}

\subsection{Pairing at $u = 0$}
\label{sec:pairing-at-u}

We compute the pairing at $u = 0$. From Proposition~\ref{prop:dp-dot-p-cofactor}, we must determine which components each of the $P_{i,j}$ and $Q_\infty$ lie on. We first recall that the special fiber consists of rational curves connected by $r$ chains of $\Pro^1$'s, each of length $d-1$. Equivalently, the dual graph is given by
\begin{figure}[h]\centering
  \[
\xygraph{
  !{<0cm,0cm>;<1.5cm,0cm>:<0cm,1.25cm>::}
  !{(0, 2) }*{\bullet}="c0"
  !{(5, 2) }*{\bullet}="clast"
  !{(1, 4) }*{\bullet}="c11"
  !{(2, 4) }*{\bullet}="c12"
  !{(3, 4) }*{\dots}="c1dots"
  !{(4, 4) }*{\bullet}="c1last"
  !{(1, 3) }*{\bullet}="c21"
  !{(2, 3) }*{\bullet}="c22"
  !{(3, 3) }*{\dots}="c2dots"
  !{(4, 3) }*{\bullet}="c2last"
  !{(2.5, 2) }*{\vdots}
  !{(1, 1) }*{\bullet}="cr1"
  !{(2, 1) }*{\bullet}="cr2"
  !{(3, 1) }*{\dots}="crdots"
  !{(4, 1) }*{\bullet}="crlast"
  "c0"-"c11"
  "c11"-"c12"
  "c12"-"c1dots"
  "c1dots"-"c1last"
  "c1last"-"clast"
  "c0"-"c21"
  "c21"-"c22"
  "c22"-"c2dots"
  "c2dots"-"c2last"
  "c2last"-"clast"
  "c0"-"cr1"
  "cr1"-"cr2"
  "cr2"-"crdots"
  "crdots"-"crlast"
  "crlast"-"clast"
}
\]
  \caption{Dual graph of $C_k$, $u=0$}
\label{fig:u-equals-zero}
\end{figure}
There are $r$ paths from the left hand vertex to the right hand vertex, and on each path, there are $d-1$ intermediate vertices, for a total of $r(d-1) + 2$ vertices in the graph. We label the corresponding components of the fiber as follows. Before desingularizing, the special fiber is given by $x(y^r - x - 1) = u^d$. The component $x = u = 0$ corresponds to the left-most vertex, and we call it $C_0$. The component $y^r - x - 1 = u = 0$ corresponds to the the right-most vertex, which we label $C_{r(d-1)+1}$. Each horizontal path corresponds to a $\lfloor \frac{d}{2}\rfloor$-fold blow-up of one of the nodes; these nodes are given by $x = 0$, $y = \omega^j$ where $\omega$ is the primitive $r$th root of unity given by $\omega = \zeta^{d/r}$. Label each of the horizontal components as $C_{j(d-1) + i}$, where $0 \leq j \leq r-1$, $1 \leq i \leq d-1$, $j$ increases from top to bottom, and $i$ increases from left to right. Thus if the top horizontal path corresponds to the blow-up of the node $x = 0, y = 1$, then the top left component would be $C_1$, and $C_2$ would be immediately to the right of it.

\begin{proposition}
  The section $Q_\infty$ lies on $C_0$. The section $P_{ij}$ lies on $C_{(j+1)(d-1)}$.
\end{proposition}

Note that $C_{(j+1)(d-1)} = C_{j(d-1) + d-1}$ is the right-most vertex on the $j$th horizontal path.

\begin{proof}
  Let $\sX_0$ be the affine curve given by $x_0 y_0^r = (x_0+1)(x_0+u^d)$; it is an affine model for $C$. The special fiber, as noted above, consists of two rational curves meeting at $r$ points. One sees that $Q_\infty$ does not lie above any of these nodes. Therefore the specialization of $Q_\infty$ to the fiber over $u = 0$ lies on the smooth locus of one of the components $C_0$ or $C_{r(d-1)+1}$. We observe that $\sX_0$ glues to $\sX_2$ via
  \[
  (x_0, y_0) = (w, w/z).
  \]
  Since $Q_\infty$ is given by $(w, z) = (0, 0)$, one sees that $x_0(Q_\infty) = 0$. Therefore $Q_\infty$ lies on $C_0$.

  Now consider $P_{ij}$. The chart $\sX_0$ glues to $C$ via $(x_0, y_0) = (x, y/x)$. Therefore
  \[
  (x_0(P_{ij}), y_0(P_{ij})) = (\zeta^i u, \omega^j(\zeta^iu + 1)^{d/r}).
  \]
  The specialization of $P_{ij}$ lies on the node with coordinates $(x_0, y_0, u) = (0, \omega^j, 0)$. Thus we need to blow up $\sX_0$ to determine which component $P_{ij}$ lies on. We at least know that $P_{ij}$ must lie on $C_0$, $C_{r(d-1)+1}$ or $C_{j(d-1) + k}$ for some $k$. Let $f = x$ and $g = y^r - x - 1$. We have
  \[
  f(P_{ij}) = \zeta^i u \quad g(P_{ij}) = u^d + \zeta^{i(d-1)} u^{d-1}.
  \]
  One checks that the hypotheses of Proposition~\ref{prop:fg-equals-u-blow-up} hold, whence $P_{ij}$ lies on $C_{(j+1)(d-1)}$.
\end{proof}

\begin{proposition}\label{prop:local-intersections-u-0}
  The intersection numbers $P_{00} \cdot P_{ij}$ for $ij \neq 00$ and $Q_{\infty} \cdot P_{ij}$ at $u = 0$ are all zero.
\end{proposition}

\begin{proof}
  In the pairing $Q_\infty \cdot P_{ij}$, and in $P_{00} \cdot P_{ij}$ when $j \neq 0$, the points do not even lie on the same component of the special fiber. Suppose in the latter pairing that $j = 0$, so that both points reduce to $C_{d-1}$. By Proposition~\ref{prop:fg-equals-u-blow-up}, $\frac{x_0}{u}$ is a coordinate function for $C_{d-1}$. But
  \[
  \frac{x_0(P_{00})}{u} = 1 = \zeta^i = \frac{x_0(P_{i0})}{u}
  \]
  if and only if $i = 0$.
\end{proof}

Recall that the matrix $B$ constructed in \S~\ref{sec:auxiliary-results} is obtained by deleting the first row and column from the intersection matrix for the special fiber. Using the ordering given above for the components, we get
\[
B =
\left[\begin{array}{ccccc}
  A_{d-1} & & & & e_{d-1} \\
  & A_{d-1} & & & e_{d-1} \\
  & & \ddots & & \vdots \\
  & & & A_{d-1} & e_{d-1} \\
  e_{d-1}^T & e_{d-1}^T & \cdots & e_{d-1}^T & -r
\end{array}\right]
\]
The above gives $B$ in block form. The $A_{d-1}$ are as given in Proposition~\ref{prop:am-defn-det}, and $e_{d-1} \in \R^{d-1}$ is the $(d-1)$st standard basis vector, written as a column (so $e_{d-1}^T$ is a row vector). There are $r$ copies of $A_{d-1}$ in the matrix, so that $B$ has size $r(d-1) + 1$.

Let
\[
M_{\ell} =
\left[\begin{array}{ccccc}
  1 & 2 & 3 & \cdots & \ell \\
  2 & 4 & 6 & \cdots & 2\ell \\
  \vdots &\vdots & \vdots & & \vdots \\
  \ell & 2\ell & 3\ell & \cdots & \ell^2 
\end{array}\right]
\]

and let $N_{s,t}$ be the $t \times t$ symmetric matrix given by
\[
N_{ij} = \begin{cases}
    i(t+(t-j+1)(s-1)) & \text{if } i \geq j \\
    N_{ji} & \text{if } i < j
  \end{cases},
\]
for $1 \leq i,j \leq t$.


\begin{proposition}
Let $v_{\ell} = (1,2,\ldots, \ell)^T$. Then the matrix
\[
X = \frac{d^{r-2}}{\det(-B)} \cdot
\left[\begin{array}{cccccc}
  N_{r,d-1} & M_{d-1} & M_{d-1} & \cdots & M_{d-1} & dv_{d-1} \\
  M_{d-1} & N_{r,d-1} & M_{d-1} & \cdots & M_{d-1} & dv_{d-1} \\
  \vdots &\vdots & \vdots & & & \vdots \\
  M_{d-1} & M_{d-1} & M_{d-1} & \cdots &N_{r,d-1} & dv_{d-1}\\
  dv_{d-1}^T & dv_{d-1}^T & dv_{d-1}^T & \cdots & dv_{d-1}^T &d^2 \\
   
\end{array}\right]
\]
is the inverse of the matrix $B$ defined above.
\end{proposition}
\begin{proof}
This is just algebra that can be checked by hand.
\end{proof}

\begin{proposition}
The $(d-1, (j+1)(d-1))$th entry of $B^{-1}$ is given by the equation
\[
(B^{-1})_{d-1,(j+1)(d-1)} = \frac{d^{r-2}}{\det(-B)} \cdot \begin{cases}
    d^2r-2dr-d+r+1 & \text{if } j=0 \\
    (d-1)^2 & \text{otherwise }
  \end{cases}.
\]
\end{proposition}

\subsection{Pairing at $u=\infty$}
\label{sec:pairing-at-u=infty}

Recall that the fiber at $u=\infty$ is birational to that at $u=0$. Therefore the special fibers and intersection matrices are identical. We must now determine which component our sections lie on.
\begin{proposition}
  On the fiber at $u=\infty$, $Q_\infty$ lies on $C_0$ and $P_{ij}$ lies on $C_{(i+j)(d-1) +1}$, where $i+j$ is taken modulo $r$.
\end{proposition}

\begin{proof}
  Let $\sxi$ be the affine curve
  \[
  x_\infty y_\infty^r = (x_\infty + 1)(x_\infty + {U^d})
  \]
  where $U = \frac{1}{u}$. The chart $\sxi$ glues to $\sX_0$ and $\sX_2$ via 
  \[
  (x_\infty, y_\infty) = (U^d x_0, U^{d/r} y_0) = \bigg(U^d w, U^{d/r} \frac{w}{z}\bigg).
  \]
  We are concerned with the fiber when $U = 0$. Let $C_0$ be the component given by $x_\infty = U = 0$, and $C_{r(d-1)+1}$ the component given by $y_\infty^r - x_\infty - 1 = U = 0$; label the remaining components of the special fiber of the minimal proper regular model similarly as in \S~\ref{sec:pairing-at-u}. One sees immediately that on the special fiber, $Q_\infty$ lies on the component $C_0$.

  Next, observe that $\sxi$ glues to $C$ via
  \[
  (x_\infty, y_\infty) = (U^d x, U^{d/r} y/x).
  \]
  Thus
  \[
  (x_\infty(P_{ij}), y_\infty(P_{ij})) = (\zeta^i U^{d-1}, \omega^j (\zeta^i + U)^{d/r}).
  \]
  Modulo $U$, this gives $(0,\omega^{i+j})$; that is, $P_{ij}$ lies on the corresponding node of the special fiber. Let $f = y_\infty^r - x_\infty - 1$ and $g = x_\infty$. We see that 
  \[
  f(P_{ij}) = \zeta^{i(d-1)} U + U^d \quad g(P_{ij}) = \zeta^i U^{d-1}.
  \]
  One now verifies the hypotheses of Proposition~\ref{prop:fg-equals-u-blow-up} to conclude that $P_{ij}$ lies on $C_{(i+j)(d-1) +1}$, where in the subscript $i+j$ is taken modulo $r$.
\end{proof}

\begin{proposition}\label{prop:local-intersections-u-infty}
  The intersection numbers $P_{00} \cdot P_{ij}$ for $ij \neq 00$ and $Q_{\infty} \cdot P_{ij}$ at $u = \infty$ are all zero.
\end{proposition}

\begin{proof}
  The reasoning is similar to that of Proposition~\ref{prop:local-intersections-u-0}. The only nontrivial case occurs when $i+j \equiv 0 \pmod{r}$, in which case both $P_{00}$ and $P_{ij}$ lie on $C_1$. By Proposition~\ref{prop:fg-equals-u-blow-up}, $\dfrac{y_\infty^r - x_\infty - 1}{U}$ is a coordinate function for $C_1$. Evaluating this function at $P_{00}$ and $P_{ij}$ gives the values $1$ and $\zeta^{i(d-1)}$ respectively. These are equal if and only if $i = 0$, from which $j = 0$ as well.
\end{proof}

\subsection{Pairing at $u = \zeta^k$}
\label{sec:pairing-at-u-1}

Fix $0 \leq k \leq d-1$. Since the extension $K(u)$ is unramified over $t = 1$, the special fiber at $u = \zeta^k$ is isomorphic to that of $C/K$ when $t = 1$. Recall the initial chart 
\[
\caff: (x_0 - 1) \bar{y}^r = x_0(x_0 + (u^d - 1)),
\]
where $\caff$ glues to $\sX_1$ via $x_0 = x + 1$ and $\bar{y} = y/x$. Let $u_k = u - \zeta^k$ and $\alpha_k = \frac{u^d - 1}{u_k}$. Then we can rewrite this model in a neighborhood of $u_k = 0$ as
\[
(x_0 - 1) \bar{y}^r = x_0(x_0 + \alpha_ku_k).
\]
The special fiber of $\caff$ has a single component, $F$, with multiplicity 1 and a single node at $(x_0, \bar{y}, u_k) = (0,0,0)$.
\begin{proposition}\label{prop:points-on-components-u-zeta-k}
  At $u=\zeta^k$, $Q_\infty$ lies on $F$. If $i+k \not\equiv \frac{d}{2} \pmod{d}$, then $P_{ij}$ lies on $F$. If $i + k \equiv \frac{d}{2} \pmod{d}$, then $P_{ij}$ lies on $E_1$.
\end{proposition}

\begin{proof}
  We immediately see that $Q_\infty$ does not reduce to the node of $F$, so $Q_\infty$ lies on the smooth locus of (the proper curve with affine piece) $F$. For $P_{ij}$, we observe that
  \[
  (x_0, \bar{y})(P_{ij}) = (\zeta^i u + 1, \omega^j (\zeta^i u + 1)^{d/r}).
  \]
  The reduction of this point has coordinates
  \begin{equation}
    (\zeta^{i+k} + 1, \omega^j(\zeta^{i+k} + 1)^{d/r})\label{eq:pij-on-u-equals-zeta-k}
  \end{equation}
    which lies on the node if and only if $\zeta^{i+k} = -1$. This occurs precisely when $i + k \equiv \frac{d}{2} \pmod{d}$.

  Now suppose that $\zeta^{i+k} = -1$. We consider the chart $(0,r)_b$ from section \S~???. This latter chart was given by coordinates $x_0, b_0, \tau_0$, where $t_0 = x_0 \tau_0$. In our case, $x_0$ and $b_0$ are the same, but our third coordinate, which we call $\mu_k$, is given by $u_k = x_0 \mu_k$, yielding the chart
  \[
  (x_0 - 1)x_0^{r-2}b_0^r = 1 + \alpha_k \mu_k.
  \]
  The special fiber has components $F$ given by $\mu_k = (x_0 - 1)x_0^{r-2}b_0^r - 1 = 0$ and $E_1$ given by $x_0 = 1 + \alpha_k \mu_k = 0$. Furthermore, this piece of the special fiber is smooth. We have
  \[
  (x_0, b_0, \mu_k)(P) = (\zeta^iu + 1, \omega^j(\zeta^i u + 1)^{\frac{d}{r} - 1}, -\zeta^k).
  \]
  If $d = r$, on the special fiber we obtain the point $(0, \omega^j, -\zeta^k)$. If $d > r$, we obtain $(0, 0, -\zeta^k)$. In both cases, $P_{ij}$ lies on the smooth locus of $E_1$.
\end{proof}

The intersection forms $A$ in these cases are given in \S~?? and \S~??. In all cases, the component $F$ corresponds to the last row and last column of $A$. As the matrix $B$ of Proposition~\ref{prop:dp-dot-p-cofactor} is obtained by removing this row and column, one observes that $B = A_{r-1}$, where $A_{r-1}$ is defined in Proposition~\ref{prop:am-defn-det}.

\begin{proposition}\label{prop:local-intersections-u-zeta-k}
  If $k \neq \frac{d}{2}$, then the intersection numbers $P_{00} \cdot P_{ij}$ for $ij \neq 00$ and $Q_{\infty} \cdot P_{ij}$ at $u = \zeta^k$ are all zero.
\end{proposition} 

\begin{proof}
  The only unclear case occurs for $P_{00} \cdot P_{ij}$ when $i + k \not\equiv \frac{d}{2} \pmod{d}$. In this case, \eqref{eq:pij-on-u-equals-zeta-k} gives the coordinates for both points on the special fiber, and one sees that the intersection number is zero.
\end{proof}

\begin{proposition}\label{prop:local-intersections-u-minus-1}
  If $k = \frac{d}{2}$, the intersection numbers $P_{00} \cdot P_{ij}$ for $i \neq 0$ and $Q_{\infty} \cdot P_{ij}$ at $u = -1$ are all zero. The intersection numbers $P_{00} \cdot P_{0j}$ for $j > 0$ are $\frac{d}{r} - 1$.
\end{proposition}

\begin{proof}
  The case with $Q_\infty$ is clear. The hypothesis $u = -1$ is identical to $k = \frac{d}{2}$, so that by Proposition~\ref{prop:points-on-components-u-zeta-k}, $P_{ij}$ lies on the component $F$ if and only if $i = 0$. The case $i \neq 0$ follows. We now suppose that $i = 0$ and $j > 0$. Let $R$ be the localization of
  \[
  \frac{K[x_0,b_0,\mu]}{(x_0 - 1)x_0^{r-2}b_0^r - 1 - \alpha \mu}
  \]
  at the ideal $(x_0, b_0, \mu - 1)$. Note that this is the adjusted chart $(0,r)_b$ mentioned in the proof of Proposition~\ref{prop:points-on-components-u-zeta-k} via $k = \frac{d}{2}$, $\alpha = \alpha_k$, and $\mu = \mu_k = \frac{u+1}{x_0}$. A local function for $P_{0j}$ in a neighborhood of the closed point of $R$ is given by
  \[
  f_j = b_0 - \omega^j x_0^{\frac{d}{r} - 1}.
  \]
  Then the intersection multiplicity $P_{00} \cdot P_{0j}$ is the length of 
  \[
  \frac{R}{(f_0, f_j)}.
  \]
  One checks that the latter ring is isomorphic to
  \[
  \frac{R}{\bigg(x_0^{\frac{d}{r} - 1}, b_0, \mu - 1\bigg)}
  \]
  from which the claim follows.
\end{proof}
\end{document}





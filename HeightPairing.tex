%\documentclass[reqno]{amsart}
\documentclass[pagesize,paper=letter]{scrartcl}
%\usepackage[english]{babel}
\usepackage[T1]{fontenc}
\usepackage{lmodern}
\usepackage{microtype}
\usepackage[all,cmtip]{xy}
\usepackage{amssymb,amsmath,mathrsfs,amsthm,amscd,hyperref,amsfonts,graphicx}
\bibliographystyle{plain}
\oddsidemargin 0mm
\evensidemargin 0mm
\textheight 214mm \textwidth 160mm
\headsep 10mm
\raggedbottom



\newtheorem{thm}{Theorem}[section]
\newtheorem{theorem}{Theorem}[section]
%\newtheorem{defn}[thm]{Definition}
\newtheorem{cor}[thm]{Corollary}
\newtheorem{conj}[thm]{Conjecture}
\newtheorem{lem}[thm]{Lemma}
\newtheorem{lemma}[thm]{Lemma}
\newtheorem{prop}[thm]{Proposition}
\newtheorem{proposition}[thm]{Proposition}
\newtheorem{nota}[thm]{Notation}
%\newtheorem{rem}[thm]{Remark}

\theoremstyle{definition}
\newtheorem{rem}[thm]{Remark}
\newtheorem{remark}[thm]{Remark}
\newtheorem{defn}[thm]{Definition}
\renewcommand{\thedefn}{}

\theoremstyle{remark}
\newtheorem{ex}{Example}
\newtheorem{notation}{Notation}
\newtheorem{exs}{Examples}
\renewcommand{\theex}{}
\renewcommand{\thenotation}{}
\renewcommand{\theexs}{}



%\theoremstyle{plain}
%\newtheorem{thm}[equation]{Theorem}
%\newtheorem{prop}[equation]{Proposition}
%\newtheorem{cor}[equation]{Corollary}
%\newtheorem{lemma}[equation]{Lemma}
%\newtheorem{conj}[equation]{Conjecture}

%\theoremstyle{definition}
%\newtheorem{defn}[equation]{Definition}
%\newtheorem{defns}[equation]{Definitions}
%
%\theoremstyle{remark}
%\newtheorem{rem}[equation]{Remark}
%\newtheorem{rems}[equation]{Remarks}
%\newtheorem{exer}[equation]{Exercise}
%\newtheorem{exers}[equation]{Exercises}
%\newtheorem{rem-exer}[equation]{Remark/Exercise}
%\newtheorem{rem-exers}[equation]{Remark/Exercises}
%\newtheorem{ex}[equation]{Example}
%\newtheorem{exs}[equation]{Examples}

\DeclareMathOperator{\dvsr}{div}
\newcommand{\one}{{\bf 1}}
\newcommand{\Div}{\operatorname{Div}}
\newcommand{\divi}{\operatorname{div}}
\newcommand{\sep}{\operatorname{sep}}
\newcommand{\XminusT}{(x-T)}
\newcommand{\xminusT}{(x-T)'}
\newcommand{\GG}{\mathcal{G}}
\newcommand{\iso}{\stackrel{\sim}{\rightarrow}}
\newcommand{\val}{\operatorname{val}}
\newcommand{\pr}{\operatorname{pr}}
\newcommand{\rk}{\operatorname{rank}}
\newcommand{\sy}{\bar{y}}
\newcommand{\vd}{\mathbf{d}}
\newcommand{\ve}{\mathbf{e}}


\def\bmu{{\boldsymbol\mu}}

\def \PP {{\mathbb P}^1}
\def \ZZ {{\mathbb Z}}
\def  \FF {{\mathbb F}}
\def \car {\mathop {\rm Car}}

\def\genus{\textrm{genus}}
\def\cond{\textrm{cond}}
\def\R{\mathbb{R}}
\def\Gal{\textrm{\upshape Gal}}
\def\ord{\textrm{\upshape ord}}
\def\p{\mathbb{P}}
\def\C{\mathbb{C}}
\def\Q{\mathbb{Q}}
%\def\A{\mathbf{A}}
\def\QQ{\overline{\mathbb{Q}}}
\def\Z{\mathbb{Z}}
\def\m{\mathfrak{m}}
\def\Oo{\mathcal{O}}
\def\F{\mathbb{F}}
\def\Hom{\mathrm{Hom}}
\def\tor{\mathrm{tor}}
\def \End{\mathrm{End}}
\def\hh{\textrm{H}}
\def\s{\textrm{\upshape Spec}\,}
\def\Br{\textnormal{Br}}
\def\pr{\textnormal{pr}}
\def\Nm{\textnormal{Nm}}
\def\can{\textnormal{can}}
\def\rank{\textnormal{rank}}
\def\Pic{\textnormal{Pic}}
\def\h{\textnormal{H}}
%\def\PGL{\textnormal{PGL}}
\def\torsion{\textnormal{tors}}
\def\tors{\textnormal{tors}}


% provide the letter sha:
\usepackage[OT2,T1]{fontenc}
\DeclareSymbolFont{cyrletters}{OT2}{wncyr}{m}{n}
\DeclareMathSymbol{\sha}{\mathalpha}{cyrletters}{"58}

\def\JJ{\mathcal{J}}
\def\XX{\mathcal{X}}
\def\YY{\mathcal{Y}}
\def\O{\mathcal{O}}
\def\OO{\mathcal{O}}



% abbreviations/alternate names
\def\<{\langle}
\def\>{\rangle}
\def\into{\hookrightarrow}
\def\onto{\twoheadrightarrow}
\def\isoto{\tilde{\to}}
\def\tensor{\otimes}
\def\compose{\circ}
\def\sdp{{\rtimes}}
\def\nodiv{\not|}
\def\PGL{\mathrm{PGL}}
\def\PSL{\mathrm{PSL}}
\def\SL{\mathrm{SL}}
\def\GL{\mathrm{GL}}
\def\Sp{\mathrm{Sp}}
\def\P{\mathbb{P}}

\def\ker{\text{ker}}
\def\im{\text{im}}

\DeclareMathOperator{\res}{Res}
\DeclareMathOperator{\spec}{Spec}

\def\sce{\mathscr{C}^{\textsf{ex}}}
\def\scd{\mathscr{C}}
\def\caff{C_{\textsf{aff}}}
\def\sj{\mathscr{J}}
%\def\sA{\ensuremath{\mathscr{A}}}
\def\sO{\mathcal{O}}
%\def\sO{\ensuremath{\mathscr{O}}}
%\def\sL{\ensuremath{\mathscr{L}}}
%\def\sK{\ensuremath{\mathscr{K}}}
%\def\sF{\ensuremath{\mathscr{F}}}
\def\Pro{\ensuremath{\mathbb{P}}}
%\def\sG{\mathscr{G}}
%\def\sX{\mathscr{X}}
%\DeclareMathOperator{\Supp}{Supp}
%\DeclareMathOperator{\Div}{Div}
%\DeclareMathOperator{\dv}{div}
%\DeclareMathOperator{\Pic}{Pic}
%\DeclareMathOperator{\P0}{Pic^0}
%\DeclareMathOperator{\Spec}{Spec}
\def\isom{\cong}

\begin{document}
\title{Height pairing on our curve}
\author{Shahed Sharif}

\section{Height pairing}
\label{sec:height-pairing}

The purpose of this write-up is to compute the height pairing on various sections of our curve. Specifically, we will compute $\langle P_{i,j} - Q_\infty, P_{i', j'} - Q_\infty \rangle$. By bilinearity, it suffices to compute $\langle P_{i,j} - Q_\infty, P_{0,0} - Q_\infty \rangle$ for all $i,j$. In all of the following, we base-extend to the field $K(u)$ where $u^d = t$.

\subsection{Basic theory}
\label{sec:basic-theory}

Let $P, P', Q$ be three points on our curve; we will later set $P = P_{i,j}$, $P' = P_{0,0}$, and $Q = Q_\infty$. Then the we have
\[
\langle P - Q, P' - Q\rangle = -P \cdot P' + P \cdot Q + P' \cdot Q - Q^2 + D_P \cdot P',
\]
 where $D_P$ is a divisor computed as follows ... Since $D_P$ is a fibral divisor, the calculation of $D_P \cdot P'$ will be done at each fiber. One sees that we can restrict ourselves to the bad fibers of our arithmetic surface.

\subsection{Auxiliary results}
\label{sec:auxiliary-results}

The following results will help us compute $D_P \cdot P'$. Let us suppose we restrict ourselves to a special fiber with components $C_0, C_1, \dots, C_n$.

\begin{proposition}\label{prop:dp-dot-p-cofactor}
  Suppose $Q$ intersects $C_0$, $P$ intersects $C_i$, and $P'$ intersects $C_j$, with $i,j \neq 0$. Let $A$ be the intersection matrix for the fiber, and let $B$ be the matrix obtained by deleting the first row and column from $A$. Let $\beta_{ij}$ be the $ij$ cofactor for $B$. Let $r$ be the multiplicity of $C_0$ in the special fiber. Finally, let $\phi$ be the order of the component group of the Jacobian of $C$. Then
  \[
  D_P \cdot P' = (-1)^{n+1} \frac{\beta_{ij}}{r^2 \phi}.
  \]
\end{proposition}

\begin{proof}
  Recall that we choose $D_P = \sum d_h C_h$ so that $d_0 = 0$ and $D_P \cdot C_h = (Q - P) \cdot C_h$ for every $h > 1$. If we let $\vd = (d_1, \dots, d_n)$, then this is equivalent to saying
  \[
  B\vd = -\ve_i
  \]
  where $\ve_i$ is the $i$th standard basis vector. The intersection form $A$ has the property that every $n \times n$ submatrix is nonsingular, so we have
  \[
  \vd = -B^{-1}\ve_i.
  \]
  The intersection number $P' \cdot D_P$ is simply the coefficient $d_j$, and
  \[
  D_P \cdot P' = d_j = \ve_j^T \vd = -\ve_j^TB^{-1}\ve_i.
  \]
  By symmetry of $A$, and hence of $B$, and the cofactor formula for the inverse of a matrix, we have
  \[
  D_P \cdot P' = - \frac{\beta_{ij}}{\det B}.
  \]
  But according to Corollary~1.3 in~cite Lorenzini,
  \[
  \det (-B) = r^2 \phi.
  \]
  The claim now follows.
\end{proof}

\begin{proposition}\label{prop:am-defn-det}
  Let $A_m$ be the $m \times m$ matrix whose entries are given by
  \[
  a_{ij} = \begin{cases}
    -2 & \text{if } i = j \\
    1 & \text{if } |i - j| = 1\\
    0 & \text{otherwise}
  \end{cases}.
  \]
  Then $\det A_m = (-1)^m (m+1)$.
\end{proposition}

\begin{proof}
  Consider an elliptic curve over a discrete valuation ring with type $I_{m+1}$ reduction. Let $A$ be the intersection matrix for the special fiber, where the components are ordered in the usual manner. One observes that $A_m$ is the $m \times m$ submatrix obtained from $A$ by deleting the first row and column. By Lorenzini Corollary~1.3, $\det A_m = (-1)^m \phi$, where $\phi$ is the order of the component group of our elliptic curve. But as our curve has type $I_{m+1}$ reduction, the component group is $\Z/(m+1)\Z$.
\end{proof}

\begin{proposition}
  Let $R$ be a discrete valuation ring with maximal ideal $(u)$. Suppose $C$ is given locally by $fg = u^d$, where $d \geq 3$. Let $P$ be a point on $C$ such that $f(P), g(P) \in (u)$. Let $C_1$ and $C_2$ be the components of the special fiber of $C$ given by $f = u = 0$ and $g = u = 0$ respectively. Let the minimal proper regular model of $C$ have special fiber given by the chain of rational curves $C_1, E_1, \dots, E_{d-1}, C_2$. If
  \[
  \frac{g(P)}{u}, \frac{g(P)}{f(P)}, f(P) \in (u)
  \]
  then $P$ lies on $E_{d-1}$. If furthermore $\frac{f(P)}{u}$ is a unit in $R$, then $P$ has smooth reduction.
\end{proposition}

\begin{proof}
  After doing a single blow-up, one has the chain of rational curves $C_1, E_1, E_{d-1}, C_2$. The result follows from checking each chart.
\end{proof}

\subsection{Pairing at $u = 0$}
\label{sec:pairing-at-u}

We compute the pairing at $u = 0$. From Proposition~\ref{prop:dp-dot-p-cofactor}, we must determine which components each of the $P_{i,j}$ and $Q_\infty$ lie on. We first recall that the special fiber consists of rational curves connected by $r$ chains of $\Pro^1$'s, each of length $d-1$. Equivalently, the dual graph is given by
\begin{figure}[h]\centering
  \[
\xygraph{
  !{<0cm,0cm>;<1.5cm,0cm>:<0cm,1.25cm>::}
  !{(0, 2) }*{\bullet}="c0"
  !{(5, 2) }*{\bullet}="clast"
  !{(1, 4) }*{\bullet}="c11"
  !{(2, 4) }*{\bullet}="c12"
  !{(3, 4) }*{\dots}="c1dots"
  !{(4, 4) }*{\bullet}="c1last"
  !{(1, 3) }*{\bullet}="c21"
  !{(2, 3) }*{\bullet}="c22"
  !{(3, 3) }*{\dots}="c2dots"
  !{(4, 3) }*{\bullet}="c2last"
  !{(2.5, 2) }*{\vdots}
  !{(1, 1) }*{\bullet}="cr1"
  !{(2, 1) }*{\bullet}="cr2"
  !{(3, 1) }*{\dots}="crdots"
  !{(4, 1) }*{\bullet}="crlast"
  "c0"-"c11"
  "c11"-"c12"
  "c12"-"c1dots"
  "c1dots"-"c1last"
  "c1last"-"clast"
  "c0"-"c21"
  "c21"-"c22"
  "c22"-"c2dots"
  "c2dots"-"c2last"
  "c2last"-"clast"
  "c0"-"cr1"
  "cr1"-"cr2"
  "cr2"-"crdots"
  "crdots"-"crlast"
  "crlast"-"clast"
}
\]
  \caption{Dual graph of $C_k$, $u=0$}
\label{fig:u-equals-zero}
\end{figure}
There are $r$ paths from the left hand vertex to the right hand vertex, and on each path, there are $d-1$ intermediate vertices, for a total of $r(d-1) + 2$ vertices in the graph. We lable the corresponding components of the fiber as follows. Before desingularizing, the special fiber is given by $x(y^r - x - 1) = u^d$. The component $x = u = 0$ corresponds to the left-most vertex, and we call it $C_0$. The component $y^r - x - 1 = u = 0$ corresponds to the the right-most vertex, which we label $C_{r(d-1)+1}$. Each horizontal path corresponds to a $\lfloor \frac{d}{2}\rfloor$-fold blow-up of one of the nodes; these nodes are given by $x = 0$, $y = \omega^j$ where $\omega$ is some fixed primitive $r$th root of unity. Label each of the horizontal components as $C_{j(d-1) + i}$, where $1 \leq i \leq d-1$ and $i$ increases from left to right. Thus if the top horizontal path corresponds to the blow-up of the node $x = 0, y = 1$, then the top left component would be $C_1$, and $C_2$ would be immediately to the right of it.

\begin{proposition}
  The section $Q_\infty$ lies on $C_0$. The section $P_{ij}$ lies on $C_{(j+1)(d-1)}$.
\end{proposition}

Note that $C_{(j+1)(d-1)}$ is the right-most vertex on the $j$th horizontal path.

\begin{proof}
  Forthcoming.
\end{proof}

Recall that the matrix $B$ constructed in \S~\ref{sec:auxiliary-results} is obtained by deleting the first row and column from the intersection matrix for the special fiber. Using the ordering given above for the components, we get
\[
B =
\left[\begin{array}{ccccc}
  A_{d-1} & & & & e_{d-1} \\
  & A_{d-1} & & & e_{d-1} \\
  & & \ddots & & \vdots \\
  & & & A_{d-1} & e_{d-1} \\
  e_{d-1}^T & e_{d-1}^T & \cdots & e_{d-1}^T & -r
\end{array}\right]
\]
The above gives $B$ in block form. The $A_{d-1}$ are as given in Proposition~\ref{prop:am-defn-det}, and $e_{d-1} \in \R^{d-1}$ is the $(d-1)$st standard basis vector, written as a column (so $e_{d-1}^T$ is a row vector). There are $r$ copies of $A_{d-1}$ in the matrix, so that $B$ has size $r(d-1) + 1$.
\end{document}

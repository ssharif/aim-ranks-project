\documentclass{article}
\usepackage{amsmath, amsthm, amssymb}
\usepackage[all]{xy}
\usepackage[pagebackref,colorlinks]{hyperref}
\usepackage{mathrsfs}

% Color comments!
\usepackage[usenames,dvipsnames]{color}
% Color comments

\newcommand{\jenn}[1]{{\color{magenta} \sf $\clubsuit\clubsuit\clubsuit$ Jenn: [#1]}}
\newcommand{\shahed}[1]{{\color{Purple} \sf $\clubsuit\clubsuit\clubsuit$ Shahed: [#1]}}

\newcommand{\sce}{\mathscr{C}^{\text{ex}}}
\newcommand{\scd}{\mathscr{C}}


\theoremstyle{plain}
\newtheorem*{reftheorem}{Theorem}
\newtheorem{theorem}{Theorem}[section]
\newtheorem{corollary}[theorem]{Corollary}
\newtheorem{proposition}[theorem]{Proposition}
\newtheorem{lemma}[theorem]{Lemma}
\newtheorem{conjecture}[theorem]{Conjecture}
\newtheorem{problem}{Problem}
\newtheorem{question}{Question}
\newtheorem*{question*}{Question}
\newtheorem{claim}{Claim}

\theoremstyle{definition}
\newtheorem{definition}[theorem]{Definition}

\theoremstyle{remark}
\newtheorem{remark}[theorem]{Remark}
\newtheorem{example}[theorem]{Example}

% General
\renewcommand{\emptyset}{\varnothing}
\newcommand{\hra}{\hookrightarrow}
\newcommand{\righthookarrow}{\hookrightarrow}
\newcommand{\isom}{\cong}
\newcommand{\too}{\longrightarrow}
\newcommand{\isomto}{\overset{\sim}{\longrightarrow}}
\newcommand{\nto}[1]{\overset{#1}{\longrightarrow}}
\newcommand{\nsubset}{\not\subset}
\renewcommand{\phi}{\varphi}
\newcommand{\To}{\Rightarrow}
\newcommand{\ilim}{\displaystyle\lim_{\leftarrow}}
\newcommand{\dirlim}{\displaystyle\lim_{\rightarrow}}
\newcommand{\eps}{\varepsilon}
\renewcommand{\bar}[1]{\overline{#1}}
\renewcommand{\tilde}[1]{\widetilde{#1}}
\DeclareMathOperator{\car}{char}
\DeclareMathOperator{\rk}{rk}
\DeclareMathOperator{\coker}{coker}
\DeclareMathOperator{\Hom}{Hom}
\DeclareMathOperator{\Aut}{Aut}
\DeclareMathOperator{\End}{End}
\DeclareMathOperator{\im}{im}
\DeclareMathOperator{\pgl}{PGL}
\DeclareMathOperator{\Gl}{GL}
\DeclareMathOperator{\Sl}{SL}

% Number theory
\newcommand{\Qbar}{\ensuremath{\overline{\Q}}}
\newcommand{\Kb}{\overline{K}}
\newcommand{\Fb}{\overline{F}}
\newcommand{\kb}{\overline{k}}
\newcommand{\Xbar}{\overline{X}}
\newcommand{\Cbar}{\overline{C}}
\newcommand{\R}{\ensuremath{\mathbb{R}}}
\newcommand{\C}{\ensuremath{\mathbb{C}}}
\newcommand{\F}{\ensuremath{\mathbb{F}}}
\newcommand{\fp}{\ensuremath{\mathbb{F}_p}}
\newcommand{\sm}{\ensuremath{\mathfrak{m}}}
\newcommand{\Q}{\ensuremath{\mathbb{Q}}}
\newcommand{\Z}{\ensuremath{\mathbb{Z}}}
\newcommand{\ok}{\mathscr{O}_K}
\DeclareMathOperator{\Gal}{Gal}
\DeclareMathOperator{\inv}{inv}
\DeclareMathOperator{\Nm}{Nm}
\DeclareMathOperator{\tr}{Tr}

% Algebraic geometry
\newcommand{\sA}{\ensuremath{\mathscr{A}}}
\newcommand{\sO}{\ensuremath{\mathscr{O}}}
\newcommand{\sL}{\ensuremath{\mathscr{L}}}
\newcommand{\sK}{\ensuremath{\mathscr{K}}}
\newcommand{\sF}{\ensuremath{\mathscr{F}}}
\newcommand{\A}{\ensuremath{\mathbb{A}}}
\newcommand{\Pro}{\ensuremath{\mathbb{P}}}
\newcommand{\G}{\ensuremath{\mathbb{G}}}
\newcommand{\sG}{\mathscr{G}}
\newcommand{\sX}{\mathscr{X}}
\DeclareMathOperator{\Supp}{Supp}
\DeclareMathOperator{\Div}{Div}
\DeclareMathOperator{\dv}{div}
\DeclareMathOperator{\Pic}{Pic}
\DeclareMathOperator{\P0}{Pic^0}
\DeclareMathOperator{\Spec}{Spec}

\begin{document}

\title{Regular model at infinity}
\author{Jenn Park \and Shahed Sharif}
\maketitle

\jenn{testing colour comments}
\shahed{testing colour comments}

Let $k$ be the finite field $\F_q$ and $K = k(t)$. Let $r \geq 3$ be an integer. Let $C$ be the smooth projective curve with affine model
\[
C: xy^r = (x+1)(x+t).
\]
The purpose of this note is to compute the minimal proper regular model of $C$ at $t = \infty$, as well as the component group.

\section{Desingularization}
\label{sec:desingularization}

Let $T = \frac{1}{t}$, so that our curve has affine piece given by
\[
C_{aff}:Txy^r = (x+1)(Tx+1).
\]
To desingularize the generic fiber, consider the affine curves
\begin{align*}
  C'&: Tvy^r = (v+1)(v+T) \\
  C''&: Tu = z^{r-1}(z+uT)(z+u) \\
  C'''&: Tw = z^{r-1}(wz+T)(wz+1).
\end{align*}
Now glue the 4 affine curves together via
\begin{gather*}
  v = \frac{1}{x} = \frac{z}{u} = wz \qquad y = \frac{1}{z} \\
  u = \frac{x}{y} = \frac{1}{vy} = \frac{1}{w} \qquad z = \frac{1}{y}\\
  w = \frac{y}{x} = vy = \frac{1}{u} \qquad z = \frac{1}{y}
\end{gather*}
One checks that the resulting curve is generically smooth.
\jenn{Why do we have 4 affine patches? It seems to me that the last curve is unnecessary. The curve $C$ corresponds to the patch $Z \neq 0$, where $Z$ is the variable used to projectivize $C$. The curve $C'$ corresponds to the patch $x \neq 0$, and the curve $C''$ corresponds to $y \neq 0$. It seems that $C'''$ corresponds to $x,y \neq 0$, which is the intersection of the last two affine patches... Am I missing something?}

The special fibers on these charts are as follows:
\[
\begin{array}{lll}
C_{aff} & x + 1 = 0 & C_1 \\
C' & v(v + 1) = 0 & C_2, C_1 \\
C'' & z^r(z + u) = 0 & C_z, C_1 \\
C''' & z^rw(wz+1) = 0 & C_z, C_2, C_1.
\end{array}
\]
The third column gives names for the components of the special fiber, given in order of the factors in the equation. For example, in $C'''$, $C_2$ is given by the equation $w = 0$. The components $C_1$ and $C_2$ each have multiplicity 1, while $C_z$ has multiplicity $r$. Let $P = C_1 \cap C_z$. Note that $P$ lies in the chart $C''$, but in no others. \jenn{Doesn't $P$ also lie in $C'''$?} Let $Q = C_2 \cap C_z$. Then $Q$ lies in $C'''$ but in no other chart. One checks that neither $P$ nor $Q$ is regular. Therefore we need to blow up at both points.

\subsection{Blowing-up}
\label{sec:blowing-up}

\paragraph{Blow-up at $P$.}
\label{sec:blow-up-P}

We first blow up at $P$, which is given by $(u,z,T) = (0,0,0)$. Via the change of variables
\[
z = uv \qquad T = u\tau
\]
we obtain the arithmetic surface with equation
\[
\tau = u^{r-1}v^{r-1}(v + u\tau)(v + 1).
\]
This has special fiber given by $T = u\tau = 0$, with components
\begin{align*}
  \tilde{C_1}&: (v + 1 = b = 0) \\
  \tilde{C_z}&: (v = \tau = 0) \\
  E_P&: (u = \tau = 0).
\end{align*}
In the above, $\tilde{C_\cdot}$ is the strict transform of the corresponding component with the same multiplicities as before. The third component $E_P$ is the exceptional component, and has multiplicity $(r-1)$. The special fiber therefore looks like

[picture]

The intersection points are $P_1 = \tilde{C_1} \cap E_P$ given by $u=v+1=\tau=0$, and $P_2 = \tilde{C_z} \cap E_P$ given by $u=v=\tau=0$. Writing the equation for our surface as
\[
u^{r-1}v^{r-1}(v + u\tau)(v + 1) - \tau = 0
\]
and taking the $\tau$ derivative on the left, we obtain
\[
u^rv^{r-1}(v + 1) - 1.
\]
We now observe that this expression equals $-1$ at both $P_1$ and $P_2$. Therefore both points are regular and we have resolved the singularity at $P$.

\paragraph{Blow-up at $Q$.}
\label{sec:blow-up-Q}

Recall that $Q$ lies on the affine chart with equation
\begin{equation}
  C''':Tw = z^{r-1}(wz + T)(wz + 1)\label{eq:C'''}
\end{equation}

and has coordinates $(w,z,T) = (0,0,0)$. To desingularize, we will need to do $r-1$ blow-ups. Fortunately, the blow-ups can be implemented inductively. We do the first blow-up and explain how the remaining $r-2$ blow-ups are done. Throughout, the third factor $wz + 1$ and its transforms correspond to $C_1$ and its transforms, and so we will omit them from our discussion. Also, to keep things clean we will use the same symbol to denote a component and its strict transform.

The first blow-up is accomplished by gluing together two charts with $C'''-\{Q\}$. The first chart consists of the affine curve with equation
\[
\tau = b^{r-1} w^{r-2} (bw + \tau) (b w^2 + 1)
\]
where $z = bw$ and $T = \tau w$. The special fiber is given by $T = 0$, and consists of the components with equation $\tau = b = 0$ and $\tau = w = 0$. The first component has multiplicity $r$ and is the strict transform of ${C_z}$. The second component is the exceptional component, and has multiplicity $r-1$. We call this latter component $E_{r-1}$. Notice that the strict transform of $C_2$ does not lie in this chart. The intersection of the two components occurs when $\tau = b = w = 0$. One verifies that the $\tau$-deriviative of the equation above is $1$ at the latter intersection point; therefore this chart is regular.

The second chart is given by the equation
\[
w_1 T_1 = z^{r-2}(w_1z + T_1)(w_1 z^2 + 1)
\]
where $w = w_1 z$ and $T = T_1 z$. The special fiber consists of two components given by $T_1 = w_1 = 0$ and $T_1 = z = 0$ respectively. The first component has multiplicity 1, and is the strict transform of $C_2$. The second component has multiplicity $r-1$ and is the exceptional component $E_{r-1}$. The intersection point $Q_1$ has coordinates $(w_1, z, T_1) = (0,0,0)$ and, if $r - 2 \geq 1$, is \emph{not} regular.

But now observe that if we omit the subscript $1$, we are exactly in the original situation of~\eqref{eq:C'''}. Iterating, at the $i$th stage we will have two charts, one regular and the other of the form
\[
w_i T_i = z^{r-i-1}(w_iz + T_i)(w_i z^{i+1} + 1).
\]
The special fiber here is a chain of rational curves ${C_z}$, $E_{r-1}$, $E_{r-2}$, \dots, $E_{r-i}$, ${C_2}$, where $E_j$ has multiplicity $j$, and each rational curve intersects the next one in the list at exactly one point. Every intersection point is regular except possibly for $E_{r-i} \cap C_2$ (or rather the strict transform of $C_2$). Computing partial derivatives at $(w_i, z, T_i) = (0,0,0)$, we see that this last node is regular if and only if $i = r - 1$. Thus we need to do $r - 1$ blow-ups to obtain a model regular at $Q$.

Piecing together with the charts for $P$, we have obtained a regular model for our curve $C$. We call this model $\sce$.

\paragraph{Blowing down}
\label{sec:blowing-down}

Next we must make sure our model is minimal via Castelnuovo's criterion, which states that
\begin{theorem}[\shahed{citation}]
  \shahed{Hypotheses, clean up.} Then minimal if and only if no components with $E \isom \Pro^1$, $E^2 = -1$.
\end{theorem}

\shahed{I believe that $C_2$ must be blown down.}

\section{Component group}
\label{sec:component-group}



\bibliographystyle{halpha}
\end{document}
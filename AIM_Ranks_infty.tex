\documentclass{article}
\usepackage{amsmath, amsthm, amssymb}
\usepackage[all]{xy}
\usepackage[pagebackref,colorlinks]{hyperref}
\usepackage{mathrsfs}

% Color comments!
\usepackage[usenames,dvipsnames]{color}
% Color comments

\newcommand{\jenn}[1]{{\color{magenta} \sf $\clubsuit\clubsuit\clubsuit$ Jenn: [#1]}}
\newcommand{\shahed}[1]{{\color{Purple} \sf $\clubsuit\clubsuit\clubsuit$ Shahed: [#1]}}

\newcommand{\sce}{\mathscr{C}^{\textsf{ex}}}
\newcommand{\scd}{\mathscr{C}}
\newcommand{\caff}{C_{\textsf{aff}}}


\theoremstyle{plain}
\newtheorem*{reftheorem}{Theorem}
\newtheorem{theorem}{Theorem}[section]
\newtheorem{corollary}[theorem]{Corollary}
\newtheorem{proposition}[theorem]{Proposition}
\newtheorem{lemma}[theorem]{Lemma}
\newtheorem{conjecture}[theorem]{Conjecture}
\newtheorem{problem}{Problem}
\newtheorem{question}{Question}
\newtheorem*{question*}{Question}
\newtheorem{claim}{Claim}

\theoremstyle{definition}
\newtheorem{definition}[theorem]{Definition}

\theoremstyle{remark}
\newtheorem{remark}[theorem]{Remark}
\newtheorem{example}[theorem]{Example}

% General
\renewcommand{\emptyset}{\varnothing}
\newcommand{\hra}{\hookrightarrow}
\newcommand{\righthookarrow}{\hookrightarrow}
\newcommand{\isom}{\cong}
\newcommand{\too}{\longrightarrow}
\newcommand{\isomto}{\overset{\sim}{\longrightarrow}}
\newcommand{\nto}[1]{\overset{#1}{\longrightarrow}}
\newcommand{\nsubset}{\not\subset}
\renewcommand{\phi}{\varphi}
\newcommand{\To}{\Rightarrow}
\newcommand{\ilim}{\displaystyle\lim_{\leftarrow}}
\newcommand{\dirlim}{\displaystyle\lim_{\rightarrow}}
\newcommand{\eps}{\varepsilon}
\renewcommand{\bar}[1]{\overline{#1}}
\renewcommand{\tilde}[1]{\widetilde{#1}}
\DeclareMathOperator{\car}{char}
\DeclareMathOperator{\rk}{rk}
\DeclareMathOperator{\coker}{coker}
\DeclareMathOperator{\Hom}{Hom}
\DeclareMathOperator{\Aut}{Aut}
\DeclareMathOperator{\End}{End}
\DeclareMathOperator{\im}{im}
\DeclareMathOperator{\pgl}{PGL}
\DeclareMathOperator{\Gl}{GL}
\DeclareMathOperator{\Sl}{SL}

% Number theory
\newcommand{\Qbar}{\ensuremath{\overline{\Q}}}
\newcommand{\Kb}{\overline{K}}
\newcommand{\Fb}{\overline{F}}
\newcommand{\kb}{\overline{k}}
\newcommand{\Xbar}{\overline{X}}
\newcommand{\Cbar}{\overline{C}}
\newcommand{\R}{\ensuremath{\mathbb{R}}}
\newcommand{\C}{\ensuremath{\mathbb{C}}}
\newcommand{\F}{\ensuremath{\mathbb{F}}}
\newcommand{\fp}{\ensuremath{\mathbb{F}_p}}
\newcommand{\sm}{\ensuremath{\mathfrak{m}}}
\newcommand{\Q}{\ensuremath{\mathbb{Q}}}
\newcommand{\Z}{\ensuremath{\mathbb{Z}}}
\newcommand{\ok}{\mathscr{O}_K}
\DeclareMathOperator{\Gal}{Gal}
\DeclareMathOperator{\inv}{inv}
\DeclareMathOperator{\Nm}{Nm}
\DeclareMathOperator{\tr}{Tr}

% Algebraic geometry
\newcommand{\sA}{\ensuremath{\mathscr{A}}}
\newcommand{\sO}{\ensuremath{\mathscr{O}}}
\newcommand{\sL}{\ensuremath{\mathscr{L}}}
\newcommand{\sK}{\ensuremath{\mathscr{K}}}
\newcommand{\sF}{\ensuremath{\mathscr{F}}}
\newcommand{\A}{\ensuremath{\mathbb{A}}}
\newcommand{\Pro}{\ensuremath{\mathbb{P}}}
\newcommand{\G}{\ensuremath{\mathbb{G}}}
\newcommand{\sG}{\mathscr{G}}
\newcommand{\sX}{\mathscr{X}}
\DeclareMathOperator{\Supp}{Supp}
\DeclareMathOperator{\Div}{Div}
\DeclareMathOperator{\dv}{div}
\DeclareMathOperator{\Pic}{Pic}
\DeclareMathOperator{\P0}{Pic^0}
\DeclareMathOperator{\Spec}{Spec}

\begin{document}

\title{Regular model at infinity}
\author{Jenn Park \and Shahed Sharif}
\maketitle

Let $k$ be the finite field $\F_q$ and $K = k(t)$. Let $r \geq 3$ be an integer. Let $C$ be the smooth projective curve with affine model
\[
C: xy^r = (x+1)(x+t).
\]
The purpose of this note is to compute the minimal proper regular model of $C$ at $t = \infty$, as well as the component group. In all of the following, we test regularity of various affine charts. These charts will be given by a single equation in $\mathbb{A}^3_k$ of the form $f(x_1,x_2,x_3) = g(x_1,x_2,x_3)$. We will make frequent use of the fact that a point $P_0$ on the surface is regular if and only if $\frac{\partial f}{\partial x_i}(P_0) \neq \frac{\partial g}{\partial x_i}(P_0)$ for some $i$.

\section{Desingularization}
\label{sec:desingularization}

Let $T = \frac{1}{t}$, so that our curve has affine piece given by
\[
\caff:Txy^r = (x+1)(Tx+1).
\]
To desingularize the generic fiber, consider the affine curves
\begin{align*}
  C'&: Tvy^r = (v+1)(v+T) \\
  C''&: Tu = z^{r-1}(z+uT)(z+u) \\
  C'''&: Tw = z^{r-1}(wz+T)(wz+1).
\end{align*}
Now glue the 4 equations together via
\begin{gather*}
  v = \frac{1}{x} = \frac{z}{u} = wz \qquad y = \frac{1}{z} \\
  u = \frac{x}{y} = \frac{1}{vy} = \frac{1}{w} \qquad z = \frac{1}{y}\\
  w = \frac{y}{x} = vy = \frac{1}{u} \qquad z = \frac{1}{y}
\end{gather*}
One checks that the resulting curve is generically smooth.
\begin{remark}
  If one projectivizes $\caff$, one obtains the equation
  \[
  TXY^r = Z^{r-1}(X+Z)(TX+Z)
  \]
  in $\Pro^2_K$. The chart $C''$ corresponds to the affine patch with $Y = 1$. When setting $X = 1$, one finds that the curve is not generically smooth; it has a cusp at $[1:0:0]$. Normalizing, one obtains the equation for $C'''$. 
\end{remark}


The special fibers on these charts are as follows:
\[
\begin{array}{lll}
\caff & x + 1 = 0 & C_1 \\
C' & v(v + 1) = 0 & C_2, C_1 \\
C'' & z^r(z + u) = 0 & R_1, C_1 \\
C''' & z^rw(wz+1) = 0 & R_1, C_2, C_1.
\end{array}
\]
The third column gives names for the components of the special fiber, given in order of the factors in the equation. For example, in $C'''$, $C_2$ is given by the equation $w = 0$. The components $C_1$ and $C_2$ each have multiplicity 1, while $R_1$ has multiplicity $r$. Let $P = C_1 \cap R_1$. Note that $P$ lies in the chart $C''$, but in no others. Let $Q = C_2 \cap R_1$. Then $Q$ lies in $C'''$ but in no other chart. One checks that every point but $P$ and $Q$ is regular. Therefore we need to blow up at these two points.

\subsection{Blowing-up}
\label{sec:blowing-up}

\subsubsection{Blow-up at $P$.}
\label{sec:blow-up-P}

We first blow up at $P$, which is given by $(u,z,T) = (0,0,0)$. The blow-up is given by two charts. The first chart is computed via the change of variables
\[
z = uv \qquad T = u\tau.
\]
We obtain the arithmetic surface with equation
\[
\tau = u^{r-1}v^{r-1}(v + u\tau)(v + 1).
\]
This has special fiber given by $T = u\tau = 0$, with components
\begin{align*}
  \tilde{C_1}&: (v + 1 = \tau = 0) \\
  \tilde{R_1}&: (v = \tau = 0) \\
  D_r&: (u = \tau = 0).
\end{align*}
In the above, $\tilde{C_1}$, $\tilde{R_1}$ are the strict transforms of $C_1$, $R_1$ respectively with the same multiplicities as before. To compute the multiplicity of the third component $D_r$, we consider the ring
\[
\frac{k[u,v,\tau]_{(u,\tau)}}{(u^{r-1}v^{r-1}(v + u\tau)(v + 1) - \tau)}.
\]
This is the local ring for the subscheme $D_r$. Since $D_r$ is a prime divisor, the local ring is a discrete valuation ring $\sO_{C,D_r}$ with valuation, say, $\nu$. Then the multiplicity of $D_r$ in the special fiber is $\nu(T) = \nu(u\tau)$. But this valuation equals the length of the Artinian ring
\[
\frac{\sO_{C,D_r}}{(u\tau)}.
\]
Let $\alpha = v^{r}(v+1)$. Then the above ring equals
\[
\frac{k[u,v,\tau]_{(u,\tau)}}{(\alpha u^{r-1} - \tau, u\tau)}.
\]
The first expressions allows us to substitute $\tau = \alpha u^{r-1}$. The second expression then becomes $\alpha u^r$. Since $\nu(\alpha) = 0$, our Artinian ring is isomorphic to
\[
\frac{k[u,v]_{(u)}}{(u^r)}.
\]
Therefore the multiplicity of $D_r$ in the special fiber is $r$.

The intersection points are $P_1 = \tilde{C_1} \cap D_r$ given by $u=v+1=\tau=0$, and $P_z = \tilde{R_1} \cap D_r$ given by $u=v=\tau=0$. (Observe that $\tilde{C_1} \cap \tilde{R_1} = \emptyset$.) Writing the equation for our surface as
\[
u^{r-1}v^{r-1}(v + u\tau)(v + 1) - \tau = 0
\]
and taking the $\tau$ derivative on the left, we obtain
\[
u^rv^{r-1}(v + 1) - 1.
\]
We now observe that this expression equals $-1$ at both $P_1$ and $P_z$. Therefore both points are regular.

Now we compute the second chart. This chart is given by the change of variables
\[
u=zx \qquad T=z\pi
\]
whence we obtain the equation
\[
x\pi = z^{r-1}(1 + zx\pi)(1 + x).
\]
The special fiber is given by $T = z\pi = 0$, which gives us the components
\begin{align*}
  \tilde{C_1}&: (1 + x = \pi = 0) \\
  D_r&: (z = \pi = 0) \\
  D_1&: (z = x = 0).
\end{align*}
Observe that $D_1$ has multiplicity $1$. The intersection $\tilde{C_1} \cap D_r$ occurs when $1 + x = \pi = z = 0$. One checks that this intersection point is in fact $P_1$. The intersection $D_r \cap D_1$ occurs when $x = \pi = z = 0$. But now note that $x = \frac{1}{v}$. Therefore this intersection point is distinct from $P_2$. (Or easier: $\tilde{R_1}$ does not appear in this chart for the same reason.) We call this new point $P_3$. One checks that $P_3$ is \emph{not} regular.

In order to resolve $P_3$, we will desingularize inductively. To ease notation, write $\alpha$ for $(1 + zx\pi)(1 + x)$, so that
\begin{itemize}
    \item the chart containing $P_3$ has equation $x\pi = \alpha z^{r-1}$,
    \item $\alpha(P_3) \neq 0$, and
    \item $T = z\pi$.
\end{itemize}

We now desingularize by replacing $P_3$ with 2 charts. The first chart is given by
\[
\pi_0 = \alpha x^{r-3} z_0^{r-1}
\]
where $\pi = x \pi_0$ and $z = x z_0$. Then $T = x^2 \pi_0 z_0$ the special fiber consists of 2 components, one given by $\pi_0 = z_0 = 0$ and of multiplicity $r$, the other given by $\pi_0 = X = 0$ having multiplicity $r-1$. One verifies that the first component is $D_r$; the second component, which we call $D_{r-1}$, is exceptional; and the intersection is transverse. Furthermore, taking the $\pi_0$ derivative shows that this chart is regular.

  The second chart is given by $x_0 \pi = \alpha z^{r-2}$, where $X = x_0 z$. The special fiber then consists of the components $\pi = z = 0$ with multiplicity $r-1$ and $x_0 = z = 0$ with multiplicity $1$. One verifies that these are $D_{r-1}$ and $D_1$ respectively. The intersection point is regular if and only if $r = 3$, in which case the claim is proved. If $r > 3$, then we are in essentially the same situation as with $P_3$ except that $x$ has become $x_0$ and $r$ has been replaced with $r-1$. We can therefore iterate, thus obtaining a ``tail'' of rational curves $D_r, D_{r-1}, \dots, D_1$, each crossing transversely with adjacent curves in the list, and such that $D_i$ has multiplicity $i$ in the special fiber.


\subsubsection{Blow-up at $Q$.}
\label{sec:blow-up-Q}

Recall that $Q$ lies on the affine chart with equation
\begin{equation}
  C''':wT = z^{r-1}(wz + T)(wz + 1)\label{eq:C'''}
\end{equation}
and has coordinates $(w,z,T) = (0,0,0)$. Just as in the case of $P$, we will desingularize at $Q$ recursively; however, the situation is rather more complicated.  

One should keep in mind the following: blowing-up is a local procedure; and $wz + 1 = T = 0$ is an equation for $C_1$, a component which does not pass through $Q$. Therefore, the factor $wz + 1$ and its transforms can be safely ignored when computing special fibers below.

We start by replacing $Q$ with two charts. Chart 1 is given by the equation
\[
T_0 = b^{r-1}w^{r-2}(bw + T_0)(bw^2 + 1)
\]
where $z = bw$ and $T = T_0w$. The special fiber consists of 2 components:
\begin{itemize}
    \item $R_1: T_0 = b = 0$ with multiplicity $r$, and
    \item $R_2: T_0 = w = 0$ also with multiplicity $r$.
\end{itemize}
 The second component is exceptional. By considering the $T_0$ derivative, one sees that this chart is regular.

Chart 2 is given by the equation
\[
w_1 T_1 = z^{r-2}(w_1z + T_1)(w_1 z^2 + 1)
\]
where $w = w_1 z$ and $T = T_1 z$. The special fiber consists of three components:
\begin{itemize}
    \item $C_2: T_1 = w_1 = 0$ with multiplicity $1$,
    \item $R_2: T_1 = z = 0$ with multiplicity $r$, and
    \item $E_1: z = w_1 = 0$ with multiplicity $1$.
\end{itemize}
The third component is exceptional. The intersection point $Q_1$ has coordinates $(w_1, z, T_1) = (0,0,0)$ and, if $r \geq 3$, is \emph{not} regular.

We therefore continue by blowing-up chart 2. We do this recursively as follows. For $1 \leq i \leq r-3$, let chart $(2,i)$ be the chart given by
  \[
  w_i T_i = z^{r-i-1} (w_i z + T_i) (w_i z^{i+1} + 1).
  \]
  Let $T = T_i z^i$. Then the special fiber ($T = 0$) consists of the components
  \begin{itemize}
      \item $C_2: T_i = w_i = 0$ with multiplicity 1,
      \item $R_{i+1}: T_i = z = 0$ with multiplicity $r$, and
      \item $E_i: z = w_i = 0$ with multiplicity $i$.
  \end{itemize}
  Note that chart 2 is the same as chart $(2,1)$. One verifies that this chart is nonregular only at $Q_i: (w_i, z, T_i) = (0, 0, 0)$. We blow up $Q_i$ via 3 charts.

  The first chart is $(2,i+1)$, which we glue to $(2,i)$ away from $Q_i$ using $w_i = z w_{i+1}$ and $T_i = z T_{i+1}$. One verifies that this is consistent with the construction above.

  The second chart is $(2,i)a$ given by $w_i = w_a T_i$ and $z = z_a T_i$. (This is an abuse of notation; one should say $w_{i,a}$ and $z_{i,a}$ or some such.) Then $T = z_a^i T_i^{i+1}$. Our chart is given by
  \[
  w_a = z_a^{r-i-1} T_i^{r-i-2} (w_a z_a T_i + 1) (w_a z_a^{i+1} T_i^{i+2} + 1).
  \]
  The special fiber consists of
  \begin{itemize}
      \item $E_i: z_a = w_a = 0$ with multiplicity $i$, and
      \item $E_{i+1}: T_i = w_a = 0$ with multiplicity $i+1$.
  \end{itemize}
  By computing the $w_a$ derivative, one checks that this chart is regular.

  The third chart is $(2,i)b$ given by $z = w_i z_b$ and $T_i = w_i T_b$. Then $T = w_i^{i+1} z_b^i T_b$ and the equation is
  \[
  T_b = w_i^{r-i-2} z_b^{r-i-1} (w_i z_b + T_b) (w_i^{i+2} z_b^{i+1} + 1).
  \]
  The chart is regular (take the $T_b$ derivative) and the special fiber has components
  \begin{itemize}
      \item $R_{i+1}: T_b = z_b = 0$ with multiplicity $r$, and
      \item $R_{i+2}: T_b = w_i = 0$ with multiplicity $r$.
  \end{itemize}

  The remaining and, as it will turn out, final case occurs when $i = r - 2$. (Note that this is the only case we need consider when $r = 3$.) In this case, we have chart $(2, r-2)$ given by
  \[
  w_{r-2} T_{r-2} = z (w_{r-2} z + T_{r-2}) (w_{r-2} z^{r-1} + 1)
  \]
  with $w_{r-2}, T_{r-2}$ given as above, and $T = T_{r-2} z^{r-2}$. We blow up as before; that is, construct the charts $(2, r-1)$, $(2, r-2)a$ and $(2, r-2)b$ as above. We now verify that there are two differences: every chart is regular, and the special fibers for $(2, r-1)$ and $(2,r-1)b$ differ from the cases for smaller $i$.

  For chart $(2, r-1)$, the equation is
  \[
  w_{r-1} T_{r-1} = (w_{r-1} z + T_{r-1}) (w_{r-1} z^{r} + 1)
  \]
  and $T = T_{r-1} z^{r-1}$. The special fiber consists of the components
  \begin{itemize}
      \item $C_2: T_{r-1} = w_{r-1} = 0$ with multiplicity 1, and
      \item $R_r: T_{r-1} = z = 0$ with multiplicity $r$.
  \end{itemize}
  One checks that the $T_{r-1}$ partial derivative gives equality if and only if $z = 0$ and $w_{r-1} = 1$. But plugging these values into the $z$-derivative yields an inequality. Therefore this chart is regular.

  For chart $(2, r-2)a$, the equation is
  \[
  w_a = z_a (w_a z_a T_{r-2} + 1) ( w_a z_a^{r-1} T_i^{r} + 1)
  \]
  with special fiber consisting of the components
  \begin{itemize}
      \item $E_{r-2}: z_a = w_a = 0$ with multiplicity $r-2$, and
      \item $E_{r-1}: T_{r-2} = w_a - z_a = 0$ with multiplicity $r-1$.
  \end{itemize}
  The argument for regularity is the same as for the general charts $(2,i)a$.

  Finally, the chart $(2, r-2)b$ is given by
  \[
  T_b = z_b (w_{r-2} z_b + T_b) (w_{r-2}^{r} z_b^{r-1} + 1)
  \]
  with special fiber consisting of the components
  \begin{itemize}
      \item $R_{r-1}: T_b = z_b = 0$ with multiplicity $r$,
      \item $E_r: T_b = w_{r-2} = 0$ with multiplicity $r$, and
      \item $E_{r-1}: w_{r-2} = z_b - 1 = 0$ with multiplicity $r-1$.
  \end{itemize}
  Checking the $T_b$ deriviative, we see that we have equality only along $E_{r-1}$. But the $w_{r-2}$ derivative along $E_{r-1}$ always yields an inequality. Therefore this chart is regular.

  Let $\scd$ be the resulting regular model for $C$, and write $C_k$ for its special fiber at $T = 0$.

\subsubsection{Dual graph}
\label{sec:dual-graph}

Recall that the dual graph of $C_k$ is defined to be the graph such that the set of irreducible components of $C_k$ is the set of vertices, and the set of intersection points between two such components is the set of edges between the corresponding vertices. Putting our calculations above together, we obtain the dual graph for $C_k$ pictured in Figure~\ref{fig:superelliptic-dual-graph}.
\begin{figure}\centering
    \[
\xygraph{
  !{<0cm,0cm>;<1.5cm,0cm>:<0cm,1.25cm>::}
  !{(1,5) }*{\bullet}="r1"
  !{(2,5) }*{\bullet}="r2"
  !{(3,5) }*{\quad\cdots\quad}="rspace"
  !{(4,5) }*{\bullet}="rrm"
  !{(1,5.6) }*{R_1}
  !{(2,5.6) }*{R_2}
  !{(4,5.6) }*{R_{r-1}}
  !{(0,4) }*{C_1\;\bullet}="c1"
  !{(5,4) }*{\bullet\; C_2}="c2"
  !{(1,0) }*{\bullet}="d1"
  !{(1,1) }*{\bullet}="d2"
  !{(1,1.5) }*{}="dbelow"
  !{(1,2.1) }*{\vdots}="dspace"
  !{(1,2.6) }*{}="dabove"
  !{(1,3) }*{\bullet}="drm"
  !{(1,4) }*{\bullet}="dr"
  !{(.6,0) }*{D_1}
  !{(.6,1) }*{D_2}
  !{(.5,3) }*{D_{r-1}}
  !{(1.4,4) }*{D_r}
  !{(4,0) }*{\bullet}="e1"
  !{(4,1) }*{\bullet}="e2"
  !{(4,1.5) }*{}="ebelow"
  !{(4,2.1) }*{\vdots}="espace"
  !{(4,2.6) }*{}="eabove"
  !{(4,3) }*{\bullet}="erm"
  !{(4,4) }*{\bullet}="er"
  !{(4.4,0) }*{E_1}
  !{(4.4,1) }*{E_2}
  !{(4.6,3) }*{E_{r-1}}
  !{(3.6,4) }*{E_r}
  "r1"-"r2"
  "r2"-"rspace"
  "rspace"-"rrm"
  "r1"-"dr"
  "c1"-"dr"
  "dr"-"drm"
  "drm"-"dabove"
  "dbelow"-"d2"
  "d2"-"d1"
  "rrm"-"er"
  "c2"-"er"
  "er"-"erm"
  "erm"-"eabove"
  "ebelow"-"e2"
  "e2"-"e1"
}
\]



%%% Local Variables:
%%% TeX-master: "AIM_Ranks_Infty"
%%% End:

  \caption{Dual graph of $C_k$}
\label{fig:superelliptic-dual-graph}
\end{figure}

All intersections are transverse. The components $C_1$ and $C_2$ have multiplicity $1$. The $R_i$ each have multiplicity $r$. The $D_i$ and $E_i$ have multiplicity $i$.


\subsection{Minimality}
\label{sec:minimality}

Next we must make sure our model is minimal via Castelnuovo's criterion, which we now state.
\begin{theorem}[Prop.~IV.7.5 in \cite{silvermanATAEC}]\label{thm:castelnuovo}
  Let $R$ be a discrete valuation ring with algebraically closed residue field. Let $\scd$ be a regular arithmetic surface proper over $R$ whose generic fiber is a nonsingular projective curve of genus $g \geq 1$. Then $\scd$ is minimal if and only if the special fiber of $\scd$ has no divisors $E$ for which $E \isom \Pro^1$ and $E^2 = -1$.
\end{theorem}

We need to check that $\scd$ is a minimal model via Theorem~\ref{thm:castelnuovo}. Every irreducible component is isomorphic to $\Pro^1$, so it remains to check the self-intersections. To compute these self-intersections we use the following theorem.
\begin{theorem}[Prop.~IV.7.3 in \cite{silvermanATAEC}]\label{thm:fibral-intersect-total}
  Let $\scd$ be a regular arithmetic surface proper over a discrete valuation ring, and let $C_k$ be the special fiber. Let $D$ be a fibral divisor. Then $D \cdot C_k = 0$.
\end{theorem}

In our case, all of the geometric intersections are transverse, so we can simplify the above theorem.
\begin{proposition}
  Let $\scd$ be a regular arithmetic surface proper over a discrete valuation ring, and let $C_k$ be the special fiber. Let $C_k^{\sf red}$ be the associated reduced scheme. Suppose that every singular point of $C_k^{\sf red}$ consists of a transverse crossing of two irreducible components. Let $V$ be an irreducible component of $C_k$. Label the remaining irreducible components $V_1, V_2,$ etc. Then
  \[
  V^2 = -\frac{1}{m} \sum m_i n_i
  \]
  where $m$ is the multiplicity of $V$, $m_i$ is the multiplicity of $V_i$, and $n_i$ is the number of intersection points of $V$ and $V_i$.
\end{proposition}

\begin{proof}
  The total fiber is
  \[
  C_k = mV + \sum m_i V_i.
  \]
  By Theorem~\ref{thm:fibral-intersect-total}, we have
  \begin{align*}
    0 &= V \cdot C_k \\
    &= mV^2 + \sum m_i (V \cdot V_i) \\
    &= mV^2 + \sum m_i n_i
  \end{align*}
  from which the claim follows.
\end{proof}

From the proposition, one verifies that $C_1$ and $C_2$ have self-intersection $-r$, and the remaining components each have self-intersection $-2$. By Castelnuovo's criterion, $\scd$ is a minimal model. 




\section{Component group}
\label{sec:component-group}

To compute the component group of the Jacobian of a curve, one typically first base-extends to a strictly henselian ring. The following result of Bosch-Liu allows us to do this without complication:

\shahed{Insert result here.}

The background material in the remainder of this section can be found in~\cite[Ch. 9]{blr}. In the latter, the authors define the \emph{geometric multiplicity} of a component of the special fiber, and use the notation $e_i$. Our base field is algebraically closed, and it follows that all geometric multiplicities are $1$. (See \cite[Defn~9.1.3]{blr}.)

In order to compute the component group at $\infty$ of $C$, we make use of the \emph{intersection matrix} of the special fiber. This is the matrix whose $(i,j)$ entry is the intersection number $(V_i \cdot V_j)$, where $V_i$ is some labeling of the irreducible components of the special fiber. We will order the components as follows: $D_1, D_2, \dots, D_r, R_1, \dots, R_{r-1}, E_r, \dots, E_1, C_1, C_2$. Then the intersection matrix is
\[
A = \left[\begin{array}{rrrrrrrrr|rr}
  -2 & 1 & & & & & & & & & \\
  1 & -2 & 1 & & & & & & & & \\
  & 1 & -2 & 1 & & & & & & & \\
  & & & & & & & & & & \\
  & & & & & & & & & 1 & \\
  & & & & & \ddots & & & & \vdots & \\
  & & & & & & & & & & 1 \\
  & & & & & & & & & & \\
  & & & & & & & 1 & -2 & & \\ \hline
  & & & & 1 & \dots & & & & -r & \\
  & & & & & & 1 & & & & -r
\end{array}\right].
\]
The $1$s in the right-hand rectangle in the matrix occur at rows $r$ and $2r$---that is, at the rows corresponding to $D_r$ and $E_r$. The $1$s in the bottom rectangle similarly occur at the $r$th and $2r$th columns respectively. The component group can then be computed via the following.
\begin{theorem}[Corollary 9.6.3 of \cite{blr}]\label{thm:elementary-divisors-comp-group}
  Suppose that $\scd$ is a flat, proper relative curve over a strictly henselian discrete valuation ring. Suppose also that $\scd$ is generically geometrically irreducible and that the residue field is algebraically closed. Let $n_1, \dots, n_{v-1}, 0$ be the elementary divisors of the intersection matrix $A$ for $C_k$. Let $J$ be the Jacobian of the generic fiber of $\scd$. Then the group of connected components of the N\'eron model of $J$ is isomorphic to
  \[
  \frac{\Z}{n_1\Z} \oplus \cdots \oplus \frac{\Z}{n_{v-1}\Z}.
  \]
\end{theorem}
Implicit in the statement of the theorem is that the intersection matrix has rank $r-1$. In our case, one can at least see that the vector of multiplicities $(1,2,\dots,r,r,\dots,r,r-1,\dots,2,1,1,1)$ lies in the kernel of $A$; this in fact holds in general.

We must therefore compute the elementary divisors of $A$. Recall that the elementary divisors are the diagonal entries in the Smith normal form of $A$. The Smith normal form is a diagonal matrix obtained from $A$ via any combination of elementary row and column operations. 

\paragraph{Step 1:}
\label{sec:step-1}

For $i = 1, \dots, 3r-3$, we do the following column operations:
\begin{itemize}
    \item Add twice column $i$ to column $i+1$.
    \item Subtract column $i$ from column $i+2$.
\end{itemize}
We then add twice column $3r-2$ to column $3r-1$. An easy induction shows that the resulting matrix is
\[
\left[\begin{array}{rrrrrrrr|rr}
  -2 & -3 & & & \dots & & & -3r & & \\
  1 & & & & & & & & & \\
  & 1 & & & & & & & & \\
  & & & & & & & & 1 & \\
  & & \ddots & & & & & & & \\
  & & & & & & & & & 1 \\
  & & & & & & 1 & 0 & & \\ \hline
  & & 1 & 2 & & \dots & & 2r & -r & \\
  & & & & 1 & 2 & \dots & r & & -r
\end{array}\right]
\]
Once again, the $1$s in the right-hand rectangle occur at rows $r$ and $2r$ respectively, and the $1$s in the bottom rectangle occur in columns $r$ and $2r$ respectively.

\paragraph{Step 2:}
\label{sec:step-2}

We now subtract column $r$ from column $3r$ (the second column from the right), and subtract column $2r$ from column $3r+1$:
\[
\left[\begin{array}{rrrrrrrr|rr}
  -2 & -3 & & & \dots & & & -3r & r+1 & 2r+1 \\
  1 & & & & & & & & & \\
  & 1 & & & & & & & & \\
  & & \ddots & & & & & & & \\
  & & & & & & 1 & 0 & & \\ \hline
  & & 1 & 2 & & \dots & & 2r & -r-1 & -r \\
  & & & & 1 & 2 & \dots & r & & -r-1
\end{array}\right].
\]

\paragraph{Step 3:}
\label{sec:step-3}

By the obvious row operations, we can zero out the first $3r-2$ entries of the first row and the last two rows. We then rearrange rows to obtain
\[
\left[\begin{array}{c|rrr}
  I & & & \\ \hline
  & -3r & r+1 & 2r+1 \\
  & 2r & -r-1 & -r \\
  & r & -r-1 &
\end{array}\right].
\]
In the above, $I$ denotes the $(3r-2) \times (3r-2)$ identity matrix. By Theorem~\ref{thm:elementary-divisors-comp-group}, we may consider only the $3\times 3$ block in the bottom right and omit everything else.

\bibliographystyle{alpha}
\bibliography{./aimbiblio}
\end{document}
\documentclass{article}
\usepackage{amsmath, amsthm, amssymb}
\usepackage[all]{xy}
\usepackage[pagebackref,colorlinks]{hyperref}
\usepackage{mathrsfs}

% Color comments!
\usepackage[usenames,dvipsnames]{color}
% Color comments

\newcommand{\jenn}[1]{{\color{magenta} \sf $\clubsuit\clubsuit\clubsuit$ Jenn: [#1]}}
\newcommand{\shahed}[1]{{\color{Purple} \sf $\clubsuit\clubsuit\clubsuit$ Shahed: [#1]}}

\newcommand{\sce}{\mathscr{C}^{\textsf{ex}}}
\newcommand{\scd}{\mathscr{C}}
\newcommand{\caff}{C_{\textsf{aff}}}


\theoremstyle{plain}
\newtheorem*{reftheorem}{Theorem}
\newtheorem{theorem}{Theorem}[section]
\newtheorem{corollary}[theorem]{Corollary}
\newtheorem{proposition}[theorem]{Proposition}
\newtheorem{lemma}[theorem]{Lemma}
\newtheorem{conjecture}[theorem]{Conjecture}
\newtheorem{problem}{Problem}
\newtheorem{question}{Question}
\newtheorem*{question*}{Question}
\newtheorem{claim}{Claim}

\theoremstyle{definition}
\newtheorem{definition}[theorem]{Definition}

\theoremstyle{remark}
\newtheorem{remark}[theorem]{Remark}
\newtheorem{example}[theorem]{Example}

% General
\renewcommand{\emptyset}{\varnothing}
\newcommand{\hra}{\hookrightarrow}
\newcommand{\righthookarrow}{\hookrightarrow}
\newcommand{\isom}{\cong}
\newcommand{\too}{\longrightarrow}
\newcommand{\isomto}{\overset{\sim}{\longrightarrow}}
\newcommand{\nto}[1]{\overset{#1}{\longrightarrow}}
\newcommand{\nsubset}{\not\subset}
\renewcommand{\phi}{\varphi}
\newcommand{\To}{\Rightarrow}
\newcommand{\ilim}{\displaystyle\lim_{\leftarrow}}
\newcommand{\dirlim}{\displaystyle\lim_{\rightarrow}}
\newcommand{\eps}{\varepsilon}
\renewcommand{\bar}[1]{\overline{#1}}
\renewcommand{\tilde}[1]{\widetilde{#1}}
\DeclareMathOperator{\car}{char}
\DeclareMathOperator{\rk}{rk}
\DeclareMathOperator{\coker}{coker}
\DeclareMathOperator{\Hom}{Hom}
\DeclareMathOperator{\Aut}{Aut}
\DeclareMathOperator{\End}{End}
\DeclareMathOperator{\im}{im}
\DeclareMathOperator{\pgl}{PGL}
\DeclareMathOperator{\Gl}{GL}
\DeclareMathOperator{\Sl}{SL}

% Number theory
\newcommand{\Qbar}{\ensuremath{\overline{\Q}}}
\newcommand{\Kb}{\overline{K}}
\newcommand{\Fb}{\overline{F}}
\newcommand{\kb}{\overline{k}}
\newcommand{\Xbar}{\overline{X}}
\newcommand{\Cbar}{\overline{C}}
\newcommand{\R}{\ensuremath{\mathbb{R}}}
\newcommand{\C}{\ensuremath{\mathbb{C}}}
\newcommand{\F}{\ensuremath{\mathbb{F}}}
\newcommand{\fp}{\ensuremath{\mathbb{F}_p}}
\newcommand{\sm}{\ensuremath{\mathfrak{m}}}
\newcommand{\Q}{\ensuremath{\mathbb{Q}}}
\newcommand{\Z}{\ensuremath{\mathbb{Z}}}
\newcommand{\ok}{\mathscr{O}_K}
\DeclareMathOperator{\Gal}{Gal}
\DeclareMathOperator{\inv}{inv}
\DeclareMathOperator{\Nm}{Nm}
\DeclareMathOperator{\tr}{Tr}

% Algebraic geometry
\newcommand{\sA}{\ensuremath{\mathscr{A}}}
\newcommand{\sO}{\ensuremath{\mathscr{O}}}
\newcommand{\sL}{\ensuremath{\mathscr{L}}}
\newcommand{\sK}{\ensuremath{\mathscr{K}}}
\newcommand{\sF}{\ensuremath{\mathscr{F}}}
\newcommand{\A}{\ensuremath{\mathbb{A}}}
\newcommand{\Pro}{\ensuremath{\mathbb{P}}}
\newcommand{\G}{\ensuremath{\mathbb{G}}}
\newcommand{\sG}{\mathscr{G}}
\newcommand{\sX}{\mathscr{X}}
\DeclareMathOperator{\Supp}{Supp}
\DeclareMathOperator{\Div}{Div}
\DeclareMathOperator{\dv}{div}
\DeclareMathOperator{\Pic}{Pic}
\DeclareMathOperator{\P0}{Pic^0}
\DeclareMathOperator{\Spec}{Spec}

\begin{document}

\title{Regular model at infinity}
\author{Jenn Park \and Shahed Sharif}
\maketitle

Let $k$ be the finite field $\F_q$ and $K = k(t)$. Let $r \geq 3$ be an integer. Let $C$ be the smooth projective curve with affine model
\[
C: xy^r = (x+1)(x+t).
\]
The purpose of this note is to compute the minimal proper regular model of $C$ at $t = \infty$, as well as the component group.

\section{Desingularization}
\label{sec:desingularization}

Let $T = \frac{1}{t}$, so that our curve has affine piece given by
\[
\caff:Txy^r = (x+1)(Tx+1).
\]
To desingularize the generic fiber, consider the affine curves
\begin{align*}
  C'&: Tvy^r = (v+1)(v+T) \\
  C''&: Tu = z^{r-1}(z+uT)(z+u) \\
  C'''&: Tw = z^{r-1}(wz+T)(wz+1).
\end{align*}
Now glue the 4 equations together via
\begin{gather*}
  v = \frac{1}{x} = \frac{z}{u} = wz \qquad y = \frac{1}{z} \\
  u = \frac{x}{y} = \frac{1}{vy} = \frac{1}{w} \qquad z = \frac{1}{y}\\
  w = \frac{y}{x} = vy = \frac{1}{u} \qquad z = \frac{1}{y}
\end{gather*}
One checks that the resulting curve is generically smooth.
\begin{remark}
  If one projectivizes $\caff$, one obtains the equation
  \[
  TXY^r = Z^{r-1}(X+Z)(TX+Z)
  \]
  in $\Pro^2_K$. The chart $C''$ corresponds to the affine patch with $Y = 1$. When setting $X = 1$, one finds that the curve is not generically smooth; it has a cusp at $[1:0:0]$. Normalizing, one obtains the equation for $C'''$. 
\end{remark}


The special fibers on these charts are as follows:
\[
\begin{array}{lll}
\caff & x + 1 = 0 & C_1 \\
C' & v(v + 1) = 0 & C_2, C_1 \\
C'' & z^r(z + u) = 0 & C_z, C_1 \\
C''' & z^rw(wz+1) = 0 & C_z, C_2, C_1.
\end{array}
\]
The third column gives names for the components of the special fiber, given in order of the factors in the equation. For example, in $C'''$, $C_2$ is given by the equation $w = 0$. The components $C_1$ and $C_2$ each have multiplicity 1, while $C_z$ has multiplicity $r$. Let $P = C_1 \cap C_z$. Note that $P$ lies in the chart $C''$, but in no others. Let $Q = C_2 \cap C_z$. Then $Q$ lies in $C'''$ but in no other chart. One checks that every point but $P$ and $Q$ is regular. Therefore we need to blow up at these two points.

\subsection{Blowing-up}
\label{sec:blowing-up}

\paragraph{Blow-up at $P$.}
\label{sec:blow-up-P}

We first blow up at $P$, which is given by $(u,z,T) = (0,0,0)$. Via the change of variables
\[
z = uv \qquad T = u\tau
\]
we obtain the arithmetic surface with equation
\[
\tau = u^{r-1}v^{r-1}(v + u\tau)(v + 1).
\]
This has special fiber given by $T = u\tau = 0$, with components
\begin{align*}
  \tilde{C_1}&: (v + 1 = \tau = 0) \\
  \tilde{C_z}&: (v = \tau = 0) \\
  E_P&: (u = \tau = 0).
\end{align*}
In the above, $\tilde{C_1}$, $\tilde{C_z}$ are the strict transforms of $C_1$, $C_z$ respectively with the same multiplicities as before. To compute the multiplicity of the third component $E_P$, we consider the ring
\[
\frac{k[u,v,\tau]_{(u,\tau)}}{(u^{r-1}v^{r-1}(v + u\tau)(v + 1) - \tau)}.
\]
This is the local ring for the subscheme $E_P$. Since $E_P$ is a prime divisor, the local ring is a discrete valuation ring $\sO_{C,E_P}$ with valuation, say, $\nu$. Then the multiplicity of $E_P$ in the special fiber is $\nu(T) = \nu(u\tau)$. But this valuation equals the length of the Artinian ring
\[
\frac{\sO_{C,E_P}}{(u\tau)}.
\]
Let $\alpha = v^{r}(v+1)$. Then the above ring equals
\[
\frac{k[u,v,\tau]_{(u,\tau)}}{(\alpha u^{r-1} - \tau, u\tau)}.
\]
The first expressions allows us to substitute $\tau = \alpha u^{r-1}$. The second expression then becomes $\alpha u^r$. Since $\nu(\alpha) = 0$, our Artinian ring is isomorphic to
\[
\frac{k[u,v]_{(u)}}{(u^r)}.
\]
Therefore the multiplicity of $E_P$ in the special fiber is $r$.

The intersection points are $P_1 = \tilde{C_1} \cap E_P$ given by $u=v+1=\tau=0$, and $P_2 = \tilde{C_z} \cap E_P$ given by $u=v=\tau=0$. (Observe that $\tilde{C_1} \cap \tilde{C_z} = \emptyset$.) Writing the equation for our surface as
\[
u^{r-1}v^{r-1}(v + u\tau)(v + 1) - \tau = 0
\]
and taking the $\tau$ derivative on the left, we obtain
\[
u^rv^{r-1}(v + 1) - 1.
\]
We now observe that this expression equals $-1$ at both $P_1$ and $P_2$. Therefore both points are regular and we have resolved the singularity at $P$.

\shahed{Problem: If we compute the self-intersection $E_P^2$, we get a nonsensical result. Namely, we use the fact that $E_P$ intersected with the total fiber is zero, and that $E_P$ only intersects $C_1$ and $C_z$. The equations for all three components are linear, so all intersections are transversal. Therefore
  \begin{align*}
    0 &= E_P \cdot (C_1 + rE_p + rC_z) \\
    &= 1 + rE_p^2 + r
  \end{align*}
  from which $E_P^2 = -1 - \frac{1}{r}$. What is the error?}


\paragraph{Blow-up at $Q$.}
\label{sec:blow-up-Q}

Recall that $Q$ lies on the affine chart with equation
\begin{equation}
  C''':Tw = z^{r-1}(wz + T)(wz + 1)\label{eq:C'''}
\end{equation}
and has coordinates $(w,z,T) = (0,0,0)$. To desingularize, we will need to do $r-1$ blow-ups. Fortunately, the blow-ups can be implemented inductively. We do the first blow-up and explain how the remaining $r-2$ blow-ups are done. Throughout, the third factor $wz + 1$ and its transforms correspond to $C_1$ and its transforms, and so we will omit them from our discussion. Also, to keep things clean we will use the same symbol to denote a component and its strict transform.

The first blow-up is accomplished by gluing together two charts with $C'''-\{Q\}$. The first chart consists of the affine curve with equation
\[
\tau = b^{r-1} w^{r-2} (bw + \tau) (b w^2 + 1)
\]
where $z = bw$ and $T = \tau w$. The special fiber is given by $T = 0$, and consists of the components with equation $\tau = b = 0$ and $\tau = w = 0$. The first component is the strict transform of ${C_z}$ and has multiplicity $r$. The second component is the exceptional component, which we call $E_1$. Its multiplicity is computed in a similar manner as that of $E_P$; that is, as the length of the Artinian ring
\[
\frac{k[b,w,\tau]_{(\tau, w)}}{(b^{r-1}w^{r-2}(bw + \tau)(bw^2 + 1) - \tau, \tau w)}.
\]
To do this computation, we let $\alpha = b^r(bw^2 + 1)$ and notice that since $r \geq 3$, the first expression in the ideal becomes $\alpha w^{r-1} - \tau$. Substituting for $\tau$ in the second expression, we obtain $\alpha w^r$. Therefore $E_1$ has multiplicity $r$.

Notice that the strict transform of $C_2$ does not lie in this chart. The intersection of $C_z$ and $E_1$ occurs when $\tau = b = w = 0$. One verifies that the $\tau$-derivative of 
\[
b^{r-1}w^{r-2}(bw + \tau)(bw^2 + 1) - \tau
\]
is $-1$ at the latter intersection point, and in fact along $E_1$ in this affine patch; therefore this chart is regular.

The second chart is given by the equation
\[
w_1 T_1 = z^{r-2}(w_1z + T_1)(w_1 z^2 + 1)
\]
where $w = w_1 z$ and $T = T_1 z$. The special fiber consists of two components given by $T_1 = w_1 = 0$ and $T_1 = z = 0$ respectively. The first component has multiplicity 1, and is the strict transform of $C_2$. The second component has multiplicity $r$ and is the exceptional component $E_{1}$. The intersection point $Q_1$ has coordinates $(w_1, z, T_1) = (0,0,0)$ and, if $r - 2 \geq 1$, is \emph{not} regular.

But now observe that if we omit the subscript $1$ from all the variables, we are almost exactly in the original situation of~\eqref{eq:C'''}; the only difference lies in the power of $z$ in the last factor. Iterating, at the $i$th stage we will have two charts, one regular and the other of the form
\[
w_i T_i = z^{r-i-1}(w_iz + T_i)(w_i z^{i+1} + 1).
\]
The special fiber here is a chain of rational curves ${C_z}$, $E_{1}$, $E_{2}$, \dots, $E_{i}$, ${C_2}$, where $E_j$ has multiplicity $r$, and each rational curve intersects the next one in the list at exactly one point. Every intersection point is regular except possibly for $E_{i} \cap C_2$ (or rather the strict transform of $C_2$). Computing partial derivatives at $(w_i, z, T_i) = (0,0,0)$, we see that this last node is regular if and only if $i = r - 1$. Thus we need to do $r - 1$ blow-ups to obtain a model regular at $Q$.

Piecing together with the charts for $P$, we have obtained a regular model for our curve $C$. We call this model $\scd$. The special fiber consists of the rational curves $C_1, E_P, C_z, E_1, \dots, E_{r-1}, C_2$, where each component intersects only the next component in the list transversely, the first and last components have multiplicity 1, and the remaining components have multiplicity $r$.

\shahed{We have the same self-intersection problem for $E_{r-1}$; that is
  \begin{align*}
    0 &= E_{r-1} \cdot (rE_{r-2} + rE_{r-1} + C_2) \\
    &= r + r E_{r-1}^2 + 1.
  \end{align*}
  Not sure what's going on here.
  }


\paragraph{Minimality}
\label{sec:minimality}

Next we must make sure our model is minimal via Castelnuovo's criterion, which states that
\begin{theorem}[\shahed{citation}]\label{thm:castelnuovo}
  \shahed{Hypotheses, clean up.} Then minimal if and only if no components with $E \isom \Pro^1$, $E^2 = -1$.
\end{theorem}

To compute these self-intersections we use the following theorem.
\begin{theorem}[\shahed{citation to Prop 7.3 in Silverman's ATAEC}]\label{thm:fibral-intersect-total}
  Let $\scd$ be a regular arithmetic surface proper over a discrete valuation ring, and let $C_k$ be the special fiber. Let $D$ be a fibral divisor. Then $D \cdot C_k = 0$.
\end{theorem}

We need to check that $\scd$ is a minimal model via Theorem~\ref{thm:castelnuovo}. We have
\begin{align*}
  C_1 \cdot C_k &= C_1 \cdot (C_1 + rE_{P}) \\
  &= C_1^2 + r
\end{align*}
so $C_1^2 = -r$. Since $r \geq 3$, $C_1$ cannot be blown down. Similarly, $C_2^2 = -r$.

For $E_{i}$ with $1 \leq i \leq r-2$, we have
\begin{align*}
  E_{i} \cdot C_k &= E_{r-1} \cdot (rE_{i-1} + rE_{i} + rE_{i+1}) \\
  &= r + rE_i^2 + r
\end{align*}
where we use the convention that $E_0 = C_z$. Therefore $E_i^2 = -2$. Similar reasoning shows that $C_z^2 = -2$.

All that remain are $E_P$ and $E_{r-1}$, for which see the comments above.

\section{Component group}
\label{sec:component-group}



\bibliographystyle{halpha}
\end{document}
\documentclass{article}
\usepackage{amsmath, amsthm, amssymb}
\usepackage[all]{xy}
\usepackage[pagebackref,colorlinks]{hyperref}
\usepackage{mathrsfs}

\theoremstyle{plain}
\newtheorem*{reftheorem}{Theorem}
\newtheorem{theorem}{Theorem}[section]
\newtheorem{corollary}[theorem]{Corollary}
\newtheorem{proposition}[theorem]{Proposition}
\newtheorem{lemma}[theorem]{Lemma}
\newtheorem{conjecture}[theorem]{Conjecture}
\newtheorem{problem}{Problem}
\newtheorem{question}{Question}
\newtheorem*{question*}{Question}
\newtheorem{claim}{Claim}

\theoremstyle{definition} 
\newtheorem{definition}[theorem]{Definition}

\theoremstyle{remark}
\newtheorem{remark}[theorem]{Remark} 
\newtheorem{example}[theorem]{Example}

% General
\renewcommand{\emptyset}{\varnothing}
\newcommand{\hra}{\hookrightarrow}
\newcommand{\righthookarrow}{\hookrightarrow}
\newcommand{\isom}{\cong}
\newcommand{\too}{\longrightarrow}
\newcommand{\isomto}{\overset{\sim}{\longrightarrow}}
\newcommand{\nto}[1]{\overset{#1}{\longrightarrow}}
\newcommand{\nsubset}{\not\subset}
\renewcommand{\phi}{\varphi}
\newcommand{\To}{\Rightarrow}
\newcommand{\ilim}{\displaystyle\lim_{\leftarrow}}
\newcommand{\dirlim}{\displaystyle\lim_{\rightarrow}}
\newcommand{\eps}{\varepsilon}
\renewcommand{\bar}[1]{\overline{#1}}
\renewcommand{\tilde}[1]{\widetilde{#1}}
\DeclareMathOperator{\car}{char}
\DeclareMathOperator{\rk}{rk}
\DeclareMathOperator{\coker}{coker}
\DeclareMathOperator{\Hom}{Hom} 
\DeclareMathOperator{\Aut}{Aut}
\DeclareMathOperator{\End}{End}
\DeclareMathOperator{\im}{im} 
\DeclareMathOperator{\pgl}{PGL} 
\DeclareMathOperator{\Gl}{GL} 
\DeclareMathOperator{\Sl}{SL} 

% Number theory
\newcommand{\Qbar}{\ensuremath{\overline{\Q}}}
\newcommand{\Kb}{\overline{K}}
\newcommand{\Fb}{\overline{F}}
\newcommand{\kb}{\overline{k}}
\newcommand{\Xbar}{\overline{X}}
\newcommand{\Cbar}{\overline{C}}
\newcommand{\R}{\ensuremath{\mathbb{R}}}
\newcommand{\C}{\ensuremath{\mathbb{C}}}
\newcommand{\F}{\ensuremath{\mathbb{F}}}
\newcommand{\fp}{\ensuremath{\mathbb{F}_p}}
\newcommand{\sm}{\ensuremath{\mathfrak{m}}}
\newcommand{\Q}{\ensuremath{\mathbb{Q}}}
\newcommand{\Z}{\ensuremath{\mathbb{Z}}}
\newcommand{\ok}{\mathscr{O}_K}
\DeclareMathOperator{\Gal}{Gal} 
\DeclareMathOperator{\inv}{inv} 
\DeclareMathOperator{\Nm}{Nm} 
\DeclareMathOperator{\tr}{Tr} 

% Algebraic geometry
\newcommand{\sA}{\ensuremath{\mathscr{A}}}
\newcommand{\sO}{\ensuremath{\mathscr{O}}}
\newcommand{\sL}{\ensuremath{\mathscr{L}}}
\newcommand{\sK}{\ensuremath{\mathscr{K}}}
\newcommand{\sF}{\ensuremath{\mathscr{F}}}
\newcommand{\A}{\ensuremath{\mathbb{A}}}
\newcommand{\Pro}{\ensuremath{\mathbb{P}}}
\newcommand{\G}{\ensuremath{\mathbb{G}}}
\newcommand{\sG}{\mathscr{G}}
\newcommand{\sX}{\mathscr{X}}
\DeclareMathOperator{\Supp}{Supp}
\DeclareMathOperator{\Div}{Div}
\DeclareMathOperator{\dv}{div}
\DeclareMathOperator{\Pic}{Pic}
\DeclareMathOperator{\P0}{Pic^0} 
\DeclareMathOperator{\Spec}{Spec}

\begin{document}

\title{Regular model at infinity}
\author{Jenn Park \and Shahed Sharif}
\maketitle

Let $k$ be the finite field $\F_q$ and $K = k(t)$. Let $r \geq 3$ be an integer. Let $C$ be the smooth projective curve with affine model 
\[
C: xy^r = (x+1)(x+t).
\]
The purpose of this note is to compute the minimal proper regular model of $C$ at $t = \infty$, as well as the component group.

\section{Desingularization}
\label{sec:desingularization}

Let $T = \frac{1}{t}$, so that our curve becomes
\[
C: Txy^r = (x+1)(Tx+1).
\]
To desingularize the generic fiber, consider the affine curves
\begin{align*}
  C'&: Tvy^r = (v+1)(v+T) \\
  C''&: Tu = z^{r-1}(z+uT)(z+u) \\
  C'''&: Tw = z^{r-1}(wz+T)(wz+1).
\end{align*}
Now glue the 4 affine curves together via
\begin{gather*}
  v = \frac{1}{x} = \frac{z}{u} = wz \qquad y = \frac{1}{z} \\
  u = \frac{x}{y} = \frac{1}{vy} = \frac{1}{w} \qquad  z = \frac{1}{y}\\
  w = \frac{y}{x} = vy = \frac{1}{u} \qquad z = \frac{1}{y}
\end{gather*}
One checks that the resulting curve is generically smooth.

The special fibers on these charts are as follows:
\[
\begin{array}{lll}
  C & x + 1 = 0 & C_1 \\
  C' & v(v + 1) = 0 & C_2, C_1 \\
  C'' & z^r(z + u) = 0 & C_r, C_1 \\
  C''' & z^rw(wz+1) = 0 & C_r, C_2, C_1.
\end{array}
\]
The third column gives names for the components of the special fiber, given in order of the factors in the equation. For example, in $C'''$, $C_2$ is given by the equation $w = 0$. The components $C_1$ and $C_2$ each have multiplicity 1, while $C_r$ has multiplicity $r$. Let $P = C_1 \cap C_r$. Note that $P$ lies in the chart $C''$, but in no others. Let $Q = C_2 \cap C_r$. Then $Q$ lies in $C'''$ but in no other chart. One checks that neither $P$ nor $Q$ is regular. Therefore we need to blow up at both points.

\subsection{Blowing-up}
\label{sec:blowing-up}

\paragraph{Blow-up at $P$.}
\label{sec:blow-up-P}

We first blow up at $P$, which is given by $(u,z,T) = (0,0,0)$. Via the change of variables
\[
z = uv \qquad T = u\tau
\]
we obtain the arithmetic surface with equation
\[
\tau = u^{r-1}v^{r-1}(v + u\tau)(v + 1).
\]
This has special fiber given by $T = u\tau = 0$, with components
\begin{align*}
  \tilde{C_1}&: (v + 1 = b = 0) \\
  \tilde{C_r}&: (v = \tau = 0) \\
  E_P&: (u = \tau = 0).
\end{align*}
In the above, $\tilde{C_\cdot}$ is the strict transform of the corresponding component with the same multiplicities as before. The third component $E_P$ is the exceptional component, and has multiplicity $(r-1)$. The special fiber therefore looks like

[picture]

The intersection points are $P_1 = \tilde{C_1} \cap E_P$ given by $u=v+1=\tau=0$, and $P_2 = \tilde{C_r} \cap E_P$ given by $u=v=\tau=0$. Writing the equation for our surface as
\[
u^{r-1}v^{r-1}(v + u\tau)(v + 1) - \tau = 0
\]
and taking the $\tau$ derivative on the left, we obtain
\[
u^rv^{r-1}(v + 1) - 1.
\]
We now observe that this expression equals $-1$ at both $P_1$ and $P_2$. Therefore both points are regular and we have resolved the singularity at $P$.

\paragraph{Blow-up at $Q$.}
\label{sec:blow-up-Q}



\section{Component group}
\label{sec:component-group}



\bibliographystyle{halpha}
\end{document}